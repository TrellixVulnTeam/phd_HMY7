\begin{center}\textbf{Abstract}\end{center}

Current state-of-the-art approaches
to computational
speech recognition and synthesis
are based on statistical analyses
of extremely large data sets.
It is currently unknown how these methods
relate to the methods that the human brain
uses to perceive and produce speech.
In this thesis,
I present a conceptual model, Sermo,
which describes some of the computations that
the human brain uses to
perceive and produce speech.
I then implement three large-scale brain models
that accomplish tasks
theorized to be required by Sermo,
drawing upon techniques
in automatic speech recognition,
articulatory speech synthesis,
and computational neuroscience.

The first model extracts features
from an audio signal by
performing a frequency decomposition
with an auditory periphery model,
then decorrelating the information
in that power spectrum
with methods commonly used in
audio and image compression.
I show that the features produced
by this model implemented
with biologically plausible spiking neurons
can be used to classify phones in
pre-segmented speech with significantly
better accuracy than the features
typically used in
automatic speech recognition systems.
Additionally, I show that this model
can be used to compare auditory periphery models
in terms of their ability to
support phone classification of pre-segmented speech.

The second model
uses a symbol-like neural representation
of a sequence of syllables
to generate a trajectory of premotor commands
that can be used to control
an articulatory synthesizer.
I show that the model
can produce trajectories
up to several seconds in length
from a static syllable sequence representation
that result in
intelligible synthesized speech.
The trajectories reflect the
high temporal variability
of human speech,
and smoothly transition between
successive syllables,
even in rapid utterances.

The third model
classifies syllables
from a trajectory of premotor commands.
I show that the model is able to
classify syllables online
despite high temporal variability,
and can produce the same
syllable representations
used by the second model.
These two models can be connected
in future work in order to implement
a closed-loop sensorimotor speech system.

Unlike current computational approaches,
all three of these models are implemented
with biologically plausible spiking neurons,
which can be simulated with neuromorphic hardware,
and can interface naturally with artificial cochleas.
All models are shown to scale
to the level of adult human vocabularies
in terms of the neural resources required,
though limitations on their performance
as a result of scaling will be discussed.

\cleardoublepage
