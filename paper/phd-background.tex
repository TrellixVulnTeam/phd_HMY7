\chapter{Background}

??? this thesis draws upon and makes contributions
to what would traditionally be thought of a disparate fields:
phonetics and phonology from linguistics,
psychoacoustics from psychology,
knowledge representation
and artificial neural networks from computer science,
spiking neural networks and
synaptic plasticity rules from computational neuroscience,
dynamical systems and control theory
from systems design engineering,
concretely implemented in software
using principles from software engineering.
As such, not all relevant background can
be given equal coverage.
In this chapter, we assume
that the reader's background
in in computer science or engineering.
We therefore spend more time
reviewing background in linguistics
and neuroscience as they fall outside
of the domain of computer science and engineering,
and so can be considered the problem domain
that we will apply our assumed knowledge to.
???yikes reword

\section{Problem domain: human speech}

Human speech is studied in many disciplines
using a variety of scientific methods.
In this thesis, we limit our coverage of human speech
to the disciplines of psychoacoustics,
phonetics and phonology,
and auditory biology.

Psychoacoustics provides a quantitative account
of how the human auditory system
responds to incoming air pressure levels.
As such, it provides a method to verify
that our model responds to sounds
as humans do.
A primary assumption behind this work
is that the human auditory system
has evolved to be adept at processing speech;
psychoacoustics describes many of the ways
in which the human auditory system
manipulates sound,
which we aim to reproduce.???ugh this needs work

Phonetics and phonology
provide insight into the basic units of speech.
While a full conversational system
would draw upon all areas of linguistics,
we believe that phonetics and phonology
provides the most relevant insights
when assessing speech from the level of
incoming air pressure levels.

Auditory neurobiology describes both
the and physical substrate organization
of the system that we aim to emulate.
We do no aim to fully replicate
biological neurons,
but by using a simple approximation
of biological neurons,
we constrain our algorithmic choices
to those that could be implemented
in a biological system.
Similarly, by examining the organization
of the auditory brain structures,
we constrain the space of possible
network topologies
to those that match
a network that we know to be successful.

\subsection{Psychoacoustics}

\subsection{Phonetics and phonology}

Make sure to mention monophthong v. diphthong

??? Also talk briefly about conversational shadowing

\subsection{Auditory neurobiology}

\section{Prior modeling approaches}

In making early steps toward
an integrated speech recognition and synthesis system,
we are applying techniques
in artificial intelligence and control theory
to the speech domain reviewed in the previous section.
There is a long history of applying these techniques
to this domain;
in this section we review prior modeling efforts
and contrast them with the model we will describe
in future chapters.

\subsection{Auditory periphery modeling}

??? figure like izhikevic, with auditory model + efficiency?

??? summary table with phenomena captured, etc

\subsection{Automatic speech recognition}

\subsection{Speech synthesis}

\subsubsection{Articulatory speech synthesis}

??? summary table with different vocal tract models

??? summary table with different acoustic models

??? in the tables, also note available implementations
(so we can justify writing our own).
Include programming language in this

??? include online vs batch in table

\subsection{Speech motor control}

??? note that many art. synths consider this part of their
synthesizer (control model).
But we will consider it separately because
it is a primary contribution of this thesis.

??? Saltzman stuff on task dynamics
is highly related to what we want to do.
But with SPA stuff on top.

??? also hosung nam's work

??? also mention near the end that people haven't
yet connected the Saltzman task dynamics stuff
to the brain; that'll be one of our contributions

\subsection{Integrated recognition and synthesis systems}

??? Review DIVA model

\section{

\subsection{Vector symbolic architectures}

??? make links to knowledge representation in CS

??? give general background for the math in methods?

%% ~8-20 pages

%% - More than a literature review
%% - Organize related work - impose structure
%% - Be clear as to how previous work being described relates to your own.
%% - The reader should not be left wondering why you've described something!!
%% - Critique the existing work - Where is it strong where is it weak?
%%   What are the unreasonable/undesirable assumptions?
%% - Identify opportunities for more research (i.e., your thesis).
%%   Are there unaddressed, or more important related topics?
%% - After reading this chapter, one should understand the motivation for
%%   and importance of your thesis
%% - You should clearly and precisely define all of the key concepts
%%   dealt with in the rest of the thesis, and teach the reader what s/he
%%   needs to know to understand the rest of the thesis.
