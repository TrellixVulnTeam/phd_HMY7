\chapter{Background}

??? this thesis draws upon and makes contributions
to what would traditionally be thought of a disparate fields:
phonetics and phonology from linguistics,
psychoacoustics from psychology,
knowledge representation
and artificial neural networks from computer science,
spiking neural networks and
synaptic plasticity rules from computational neuroscience,
dynamical systems and control theory
from systems design engineering,
concretely implemented in software
using principles from software engineering.
As such, not all relevant background can
be given equal coverage.
In this chapter, we assume
that the reader's background
in in computer science or engineering.
We therefore spend more time
reviewing background in linguistics
and neuroscience as they fall outside
of the domain of computer science and engineering,
and so can be considered the problem domain
that we will apply our assumed knowledge to.
???yikes reword

\section{Problem domain: human speech}

Human speech is studied in many disciplines
using a variety of scientific methods.
In this thesis, we limit our coverage of human speech
to the disciplines of psychoacoustics,
phonetics and phonology,
and auditory biology.

Psychoacoustics provides a quantitative account
of how the human auditory system
responds to incoming air pressure levels.
As such, it provides a method to verify
that our model responds to sounds
as humans do.
A primary assumption behind this work
is that the human auditory system
has evolved to be adept at processing speech;
psychoacoustics describes many of the ways
in which the human auditory system
manipulates sound,
which we aim to reproduce.???ugh this needs work

Phonetics and phonology
provide insight into the basic units of speech.
While a full conversational system
would draw upon all areas of linguistics,
we believe that phonetics and phonology
provides the most relevant insights
when assessing speech from the level of
incoming air pressure levels.

Auditory neurobiology describes both
the and physical substrate organization
of the system that we aim to emulate.
We do no aim to fully replicate
biological neurons,
but by using a simple approximation
of biological neurons,
we constrain our algorithmic choices
to those that could be implemented
in a biological system.
Similarly, by examining the organization
of the auditory brain structures,
we constrain the space of possible
network topologies
to those that match
a network that we know to be successful.

\subsection{Phonetics and phonology}

- phonetics: describing the sounds we use in speaking.
  Relating to the way we produce them and the way they sound.
  phonology: more abstract parts of sounds; look at restrictions
  and regularities in a language by studying syllables;
  suprasegmental parts (i.e., prosody:
  things that extend over several segments (phonemes)).
  - phonemic transcription: just the phonemes. Phonetic transcription:
    includes diacritics and other marks to differentiate allophones.

  - pick a way to write phonemes (probably IPA, in [square brackets])

- vowels: aa, ii, uu extremes, blend together for other vowels

  ??? show IPA image

  these are pure vowels; there are also diphthongs which consist
  of a movement or glide from one vowel to another (8 in english).
  Then there are triphthongs, which glide from one vowel
  to another then another, without interruption.
  These are tough to voice and to recognize, as the middle
  vowel is barely there.

- larynx (vocal folds, glottis): big deal! differentiate between
  voiced and voiceless phonemes. Can be varied a lot
  to change air pressure, etc. Can vary:
  intensity (speaking vs shouting),
  frequency (low vs high pitch),
  quality (e.g., harsh, breathy, murmured, creaky)

- consonants: constrictions at some point in the vocal tract.
  Plosives: air compressed behind complete constriction then released;
  in English, p t k b d g. (p t k are voiceless, also known as fortis
  for strong; b d g are voiced or lenis for weak).
  Fricatives: narrow passage consonants, make hissing sounds;
  continuant because you can keep making them; e.g., in English, s, f.
  Affricates: begin as plosives, end as fricatives, with same articulator
  (homorganic); e.g., in English, ch, sh.
  Nasals: air escapes only from the nose; in English, n, m, ng.
  Approximants: l is a lateral approximant; r another approximant
  (retroflex in other languages); j(y) and w.

- Syllables: difficult to actually segment a sentence into syllables.
  Native listeners will do it differently. (My thought: syllables
  are a sort of necessary bridge between phonemes and words to make
  it faster to learn how to voice new words in a language. But,
  it's not a one-to-one mapping; there may be multiple syllable
  sequences that go from phoneme -> word, but they'll sound subtly different.)
  Syllables consist of a loud center with little or no obstruction
  of airflow; before and after the center, greater obstruction and/or
  quieter sound.
  Minimum syllable: usually V (or something like shh like shushing someone)
  Syllable w/ onset: CV, CCV
  Syllable w/ coda: VC, VCC
  Syllable w/ onset and coda: CVC CCVCC, etc
  No word begins with more than three consonants.
  A small number of words end with four consonants; none more than four.

- Include? P. 71 syllable structure in English

- Also P. 71 with onset -> rhyme (which is peak -> coda)

- Syllables can be strong or weak (phonetics).
  Weak syllables are shorter, less sound, and have some vowels
  that can't exist in strong syllables. Can also have syllables
  with no vowel (e.g., /l/ in bottle), called syllabic consonant.
  In English, schwa often stands in for other vowels in weak syllables.
  Most vowels show up only (or at least most often) in either strong
  syllables (long vowels, diphthongs) or weak syllables (schwa).

- Stress: in production, this seems like more muscles are used;
  this corresponds to scaling of our trajectory perhaps?
  In perception, they are prominent because:
  1) louder (at least, that's how it feels, though loudness is
  difficult to detect in isolation)
  2) longer
  3) different pitch than other syllables
  4) different quality (e.g., using a vowel associated with strong
     syllables rather than weak; using a bunch of schwas will make
     the other vowels sound more prominent, for example).
  Seems to be three levels of stress (primary, secondary, unstressed).
  In some levels, stress is rule-based (e.g., last syllable in French).
  Not so in English. Only strong syllables can be stressed;
  other rules apply, but there are exceptions, so for some words
  stress is part of the representation. Also matters what part of speech...
  look, it's complicated. ``Generative phonology'' has more to say
  about how this might occur, but in general, we'll leave stress
  at this simplistic level and not concern ourselves with it.

- Rhythm: Stress-timing: English is stress timed?
  Evidence isn't strong.
  French and others are syllable-timed.
  Foot: a unit of rhythm. Starts with stressed syllable, includes
  all other syllables until the next stress.

- Pronunciation is way different in connected speech versus
  careful pronunciation in isolation. E.g.,
  Assimilation: phonemes in a syllable change due to the phoneme
  directly after or before.
  Elision: Sometimes sounds disappear; deleted phonemes.

- Intonation: Mostly pitch based. Relative pitch of course, we're
  looking for contrasts. And we're only interested in pitch
  changes under the speaker's control.
  Tone: the overall pitch behavior of an utterance.
  Rarely use level tone for one-syllable utterances.
  Falling tone for definite, final utterances;
  rising tone for questioning.
  (Obs: not really any rules about how to use tone;
  mostly learned through experience, kind of like phonemes
  I guess?) There are tone languages also.
  Fall-rise tone is common (e.g., reluctant agreement),
  rise-fall less common (strong approval, disapproval, surprise).
  Anything else is not commonly found.
  Note: ordinary speech usually occurs in the lower part
  of the speaker's pitch range. Emphatic falling tone
  starts higher, doesn't go lower.

- Tone-unit: the length of a tone sequence. (form)
  Shortest would be one syllable. But, often the hierarchy is:
  utterance -> one or more tone-units -> one or more feet ->
  one or more syllables -> one or more phonemes
  (note: we would like a few hierarchy levels, but we can say
  that building additional hierarchy levels is just more of
  the same and therefore not necessary in this thesis).
  Tonic syllable is necessary in tone-unit (like vowel in a syllable).
  Tone-unit is: pre-head -> head -> tonic syllable -> tail.
  All but tonic syllable are optional.
  Pitch change usually happens on tonic syllable.
  On rise/fall, tail continues the rise/fall.
  Difficult to segment tone-units.

- Function of intonation: 1. Expression of emotions and attitudes.
  2. Produces effect of prominence on stressed syllables.
  3. Exaggerates grammar and syntactic structure for listener.
  4. Clues listener into new and given information.

- Components of intonation: sequential, prosodic, paralinguistic.
  Sequential is the sequence of pre-head, head, tonic syllable, tail,
  pauses, and tone-unit boundaries. Prosodic include the
  width of pitch range, key, loudness, speed, and voice quality.
  ``Constrasts among prosodic components should be seen as relative
  to these `background' speaker characteristics.''
  Paralinguistic is body language.

- Autosegmental intonation: all intonation is just
  H (high tone) or L (low tone). A fall is HL; rise is LH.
  Stressed (accented) syllables have a * following them.
  \% marks major tone-unit boundaries.
  E.g., it's time to leave -> H* H*L\%.

???brief intro, explain the difference
between phonetics and phonology
% http://www.phon.ox.ac.uk/jcoleman/PHONOLOGY1.htm
???include phonemic vs phonetic transcription

Our treatment of phonetics and phonology
focuses on the hierarchical levels
in human linguistic expression,
paying particular attention to two
sets of characteristics in each level
and in each mapping between two levels:
temporal patterns and constraints,
and the existence of rule-like regularities.
For example,
if our model were to deal with sentences,
it would be important to know that,
temporally, sentences occur sequentially,
and cannot co-occur.
There are grammatical rules
about what words can be in a sentence,
and in what order they can be arranged,
but there are few rules placing constraints
between sentences---any sentence \textit{could}
follow any other sentence,
though there may be semantic issues.
Similar constraints exist at
the lower levels of language that
phonetics and phonology examine,
and these constraints will
inform the structure of the model
we present in Chapter~???ref.

The specific hierarchy that we will
present in this section is pictured
in Figure~???ref.
We will arbitrarily begin at
the lowest (rightmost???) level and work upwards
(though a presentation starting at
the highest level would be equally valid).
For clarity, we will use English for examples,
except when demonstrating that a particular
quality differs across languages.\footnote{
  When not explicitly cited,
  the information in Section~???ref
  should be considered
  basic phonetics and phonology
  that would be covered in an
  introductory undergraduate level
  class or textbook
  (I used ???refRoach as a general reference).}

??? talk about gestures here first,
as part of articulatory phonology,
cite Bernd.

The smallest linguistically relevant unit
is the phoneme, which describes
a short sound produced by one or more
vocal tract gestures
on the order of tens of milliseconds.
Phonemes are noted by their ability
to change the meaning of some word
when swapped in speech;
for example, the only difference between
the word ``bad'' and ``mad'' is the
consonant sound at the beginning,
and therefore those two consonant sounds
are each separate phonemes.
Even though different individuals
voice each phoneme differently,
the important quality is that
the particular sound is recognized
as a particular phoneme
in the context of an utterance.
A helpful analogy can be made between
phonemes in speech
and alphabetical letters in writing.
With no knowledge of English,
seeing the sentence,
``A bird has a wing,''
one might think that ``A'' and ``a''
are different letters.
However, with enough examples,
one could surmise that in every situation
where ``A'' is used, it appears at the
start of a sequence of letters,
and it could have instead
been replaced by ``a''
had it not appeared at the start of the sequence.
Therefore, ``A'' and ``a'' represent
the same underlying ``letter.''
Similarly, despite individual differences
in the pitch, speed, volume, and quality
of how one voices a particular phoneme,
that phoneme is still considered the same
if it plays the same role
in a linguistically relevant sequence of phonemes.

Generally, there are two types of phonemes:
vowels and consonants.
Vowels are longer sounds
made when the vocal tract is mostly open.
Consonants are shorter sounds
made when some part of the vocal tract
is constricted or transiently closed.
In almost all languages,
there are more consonant phonemes
than there are vowel phonemes,
though pronunciation varies significantly
between dialects,
and transcribing the full set of
phonemes in a language is
not a purely objective exercise.
For example, while most dialects
of English recognize 24 consonant phonemes,
General American English has been transcribed
as having 16 vowel phonemes ???citeWikipedia?,
while Received Pronunciation English
has been transcribed as having 25 vowel phonemes
???citeWells1982.

???vowels

???consonants

???talk about IPA % https://www.tug.org/TUGboat/tb17-2/tb51rei.pdf

??? Generative phonology?

??? Note: constraints and rules aren't necessarily all going to
be taken into account. However, we do want to be sufficiently
flexible or inflexible such that were we to take into account
all rules, we would still scale to biological limits.

\subsection{Psychoacoustics}

\subsection{Auditory neurobiology}

\subsection{Other?}

??? Also talk briefly about conversational shadowing

\section{Prior modeling approaches}

In making early steps toward
an integrated speech recognition and synthesis system,
we are applying techniques
in artificial intelligence and control theory
to the speech domain reviewed in the previous section.
There is a long history of applying these techniques
to this domain;
in this section we review prior modeling efforts
and contrast them with the model we will describe
in future chapters.

\subsection{Auditory periphery modeling}

??? figure like izhikevic, with auditory model + efficiency?

??? summary table with phenomena captured, etc

\subsection{Automatic speech recognition}

\subsection{Speech synthesis}

\subsubsection{Articulatory speech synthesis}

??? summary table with different vocal tract models

??? summary table with different acoustic models

??? in the tables, also note available implementations
(so we can justify writing our own).
Include programming language in this

??? include online vs batch in table

\subsection{Speech motor control}

??? note that many art. synths consider this part of their
synthesizer (control model).
But we will consider it separately because
it is a primary contribution of this thesis.

??? Saltzman stuff on task dynamics
is highly related to what we want to do.
But with SPA stuff on top.

??? also hosung nam's work

??? also mention near the end that people haven't
yet connected the Saltzman task dynamics stuff
to the brain; that'll be one of our contributions

\subsection{Integrated recognition and synthesis systems}

??? Review DIVA model

\section{Methods used? Maybe move?}

\subsection{Vector symbolic architectures}

??? make links to knowledge representation in CS

??? give general background for the math in methods?

%% ~8-20 pages

%% - More than a literature review
%% - Organize related work - impose structure
%% - Be clear as to how previous work being described relates to your own.
%% - The reader should not be left wondering why you've described something!!
%% - Critique the existing work - Where is it strong where is it weak?
%%   What are the unreasonable/undesirable assumptions?
%% - Identify opportunities for more research (i.e., your thesis).
%%   Are there unaddressed, or more important related topics?
%% - After reading this chapter, one should understand the motivation for
%%   and importance of your thesis
%% - You should clearly and precisely define all of the key concepts
%%   dealt with in the rest of the thesis, and teach the reader what s/he
%%   needs to know to understand the rest of the thesis.
