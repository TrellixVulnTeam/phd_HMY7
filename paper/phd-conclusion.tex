\chapter{Conclusion}
\label{chapt:conclusion}

In this thesis,
I proposed a conceptual model
called Sermo which
provides a biologically constrained framework
for a computational closed-loop speech system.
I implemented three models
with biologically plausible spiking neurons
that span aspects of speech
relating to perception, production,
and the interface of the two,
sensorimotor integration.

The first neural model
extracts acoustic features
from speech signals
using auditory periphery models
and connections between neural ensembles.
The feature vector,
called neural cepstral coefficients (NCCs),
performs significantly better than
a feature vector commonly used
in automatic speech recognition
on a phone classification task
using pre-segmented speech.
Additionally, the model is able
to do apples-to-apples comparisons
between five auditory periphery models
in terms of how well they
process pre-segmented speech samples
for classification.
We found that the
most complicated (Tan Carney) model
and the least complicated (Gammatone) model
performed the best.
The general trend is that
more complicated models
perform better
(with the exception of the Gammatone model).

The second neural model
sequences a list of syllables
and generates production information trajectories
from those syllables,
which can be synthesized by an articulatory synthesizer.
The trajectories generated were
significantly closer
to the target trajectories than
random syllables,
and synthesized speech was intelligible
in 35\% of cases,
and at least somewhat intelligible in 75\% of cases.
Scaling the model to the level of
an adult vocabulary would take up
only 0.1637 cm$^3$ of cortex,
though in its current state,
the model does not perform well
when scaled up
to the size of an adult syllabary.

The third neural model
recognizes syllables from a continuous
trajectory of production information.
Classification accuracy reached
over 80\% in some experiments,
despite using no acoustic information.
The model also produced semantic pointers
for the recognized syllables;
the accuracy of the memory representation
reached 90\% in some experiments,
though chance values were 33\% due to
a small set of possible syllables.
The model is able to operate continuously
without needing to be reset,
but scales poorly to rapid utterances
and large repositories of syllables.

The conceptual and neural models
presented here
represent the first steps
in applying large-scale neural modeling techniques
to speech.
As such, we are optimistic that
future models can overcome
the scaling issues that these models
currently face,
and can result in a biologically constrained
closed-loop model of speech,
enabling computers to
have natural interactions with humans.
