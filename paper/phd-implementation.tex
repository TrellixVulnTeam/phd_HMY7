\chapter{Implementation}
\label{chapt:implementation}

In this thesis, we implement and evaluate
three neural models that make up
parts of the conceptual Sermo model.
Each model provides
an explanation for how the brain might
solve a task necessary for
speech recognition or synthesis
using the computational devices at hand,
namely spiking neurons.

\section{Neural cepstral coefficients (NCCs)}
\label{sec:ncc}

\probbox{A raw audio waveform has high temporal resolution,
  but low spatial resolution, as it is a one-dimensional signal.
  Audio must therefore be preprocessed such that
  temporal information in the recent past
  is projected into some feature space
  at the current moment in time.
  Automatic speech recognition systems preprocess speech by
  extracting feature vectors containing cepstral coefficients.
  How can a neural model like Sermo preprocess speech?}

The frontend of most automatic speech recognition systems
extracts a feature vector from the incoming audio waveform.
As reviewed in Section~\ref{sec:prev-mfcc},
Mel-frequency cepstral coefficients (MFCCs)
have been used successfully in many ASR systems.
To a certain extent, MFCCs are motivated
by the organization of the human auditory system.
We propose neural cepstral coefficients (NCCs)
as an alternative feature vector representation
for automatic speech recognition
and other tasks involving speech processing.
Unlike MFCCs, NCCs are both motivated
by the organization of the human auditory system
and are implemented the same basic computational units
as the human auditory system, spiking neurons.

\subsection{Model overview}

\fig{ncc}{0.4}{Neural Cepstral Coefficient pipeline.}{
  Pipeline for generating neural cepstral coefficients (NCCs),
  compared to MFCCs.
  While the NCC pipeline initially appears to be simpler,
  the missing components from the MFCC pipeline
  are instead incorporated into parts of the NCC pipeline.
  The logarithmic compression is part of the
  auditory periphery model.
  No explicit smoothing step is necessary,
  as signals will be smoothed and normalized
  in the NCC model by virtue of being represented
  by spiking neurons, and filtered by
  the synaptic filters between neurons.}

Figure~\ref{fig:ncc} shows the NCC processing pipeline,
as compared to the MFCC processing pipeline.
While the pipelines implement
essentially the same computations,
the manner in which those computations
are performed differs significantly.
The primary difference is that
in the NCC pipeline,
incoming audio is processed
in a continuous online fashion,
rather than in a frame-based fashion.
Typical MFCC extraction algorithms
break the audio signal into
overlapping fixed-length windows
called frames,
which are processed independently
until the smoothing stage,
where discontinuities
due to frame boundaries are
minimized.
The NCC extraction algorithm,
on the other hand,
maintains internal state
at each processing stage,
and updates that internal state
for each incoming audio sample.
As each internal state update
depends on the current state,
changes occur smoothly through time,
which obviates the need for
an explicit smoothing step.

Another implementation difference
is in the computation of
the inverse discrete cosine transform.
Typically, this computation is done as
\begin{equation}
  y_k = \frac{x_0}{\sqrt{N}} + \sqrt{\frac{2}{N}} \sum_{n=1}^{N-1}
  x_n \cos \left( \frac{\pi}{N} n \left( k + \frac{1}{2} \right) \right)
  \text{ for } 0 \le k < N,
\end{equation}
where $y_k$ is the $k$th cepstral coefficient,
$x_n$ is the $n$th auditory filter output,
and $N$ is the number of auditory filter outputs.
In matrix notation,
this equation can be expressed as
\begin{align} \label{idct}
  \V{k} &= \left[ 0, 1, \ldots, N-1 \right] & 1 \times N \text{ vector} \nonumber \\
  \V{s} &= \left[ \sqrt{2}, 1, 1, \ldots, 1 \right] & 1 \times N \text{ vector} \nonumber \\
  \V{T} &= \sqrt{2}{N} \, \V{s} \circ \cos \left( \frac{\pi}{N} \left(
    \V{k} + \frac{1}{2} \right) \otimes \V{k} \right)
    & N \times N \text{ matrix} \nonumber \\
  \V{y} &= \V{T}\V{x} & N \times 1 \text{ vector}
\end{align}
where $\circ$ is the Hadamard (element-wise) product,
and $\otimes$ is the outer product.
This rearranged equation
allows us to precompute a matrix, $\V{T}$,
that maps auditory filter outputs
directly to cepstral coefficients,
enabling us to embed this matrix
in the connection weights between
two ensembles
(see Figure~\ref{fig:ncc-network}).
By restricting $\V{T}$
to the first $M$ rows,
Equation~\eqref{idct} yields the first
$M$ cepstral coefficients;
typically, ASR systems
use $N\ge26$ auditory filter outputs
and $M=13$ cepstral coefficients.

\subsection{NCC neural model}
\label{sec:ncc-neural}

The NCC pipeline is implemented
in a Nengo network that uses
the Brian hears library
of auditory filter models
\citep{fontaine2011},
and two layers of
leaky integrate-and-fire (LIF) neuron ensembles
connected in a feed-forward manner
(see Figure~\ref{fig:ncc-network}).
While Brian has the capability
to simulate spiking neurons,
we only use Brian's implementation of
auditory periphery models,
which do not emit spikes.
These models,
summarized in Section~\ref{sec:periphery-models},
emulate the effect of the
audio waveform on the basilar membrane,
and the resulting inner hair cell activity
arising from basilar membrane deflections.
The auditory periphery model
also includes rectification and compression,
and therefore accomplishes a significant portion
of the MFCC extraction pipeline,
which is unsurprising as MFCC extraction
is designed to emulate certain aspects
of the human auditory system.

\fig{ncc-network}{0.8}{Neural Cepstral Coefficient network.}{
  Network diagram for the neural cepstral coefficient model.
  A bank of auditory filters processes incoming sounds,
  projecting their output to an ensemble
  for each auditory filter.
  Those ensembles are connected to ensemble
  representing cepstral coefficients
  through the linear $\V{T}$ matrix,
  which implements the inverse discrete cosine transform.
  Ensembles representing cepstral coefficients
  are projected to networks that compute
  the temporal derivative of the coefficient.
  Both the cepstral coefficient and derivative network outputs
  are concatenated together to form
  the final output feature vector.}

The simulation timesteps
for the Nengo model and the contained
Brian auditory filter model
are uncoupled.
The auditory filter model
is always run with the same timestep
as the auditory stimulus;
that is, its internal state is updated
for each audio sample.
The Nengo model, on the other hand,
runs at a timestep independent
of the auditory stimulus.
On each Nengo timestep,
therefore, one or more samples
of the audio input
is fed through the
auditory filter model.
The \textit{auditory nerve} ensembles
receive as input the current
state of the auditory filter model,
regardless of how many timesteps
the Brian model has advanced.

The \textit{auditory nerve} ensembles
mimic the activity of spiral ganglion cells
projecting down the auditory nerve.
For each characteristic frequency,
a heterogeneous population of neurons
is driven by the current output of
the auditory filter model.
Each characteristic frequency
drives an independent neural ensemble,
meaning that this layer of ensembles
can be thought of as being cochleotopically organized,
as the mean activity of an individual neuron
only gives information about the output
of a single auditory filter
(although that filter may be modulated
by power in nearby frequencies).
Together, the decoded output of
each ensemble gives an approximation of
the compressed power in the part of the spectrum
that the auditory filter is sensitive to.
This quantity is essentially
the output of the auditory filter
(i.e., the first three steps in the MFCC pipeline),
but now represented in
cochleotopically organized spiking neurons.
It is also the $\V{x}$ vector in
the inverse discrete cosine transform
(Equation~\eqref{idct}).

The \textit{cepstra} ensembles
each represent one cepstral coefficient.
The inverse discrete cosine transform
for that coefficient is computed
through the connections between
the auditory nerve ensembles
and the cepstral ensembles.
Specifically, the connection
between each auditory nerve ensemble
and each cepstrum ensemble
scales the decoded value of the
auditory nerve ensemble by
the corresponding value
the $\V{T}$ matrix.
Since this is a linear transform,
it is done by computing
normal decoding weights
according to equation~\eqref{eq:linear-trans}
and scaling the decoders by $\V{T}_{i,j}$.
Since the currents coming from
multiple connections to an ensemble
are summed, the cepstral ensembles
end up representing the
dot product between the row of
$\V{T}$ that corresponds to its cepstrum
and the $\V{x}$ vector,
which is collectively represented
by the auditory nerve ensembles.

The end result of the networks
is the decoded output
of the \textit{cepstra} ensembles,
which is a continuously varying,
automatically smoothed set of cepstral coefficients.
We hypothesize that these coefficients can be used
as the feature vectors in an
automatic speech recognition system.

\subsection{Delta neural cepstral coefficients}

Another common technique in designing frontends
for automatic speech recognition is to
include temporal derivatives
in the feature vector.
Appending MFCC derivatives
to the feature vector
(also known as MFCC velocities,
or Delta-Cepstral Coefficients [DCCs])
improves ASR accuracy.
Appending derivatives of the MFCC derivatives
(also known as MFCC accelerations,
or Double Delta-Cepstral Coefficients [DDCC])
further improves recognition,
though by a smaller margin than
the improvement from the first derivative
\citep{furui1986,kumar2011}.

Temporal derivatives in these systems
are typically computed through straightforward
finite differences.
That is, for an MFCC $y_k$
at frame $t$, the corresponding delta is
\begin{equation}
  \label{dcc}
  \delta_k(t) = \frac{\sum_{n=1}^N y_k(t+n) - y_k(t-n)}{
    2 \sum_{n=1}^N n},
\end{equation}
where $N$ is a small number of frames
(typically between one and three).
The double delta applies the same operation
to the $\delta_k$ vectors.

\subsection{Delta NCC neural model}
\label{sec:impl-deriv}

Implementing temporal differentiation
in spiking neural networks
is not as straightforward as implementing
Equation~\eqref{dcc} in neurons,
as the neural simulation operates
online in continuous time,
and therefore does not automatically
have access to the information
from previous timesteps.
However, there are several ways
to build networks that maintain
information from the past
and can therefore implement temporal differentiation.

\citet{tripp2010} describes six such models,
two of which use feedforward connections,
two use adaptive neuron models,
and two implement linear time-invariant (LTI) filters
through recurrent connections.
All six models successfully
compute a temporal derivative,
so for simplicity we implement
the two feedforward temporal differentiation models,
with the assumption that
more sophisticated temporal differentiation
models could be used if desired.

\fig{tripp}{0.8}{Feedforward temporal differentiation from \citet{tripp2010}.}{
  Two feedforward temporal differentiation models from \citet{tripp2010}.
  (Left) Implementing temporal differentiation
  with an intermediate ensemble.
  (Right) Implementing temporal differentiation
  with no additional ensembles,
  but two connections between two ensembles
  with different synaptic filters.
  See text for more details.
  Adapted from \citet{tripp2010}.}

In the first feedforward differentiation model,
an intermediate ensemble is introduced
between the ensemble representing the input signal
and the ensemble that we want to represent
the derivative of the input signal
(see Figure~\ref{fig:tripp}, left).
The \textit{out} ensemble
receives direct positively scaled input from
the \textit{in} ensemble,
and delayed negatively scaled input from
the \textit{d} ensemble.
The slight delay between the positive signal
communicated from the \textit{in}
and the negative signal
from the \textit{d} ensemble
results in the \textit{out} ensemble
representing the temporal derivative
of the input signal.

Information transmitted between two ensembles
is slightly delayed
because changes to the input currents
of the first ensemble
must first be integrated for one or more
timesteps before those changes can result in
spikes that influence
the second ensemble's input currents.
The amount of time between
when input signal changes are reflected
in the first and second populations
depends on several factors,
including the simulation timestep,
the neuron model, and the synaptic filter model.
In the feedforward differentiation model
with an intermediate ensemble,
all connections use a lowpass filter
with a relatively long time constant
($\tau=0.1 \text{ s}$)
as the synaptic filter model.\footnote{
  Synaptic filter time constants of around 100 ms
  are consistent with NMDA glutamate receptors,
  which \citet{young2003} have suggested
  are part of differentiator circuits
  underlying respiratory rhythm.
  Additionally, in terms of
  the biological plausibility of the network as a whole,
  \citet{tripp2010} note that this form of network
  has been observed in \textit{C. elegans}
  \citep{dunn2004} and may underlie chemotaxis.}

In the second feedforward differentiation model,
two connections are made between
the \textit{in} and \textit{out} ensembles.
The first positively scaled connection
uses a synaptic filter with a fast time constant
($T_1=0.005 \text{ s}$),
and the second negatively scaled connection
uses a synaptic filter with a slow time constant
($T_2=0.1 \text{ s}$).\footnote{
  Synaptic filter time constants of 5 ms
  are consistent with some types of
  AMPA receptors \citep{jonas1993}.}
Like the first model,
this approach exploits the delays introduced
by synaptic filtering,
subtracting a more slowly changing signal
from a signal changing quickly,
resulting in a temporal derivative.
While this network has the advantage
of requiring fewer neurons
as no intermediate connection is introduced,
it adds the constraint that
the neurons in the \textit{out} ensemble
contain both fast and slow
neurotransmitter receptors.\footnote{
  Since AMPA and NMDA are both glutamate receptors,
  no constraint is placed on the \textit{in} ensemble.
  \citet{tripp2010} note that a derivative circuit
  in the vestibular system of oyster toadfish
  \citep{holstein2004}
  is likely to match the form of the second
  feedforward differentiation model.}

These differentiator networks
are added to the NCC network
(see Figure~\ref{fig:ncc-network})
to compute the desired derivative.
The cepstral coefficients and derivatives
are finally concatenated togther
in a final step to produce
the NCC feature vector.

\section{Syllable sequencing and production}

\probbox{Once the intention to voice a syllable sequence
  has been formed, the motor system must translate
  a sequence of conceptual syllable representations
  to a continuous trajectory of motor commands
  that will cause muscle activations in the vocal tract.
  How can a neural model like Sermo generate
  a motor command trajectory from
  a symbolic syllable sequence representation?}

We base our model on the linguistic theory
that the speech production system
is organized at the level of syllables,
and that production information trajectories
are explicitly stored
for syllables that are frequently voiced.
In this neural model,
we describe how sequences of syllables are represented
and iterated through with specific timing,
and how those representations
generate production information trajectories
that can control an articulatory synthesizer.\footnote{
  Parts of the model described in this section were created
  in collaboration with Bernd Kr\"{o}ger.
  A preliminary version of this model was presented
  in \citet{kroger2014}.}
We employ several modules in the semantic pointer architecture
to iterate though the syllable sequence,
and use DMPs to generate production information trajectories.
We propose a novel combination of an associative memory
and a set of DMP networks
that we call a
\textit{temporal output associative memory}.

\subsection{Model overview}
\label{sec:impl-prod-overview}

A sequence of syllables is represented
by a semantic pointer
that tags each syllable with its place
in the list; i.e.,
\begin{equation}
  \text{SEQ} = \text{SYLL1} \bind \text{POS1} \oplus
    \text{SYLL2} \bind \text{POS2} \ldots
\end{equation}
For example, the word ``Sermo''
(\ipa{/s3rmoU/})
would be represented as
\begin{equation}
  \text{SERMO} = \text{S3R} \bind \text{POS1} \oplus
    \text{MOU} \bind \text{POS2}.
\end{equation}
Representing a syllable sequence
by binding syllables to list positions
allows us to query the sequence
for its constituent syllables,
as was done in \citet{choo2010,eliasmith2012}.
For example, to query the model
for the first syllable,
we can bind the sequence semantic pointer
with the inverse of the first position pointer.
\begin{equation}
  \text{S3R} \approx \text{SERMO} \bind \text{POS1}^{-1}
\end{equation}

The semantic pointers representing syllable positions
are constructed such that subsequent positions
are the result of binding a position with
a randomly generated $\text{NEXTPOS}$ pointer.
For example,
\begin{equation}
  \label{nextpos}
  \text{POS2} = \text{POS1} \bind \text{NEXTPOS}
\end{equation}
The initial position, $\text{POS1}$,
and semantic pointers for each syllable
are also randomly generated.\footnote{
  Syllable semantic pointer could be constructed to
  reflect how the syllable is produced; e.g.,
  it could have a representation like
  $\text{BA} = \text{B} \bind \text{ONSET} \oplus
    \text{A} \bind \text{NUCLEUS}$,
  where semantic pointers for the individual phonemes
  and syllable parts are randomly generated.
  However,
  as we will show in the remainder of this section,
  we do not need this information to
  produce syllables.
  Therefore, we deliberately do not impose
  any structure on the syllable semantic pointers
  in this model because we anticipate
  that higher-level linguistic subsystems in Sermo
  will dictate the features of the syllable
  that must be contained
  in the syllable semantic pointer representation.}
Throughout the simulation,
the current syllable to be voiced
is computed by binding the sequence
with the current position pointer.
\begin{equation}
  \text{SYLL} = \text{SEQ} \bind \text{CURRPOS}^{-1},
\end{equation}
where $\text{CURRPOS}$ starts as $\text{POS1}$
and changes over the course of the simulation
by binding with the $\text{NEXTPOS}$ pointer
when each syllable has been voiced.

It should be noted that
this choice of representation
sets a soft upper bound
on the number of syllables in a sequence.
In \citet{eliasmith2012},
it was shown that a similar network
could represent lists with six elements accurately;
elements in the middle of
lists with seven or more elements
were sometimes unable to be queried.
However, we assume that the
linguistic systems producing
syllable sequences to be voiced
will only produce sequences
of up to three syllables
at a time.

Each syllable semantic pointer
is associated with a
trajectory generation subsystem
based on dynamic movement primitives
(DMPs; \citealp{schaal2006,dewolf2015}).
DMPs allow for reliable trajectory generation
that can be temporally scaled,
allowing for syllables to be voiced
quickly or slowly.
Each DMP generates a trajectory
of vocal tract gestures,
which are then mapped to
motor commands that cause
muscle activations in the vocal tract.

Traditionally, DMPs can be either discrete or rhythmic,
depending on whether the trajectory
is followed repetitively.
Since syllables are voiced
one after another in typical speech,
a discrete DMP would seem to be
the obvious choice;
however, it is possible for
the same syllable to be voiced
twice in a row,
and some forms of singing
feature rhythmic repetitions
of the same syllable,
necessitating the use of rhythmic DMPs.

Using rhythmic DMPs
means that when a syllable
is only voiced once,
it must be explicitly stopped
once the DMPs traverses
its limit cycle once.
At the same time,
the DMP for the next syllable must begin,
or have already begun.
We accomplish this precise timing
by detecting when the current syllable
is nearing completion
and starting a control network
that transitions smoothly
between the currently active syllable
and the next active syllable.
When a syllable is repeated,
the end result should appear as though
the limit cycle continued undisturbed.
The implementation of this network
depends heavily on neural inhibition,
and therefore will be discussed
in Section~\ref{sec:impl-prod-neuralmodel}.

As reviewed in Section~\ref{sec:methods-dmp},
DMPs typically consist of a point attractor
and a forcing function.
Currently, we do not implement
separate point attractors for each DMP,
and instead assume that all DMPs
have the same point attractor,
whose stable point is
to have no active vocal tract gestures.
The forcing function for each DMP
generates a sequence of temporally coordinated
vocal tract gestures
that are defined manually.\footnote{
  We assume that these vocal tract gesture sequences
  would be learned through babbling and imitation phases,
  as are modeled in DIVA and the Kr\"{o}ger model.
  We leave the incorporation of this learning process
  into the Sermo model as future work.}

\subsection{Neural model}
\label{sec:impl-prod-neuralmodel}

\fig{prod-network}{0.8}{Syllable sequencing and production network.}{
  Network diagram for the syllable sequencing and production model.
  (A) The sequence network iterates through
  syllable positions in interconnected
  working memory modules,
  which are switched through a connection from
  the sequencer network.
  Syllable positions are bound with the
  sequence and cleaned up through
  associative memories to represent
  the current and next syllable targets.
  DMPs assocaited with those syllable targets
  are disinhibited by the associative memories.
  (B) The sequencer network receives input
  from all DMPs in order to keep track
  of the timing of the current syllable.
  When the current syllable is nearly finished,
  the reset ensemble is activated,
  resetting all disinhibited DMPs to
  their initial position.
  Working memory gates are switched
  when the reset ensemble activates
  and deactivates.
  When the reset ensemble is active,
  the timer is advanced through a
  recurrent connection with a
  separate ensemble, which is normally inhibited.
  (C) DMP networks are primarily composed of
  an oscillator which represents
  the two-dimensional canonical system state.
  Production information trajectories
  are decoded from the system state
  and projected to the production information network.
  Input from associative memories
  disinhibit the oscillator;
  input from the reset ensemble in the
  sequencer reset the oscillator to an initial position.
  The DMP must be both disinhibited and
  driven to the initial position
  to generate a production information trajectory.
  (D) Production information network.
  Aggregates information from all DMP networks.}

SPA modules implemented in Nengo are used
to robustly iterate through
the syllable sequence list.
The input sequence is represented
in a \textit{state} module,
which does not modify the sequence.
The current and next position pointers
is tracked by two interconnected
\textit{working memory} modules,
as was done in \citet{eliasmith2012}.
The current syllable is computed
by binding the \textit{state}
with the inverse of the
current pointer represented
in \textit{working memory};
similarly for the next pointer,
which is stored in another
working memory module.
Both the current and next syllable
representations are cleaned up with
\textit{associative memories}.
While the cleaned up semantic pointer
associated with syllables are available,
they are not used;
instead, the activity of the ensemble
that is associated with the syllable
disinhibits the oscillator
associated with a DMP network
whose forcing function
follows the production information trajectory
for that syllable.
A timing network that we call
a \textit{sequencer} ensures
that only one syllable DMP
is active at one time,
and that syllables transition
immediately and smoothly.
See Figure~\ref{fig:prod-network}
for an illustration.

\subsubsection{SPA modules}

Networks for representing semantic pointers
and doing common cognitively relevant
manipulations on those pointers
have been previously created
for Spaun.
This model uses those networks
as described in \citet{eliasmith2013}.

The binding modules compute
circular convolution,
implemented as multiplication
in the Fourier domain.
The design of these networks
is described in more detail in
\citet{choo2010}.

The working memory modules
maintain a memory trace
through recurrent connections
implementing an integrator.
While the general design of the working memory
is described in more detail elsewhere
\citep{choo2010},
how the two working memory modules
interact warrants further explanation.
In order to calculate the next position,
we require the current position
(see Equation~\eqref{nextpos}).
The straightforward approach
of computing this in a single recurrent loop
from one working memory to itself
fails because one of the two pointers
being used in the computation
is also the result of the computation.
To remedy this situation, we exploit the fact that
each working memory module is gated
such that it operates in two regimes.
In the ``acquisition'' regime,
the memory module acquires a state to remembered.
In the ``maintenance'' regime,
the memory module ignores incoming state information,
and maintains a memory of the last state
encountered in the first regime.
The two working memory modules
are connected to each other,
and are always in the opposite regime.
Initially, the first memory
acquires the $\text{POS1}$ state
from an external source.
The gating signal turns on,
which sets the first memory
into the maintenance regime,
representing the $\text{POS1}$ state.
This simultaneously sets the second memory
into the acquisition regime,
where it acquires the state
being maintained by the first memory
($\text{POS1}$ initially).
The second memory is connected
to the input of the first memory
such that it computes the
next position in the connection,
and therefore is providing
$\text{POS2}$ as input to the first memory,
though it is currently ignoring it.
When the gating signal is released,
the two memories switch regimes.
The second memory maintains
its memory of $\text{POS1}$
and therefore projects $\text{POS2}$
to the first memory,
while the first memory acquires
the $\text{POS2}$ state from the second memory.
This process repeats for each position
in the syllable sequence;
each time the gating signal
is turned on and off,
the position pointer
in the first working memory increments.

This method of incrementing the position pointer
is both robust and fast.
It takes less than 50 ms to acquire
a new memory state,
and that state can be maintained
for several seconds.
Since syllables take
on the order of hundreds of milliseconds
to be voiced,
the circuit can go through a syllable sequence
fast enough for natural conversation.
However, one aspect of the circuit as described
poses a timing challenge.
As discussed in Section~\ref{sec:impl-deriv},
there is a slight transmission delay
when sending a signal between two ensembles.
Yet, humans are capable of producing
continuous speech with
few to no discernible pauses between syllables.
Therefore, the next syllable to be voiced
must follow the current syllable very quickly,
even overlapping when appropriate.
The \textit{sequencer} network,
discussed below, facilitates this.

The associative memories
map possibly noisy input pointers
to clean output pointers.
While the input pointers
can be associated with arbitrary
output pointers,
in our case inputs and outputs
are the same syllable pointers;
this is called an autoassociative memory,
or a cleanup memory.
These memories contain an ensemble
for each semantic pointer
in the vocabulary;
the ensemble is responsible for computing
the similarity between
that semantic pointer
and some input signal.
This computation is carried out
through the connection between
the ensemble representing the input signal
and the ensemble representing
the similarity to the desired pointer.
The weights on this linear connection
are set to be the semantic pointer itself;
the encoding process (see Equation~\eqref{eq:encoding})
therefore computes the dot product
between the desired semantic pointer
and the input signal.
Since semantic pointers are not necessarily
orthogonal to each other,
more than one ensemble may activate
for a given input signal.
In order to clean up to a single
output semantic pointer,
all of the ensembles
representing semantic pointer similarity
mutually inhibit one another.
As long as the semantic pointers
are sufficiently dissimilar
to each other,
this mutual inhibition
implements a winner-take-all function,
resulting in only a single ensemble
being active: the ensemble
representing similarity for the
semantic pointer most similar to the input signal.

\subsubsection{DMP networks}
\label{sec:impl-dmp}

The rhythmic DMP networks
generate repetitive
production information trajectories.
The networks are based on
the general design
presented in \citet{dewolf2015},
with some modifications
to enable interactions between
the DMP networks
and the SPA modules
by way of the sequencer network.

The core of the network
is an oscillatory ensemble
that represents the system state.
As the oscillator traverses
the limit cycle,
the multidimensional forcing function
is decoded from the system state,
resulting in the predefined
production information trajectory
playing out as
the oscillator's angle moves from $0$ to $2\pi$.
The ability to drive the oscillator
at different frequencies
enables the DMP to generate trajectories
at different timescales.

The forcing function of each DMP
interpolates a vector representation of
a manually defined
vocal tract gesture score over time.
The gesture scores are
designed to be used with the VocalTractLab
articulatory synthesizer
\citep{birkholz2006,birkholz2013}.
While a full speech production system
would also include a mapping from
the gesture score to
vocal tract parameter changes
(i.e., Sermo's motor expansion system),
we leave the neural implementation
of this mapping as future work,
and instead automatically generate
a gesture score from the neurally generated
vector representation over time,
which can then be synthesized
by VocalTractLab.

We have made two modifications
to the DMP network design
in \citet{dewolf2015}
to allow multiple DMPs
to operate in quick succession.
The first modification
is that the oscillator is inhibited
by default.
Usually, the oscillatory ensemble
has intercepts distributed away from zero,
causing a ``dead zone'' in the part of the
two-dimensional vector space represented
by the ensemble.
The dead zone forces the initial input
to the oscillator
to be strong enough
to kick it out of its dead zone;
however, it eliminates the problem
of the oscillator randomly
oscillating without receiving input.
While the DMP networks used in this model
still have a dead zone,
we also include additional explicit inhibition
because the signal that
starts the oscillator is provided
to all DMPs simultaneously.
All DMPs receive this signal
for efficiency and scalability reasons.
The inhibitory signal is inhibited
(and thus the oscillator is disinhibited)
by the syllable similarity ensembles
associated with the DMP
in the two associative memories,
which requires around 400 synaptic connections
if the memory and inhibitory ensembles
contain 20 neurons each.
The alternative would be to
set up a routing network
that would route the oscillator starting signal
to the correct oscillators depending on
the state of the two associative memories.
While such a routing network can be implemented,
it would take more neurons and connections
to implement than the approach used here,
and may introduce delays that
would be detrimental to smooth sequencing.

The second modification is a more robust
method for starting the oscillator.
Previously, we noted that the DMP's oscillator
requires a start signal to begin
the oscillation.
Typically, this signal is a short pulse
of driving force pushing the oscillator
toward a part of the state space;
in our case, we aim to push the oscillator
to $[-1, 0]$, or $\theta=\pi$.
With a short pulse of driving force,
the oscillator typically does not
make it to $[-1, 0]$ exactly,
as the intrinsic dynamics of the oscillator
quickly take over,
beginning the oscillation.
Additionally, the pulse
must be presented for a sufficiently short time
such that it does not affect
the oscillator state as it traverse
its limit cycle.

In order to overcome the limitations
of the usual starting pulse,
we instead continuously drive
the oscillator to the target state
until the starting pulse is released.
To implement this continuous drive,
we add an additional ensemble
to the DMP which receives input
from the oscillator
and provides output to oscillator;
the output to the oscillator
computes the difference between
the current oscillator state
and the target state,
which in our case is $[-1, 0]$.
Like the oscillator itself,
we only want this difference
to be computed when the start pulse
is active.
Therefore, we provide an
inhibitory signal to the differencing ensemble,
which is inhibited by
the starting pulse,
effectively disinhibiting the differencing ensemble
and driving the oscillator
to the $[-1, 0]$ state.\footnote{
  Note that we set to goal state of the
  differencing ensemble to be the target state
  plus a small vector that counteracts the
  intrinsic dynamics of the oscillator.
  Since those dynamics depend on the frequency
  of oscillation, the vector added depends
  on the DMP's frequency (i.e., speed).
  Currently, we use oscillators with
  fixed frequency and therefore add a fixed vector
  to the target state.
  If controlled oscillators are used,
  it would be ideal to also modify
  the vector accordingly,
  though the gain in accuracy may not be significant.}
Since the differencing signal is active
as long as the starting pulse is active,
the start pulse effectively ``resets''
the oscillator to the target state
until the pulse is released,
at which point it starts
traversing the limit cycle.
We therefore overcome the issues with
the starting pulse,
as the oscillator will be driven
to the target state even
with the intrinsic dynamics,
and the timing pulse can
last as long as is needed,
though its timing is still critical
for natural speech trajectories.

\subsubsection{Sequencer network}

The structure of the SPA modules
and the modifications to the DMP networks
described to this point
allow the two associative memories
to disinhibit the rhythmic DMPs
associated with the current and next syllables.
All that remains is to generate
the timing signals that switch
the regimes of the working memories,
and that drive any disinhibited DMPs
to the target initial state.
The sequencer network
accomplishes these two functions.

The main idea of the sequencer network
is that we keep track of
how far along we are in the
production of a syllable,
and when we are sufficiently
close to being done producing a syllable,
we initiate the starting pulse
and transition to producing
the next syllable.
Note that this assumes that
only one syllable can be the
``active'' syllable at any one time,
but syllables can still
blend into one another
during the time of
the starting pulse.

The core of the sequencer network
is a \textit{timer} ensemble
that receives input from
all of the rhythmic DMP ensembles.
If the assumption that
at any given time
only one syllable is active holds,
then the timer ensemble effectively
represents the state of the oscillator
for the currently active syllable.
The timer is used to
switch the regimes of the working memories,
and to initiate the starting pulse
for the next syllable.
As it turns out, the timing
of these two signals is strongly linked;
the ideal time to switch the regimes
of the working memories
is right after the timing pulse starts,
and right after it ends.
This timing is ideal because
we want both associative memories
to emit the ``next'' syllable
during the starting pulse,
but for the ``current'' and ``next'' syllables
to be disinhibited for the rest
of the time so that when the next
starting pulse occurs,
the next syllable is already dinsinhibited
and ready to be voiced,
rather than needing to be disinhibited
before it can be driven by the starting pulse.

Note, however, that we aim to
have a slight delay between
when the timing pulse starts
and when the working memory switch occurs,
as the switch will effectively
inhibit the ``current'' syllable,
which we wish to be active
while the next syllable is being
driven to the initial state.
Since the current syllable's trajectory
is bringing it near the initial state,
the additional drive resulting
from the starting pulse
does not adversely affect the
trajectory.\footnote{
  If the trajectory is adversely affected,
  then the forcing function can be modified
  such that the predefined trajectory is
  warped in time to accommodate for the
  increased speed at the end of the trajectory.}
Similarly, we aim to switch
the working memories again
slightly before the starting pulse ends
to limit the amount of time
that the previous and current syllables overlap.
Unlike phonemes, syllables are not
typically coarticulated;
the overlap that we achieve by the
delay between the timing pulse
and the first working memory switch
is to reduce the delay between
successive syllables,
not necessarily to allow them
to coarticulate,
though some slight overlap is possible.

\subsubsection{Temporal output associative memory}

We refer to
the combination of an associative memory,
and DMP networks
for each input-output pair in the memory
as a temporal output associative memory.
Since we use sequences of syllables
as an input, we use the sequencer network
to control sequence timing;
however, temporal output associative memories
can also be used in other situations,
including those that allow for
motor synergies to be
flexibly weighted rather than using
an all-or-nothing strategy
as is done with syllables.
The key innovation here is
the association of
a possibly noisy semantic pointer
with an arbitrary time-varying trajectory.
Many applications of this kind of association
can be seen in human behavior;
here, we associate static syllable representations
with production information trajectories
that will drive a motor system
to produce that syllable over time.
A similar mapping could produce
other types of motor behavior
(e.g., a reach)
from static representations
of congitive intentions.

\section{Syllable recognition}
\label{sec:impl-recog}

\probbox{Speech production information is decoded
  from integrated auditory
  and non-auditory sensory information.
  Yet, we perceive the continuous production information
  trajectory as a single speech unit.
  How can we recognize a trajectory of production information
  as having been produced by a syllable target?}

Syllable recognition is in many ways
the inverse of syllable production.
Rather than expand a discrete, static representation
to a continuous, dynamic one,
we do the opposite and collapse
a continuous, dynamic representation
to a discrete, static one.
As such, we theorize that the
trajectory classification aspect of
the sensorimotor integration system
is also organized at the level of syllables.
We propose a syllable recognition network
using a novel technique based on DMPs,
which we call
inverse dynamic movement primitives
(iDMPs).
We use iDMPs to recognize production information trajectories,
and then connect these iDMPs
to associative memories
in order to produce semantic pointer representations
for recognized syllables.
We call the novel combination of
iDMPs and associative memories
\textit{temporal input associative memories}
to emphasize the similarities
between the syllable production
and recognition systems.

\subsection{Model overview}
\label{sec:impl-recog-overview}

The first part of the model
recognizes incoming production information trajectories
with inverse DMPs.
Recall that in the DMP framework,
the state of the system
is represented by $x(t)$,
and the motor trajectory
is generated by defining
a forcing function $f(x(t)) = \V{y}(t)$
that emits the motor trajectory
for that point in the state space.
In the discrete case,
the intrinsic dynamics
of the state act
as a one-dimensional ramp;
in the rhythmic case,
the intrinsic dynamics
of the state act
as a two-dimensional oscillator.

In inverse DMPs, we aim to learn the inverse function,
which maps from the motor trajectory
back to the system state that generated that
trajectory; i.e., we aim to approximate
$g(\V{y}(t)) = x(t)$.
Unfortunately, we cannot do this approximation directly
as there is no guarantee of a one-to-one mapping
from $x(t)$ to $\V{y}(t)$;
i.e., the same motor command can occur
at multiple points in the trajectory.
Therefore, we use an approach
inspired by HMMs and GVF
(see Section~\ref{sec:prev-classification})
to approximate this function.

The basic idea of iDMPs
is that we keep track of our
current estimate of the system state,
$\hat{x}(t)$,
and gate the dynamics
of the system state estimate
to progress only when the input
(i.e., an observation, $\V{O}(t)$)
is sufficiently close to what we believe
the observation should be
(i.e., a prediction $\hat{\V{y}}(t) = f(\hat{x}(t))$).
We assume that the
forcing function $f$ is known.
Then, we define a similarity function
to compare the observation to
the prediction; specifically,
\begin{equation}
  \label{idmp-similarity}
  s(\hat{\V{y}}(t), \V{O}(t)) =
    \frac{\hat{\V{y}}(t) \cdot \V{O}(t)}
    {\Vert\hat{\V{y}}(t)\Vert \Vert\V{O}(t)\Vert},
\end{equation}
where $\cdot$ is the dot product
and $\Vert\cdot\Vert$ is the L2-norm.

Recall that the intrinsic dynamics
of a discrete DMP are
\begin{equation*}
  \dot{x} = \alpha,
\end{equation*}
and the intrinsic dynamics
of a rhythmic DMP are
\begin{align*}
  \dot{x_1} &= -2 \pi f x_2 \\
  \dot{x_2} &= 2 \pi f x_1.
\end{align*}
In the iDMP,
we have estimates of the actual system state,
and we gate the dynamics
of the estimate by the similarity measure.
Given some similarity threshold
$th_s$, the discrete case of an iDMP
has dynamics
\begin{equation} \label{idmp-dynamics1}
  \dot{\hat{x}} =
  \begin{cases}
    0 & \text{if } s(\hat{\V{y}}(t), \V{O}(t)) < th_s \\
    \alpha & \text{if } s(\hat{\V{y}}(t), \V{O}(t)) \ge th_s,
  \end{cases}
\end{equation}
and the rhythmic case of an iDMP
has dynamics
\begin{align}
  \label{idmp-dynamics2}
  \dot{\hat{x_1}} &=
  \begin{cases}
    0 & \text{if } s(\hat{\V{y}}(t), \V{O}(t)) < th_s \\
    -2 \pi f x_2 & \text{if } s(\hat{\V{y}}(t), \V{O}(t)) \ge th_s,
  \end{cases} \nonumber \\
  \dot{\hat{x_2}} &=
  \begin{cases}
    0 & \text{if } s(\hat{\V{y}}(t), \V{O}(t)) < th_s \\
    2 \pi f x_1 & \text{if } s(\hat{\V{y}}(t), \V{O}(t)) \ge th_s.
  \end{cases}
\end{align}

Note that,
like other trajectory recognition algorithms,
each iDMP is associated with
a specific trajectory,
and therefore we require
as many iDMPs as we have syllables
in our vocabulary.
Also like the GVF algorithm,
iDMPs can be considered a ``tracking'' algorithm,
as the underlying state is continuously estimated.
Classification can therefore be done flexibly
depending on the characteristics
of the trajectories in the vocabulary.
Classification can be done when
any iDMP's state passes some classification threshold,
(e.g., when $\hat{x}(t) > 0.9$ in the discrete case).
It can be done on an ad hoc basis,
essentially querying the set of iDMPs
for which is mostly likely at that moment in time
(i.e., $\argmax{\hat{X}(t)}$).
It can also be done when one iDMP
is sufficiently farther along than
all other iDMPs.

It is important to note that
the $\alpha$ and $f$ parameters
used in the iDMP cannot be chosen
in the same way as in DMPs.
In DMPs,
the $\alpha$ and $f$ parameters
control how quickly
the DMP is moving through
the state space,
and therefore controls
the temporal scale
of the trajectory.
In the iDMP, however,
$\alpha$ and $f$ stay fixed,
and instead represent
the maximum speed that
we expect the iDMP to
move through the state space,
or perhaps slightly higher than
the maximum speed.
If $\alpha$ and $f$ are too low
and the estimated state
lags behind the actual state,
then the predictions will
diverge from the observations,
leading to poor tracking
(false negatives).
If $\alpha$ and $f$ are too high,
then the iDMP may advance
too far in the state space,
making an improper classification
(false positives).
Choosing a good $\alpha$ or $f$
is critical to the success of an iDMP.

While we believe that the iDMP
is well suited to a wide range
of trajectory classification problems,
it is especially useful for syllables
because it allows for
the onset of the syllable to
be the dominant factor
in differentiating between
another syllable with the same nucleus,
but no onset.
For example, if the input is a trajectory
from the syllable \ipa{[bAr]},
the iDMP for \ipa{[bAr]}
will be further along in the state space
than the iDMP for \ipa{[Ar]}.
However, an issue arises
for codas,
in that the iDMP for \ipa{[bA]}
will be farther along than
\ipa{[bAr]} when the \ipa{[r]}
is encountered.
We will refer to this as the
\textit{trajectory subpath} problem.
The best solution to the
trajectory subpath problem depends on
the characteristics of the trajectories
in the vocabulary.
In the case of syllables,
we solve the trajectory subpath problem
by introducing competition
between iDMPs with
common trajectory subpaths
(see \ref{sec:impl-recog-neural} for more details).

In the syllable recognition model,
a discrete iDMP is associated
with each syllable in the vocabulary.
Each syllable is also associated
with a randomly generated semantic pointer,
which should be the same semantic pointer
used by the syllable production system
to produce that syllable.

\subsection{Neural model}
\label{sec:impl-recog-neural}

\fig{recog-network}{0.8}{Syllable recognition network.}{
  Network diagram for the syllable recognition model.
  (A) The trajectory input consists of
  an ensemble for each possible gesture.
  One connection is made from each ensemble
  to each iDMP state ensemble.
  If the iDMP corresponds to a syllable
  that uses that gesture,
  then it projects to a dimension of that iDMP.
  If the iDMP corresponds to a syllable
  that does not sure that gesture,
  then it projects to the system state
  of the iDMP with a negative weight.
  (B) Each iDMP network is primarily composed of
  a large recurrently connected state ensemble.
  The state ensemble's recurrent connection
  predicts an observation given the current system state,
  and advances the state if the actual observation
  is sufficiently close to the prediction.
  The system state can be pushed toward its initial state
  if incorrect gestures are encountered.
  The classifier ensemble inhibits
  the state ensemble once a classification is made.
  State output is also projected to
  an associative memory.
  (C) The associative memory takes input
  from all iDMPs in order to emit
  a semantic pointer corresponding to
  classified syllables.
  The associative memory projects
  to a recurrently connected memory module
  that attempts to remember
  the last syllable encountered.}

The spiking neural implementation of
the syllable recognition model
is organized much like the production model,
but in reverse.
Each syllable has an associated
iDMP network that aims to recognize
that syllable.
The output of each iDMP is associated
with an ensemble in
an SPA associative memory,
which produces a cleaned up version
of the semantic pointer
associated with the recognized syllable.
The cleaned up version of the semantic pointer
is projected a memory module,
which remembers the recognized syllable
for a short while in order to
make it available to other networks.
Finally, a temporal classification
ensemble is used to monitor the
state of the associative memory
and reset the iDMP state ensembles
once a classification has been made.
See Figure~\ref{fig:recog-network}
for the overall structure.

\subsubsection{iDMP networks}

The iDMP associated with each syllable
is similar in structure to the DMPs
described in Section~\ref{sec:impl-dmp}.
The core of the iDMP network
is an ensemble that represents
the estimated state of the dynamical system.
A recurrent connection
from the state to itself
implements the dynamics
that updates the estimated state
based on the current observation
and prediction
(see Equations~\eqref{idmp-dynamics1}
and~\eqref{idmp-dynamics2}).

Despite the production network using rhythmic DMPs,
we use discrete iDMPs for syllable recognition.
A significant amount of effort is taken
in the production network to ensure that
the 2D oscillators are reset at the correct time.
With iDMPs, we cannot be as precise about time,
as we are estimating the system state;
it can even be advantageous
to be less concerned about exact time,
as it can allow the iDMP to classify a trajectory
before it is complete.
As such, we are essentially collecting evidence
to increase our confidence that the
observations come from the same syllable
that the iDMP is created from;
integrating that evidence over time
is naturally done with a one-dimensional integrator,
as is used in the discrete iDMP.
Additionally, since we do not need to emit
a classification immediately after
the last one (we must first accumulate
trajectory evidence)
we can safely spend some milliseconds
resetting all iDMP states
once a classification is made.

We implement the discrete dynamics
from Equation~\eqref{idmp-dynamics1}
in a single recurrent connection.
Typically,
Nengo models that compute the dot product
(as is done in Equation~\eqref{idmp-similarity})
do so by computing each 2D product separately,
as the product is a nonlinear function
that can be difficult to optimize
in high-dimensional spaces.
However, initial experimentation
showed that the exact dynamics equation
could be well approximated
in a single connection
as long as the observation
and prediction were reduced in dimensionality.
In general, this could be done
with any number of statistical dimensionality
reduction techniques;
however, in the case of production information,
the vector space already has
orthogonal basis functions
(there is no coupling between gestures),
and each syllable uses
relatively few dimensions
(usually between four and seven).
Therefore, for each syllable,
we only explicitly represent
the dimensions of the space
that are actually used for that syllable.
With this dimensionality reduction,
we found that a recurrent connection
specifying exactly Equation~\eqref{idmp-dynamics1}
was approximated well enough
to do trajectory recognition in most cases.

The fact that we implement the dynamics
in one recurrent connection
implies that the ensemble
represents both the system state
and the observation dimensions
of interest to that syllable.
In other words, the state can have
up to eight dimensions for complex syllables.
In our case,
even when high accuracy is required,
the number of neurons needed
is reasonable given some assumptions
about adult vocabularies;
see Section~\ref{sec:res-prod-scaling}
for more details.

In the previous section, we discussed
the \textit{trajectory subpath problem},
which occurs when two syllables
start with the same trajectory
but diverge near the end
(e.g., \ipa{[bA]} versus \ipa{[bAr]}).
In order to handle this problem,
we introduce an additional connection
in the model that explicitly punishes
gestures that are not
part of the iDMP's syllable.
One method to do this would be to
negatively weight any gestures
not part of the current syllable
in the recurrent connection;
however, reducing the dimensionality
of the observation space makes
this impossible.
Instead, we introduce a small \textit{reset} ensemble
in the iDMP that drives
the system state back to the initial state
whenever a gesture that is not part
of the syllable is active.\footnote{
  The reset ensemble is similar to the reset ensemble
  in the syllable production model
  in its implementation and use,
  in addition to its name.}
This ensemble is normally inhibited,
but an inhibitory connection is made
from the input signal to the reset ensemble
such that any gesture
not part of the syllable will
disinhibit the reset ensemble,
pushing the iDMP state back to the initial state.
As long as the target trajectories
are normalized to the same length,
punishing these gestures should
mitigate the trajectory subpath problem
in most cases.\footnote{
  It should be noted that humans
  are not necessarily adept at
  disambiguating trajectory subpath issues;
  consonant phonemes can be interpreted
  as being part of the coda of one syllable
  or the onset of the subsequent syllable
  in many syllable sequences.
  The form of syllable recognition
  explained by this model would
  be influenced by linguistic systems,
  or used primarily in sensorimotor situations
  like learning to voice new syllables
  and recognizing infrequently heard syllables
  in a foreign language.}

As an added benefit,
punishing gestures not in the sylalble helps
differentiate between syllables with
similar trajectory suffixes;
e.g., \ipa{[bAr]} versus \ipa{[dAr]}.
Without punishing extra gestures,
it is more likely that the iDMP
associated with \ipa{[dAr]} will catch up
to the \ipa{[bAr]} iDMP
and result in a misclassification
if the parameters in the model
are not well-tuned.

\subsubsection{SPA modules}

The associative memory is a standard
SPA associative memory,
mapping from a set of input semantic pointers
to a set of output semantic pointers.
In the recognition model, however,
we do not provide any semantic pointer input;
instead, we use the iDMP networks
to drive the ensembles associated
with each syllable semantic pointer directly.
Specifically, the dimension or dimensions
representing the system state
are connected to the similarity sensitive ensembles.

One benefit of using the associative memory
is that it separates the state tracking function
of the iDMP networks
from the winner-take-all mutual inhibition
of the associative memory,
which makes the iDMP network easier to implement.
Another important benefit is that
it provides a location for
sensorimotor information
(provided by the iDMPs)
to integrate with top-down
linguistic information.
Since neurons naturally sum
input from multiple sources,
the ensembles representing syllable similarity
can receive input from both sources,
and achieve faster or more correct classification
as a result.

The mutual inhibition and thresholding options
are critical to the proper functioning
of the associative memory
in the recognition model.
Figure~\ref{fig:recog-network} shows
the mutual inhibition in the associative memory network.
With mutual inhibition and thresholding is used,
the second layer of ensembles
in the associative memory
only become active if
the first layer of ensembles
(those that represent semantic pointer similarities)
are sufficiently active.
Thresholding is implemented by
setting ensemble properties in the second layer
to have encoders of $[1]$,
and x-intercepts that start around or above
the chosen threshold.
In doing so, the second layer of ensembles
will only become active if the
first layer ensemble associated with it
crosses that threshold.

In addition to the associative memory module,
which identifies a unique syllable semantic pointer
once the iDMP associated with it has crossed threshold,
the model contains a memory module
that maintains a decaying memory
of that pointer.
The memory is implemented as an ensemble
for each set of 16 subdimensions
of the semantic pointer
(which is 64 dimensional in this model).
Each ensemble is recurrently connected
with a weight less than 1
so that the syllable most recently classified
is remembered for a few hundred milliseconds,
which is around the amount of time that we expect
for the next classification to take place.
Using a recurrent weight
with weight less than 1
is critical in this case,
as if the memory used recurrent connections
with weight 1,
all of the syllable pointers
encountered in the past would be
superimposed on one another.
Therefore, we assume that
only the most recent syllable classified
is of interest to downstream systems.

\subsubsection{Temporal classification}

The final part of the model is
a small \textit{classifier} ensemble
that resets the iDMP states
when a classification is made.
It receives input from all of the
second (thresholded) layer of ensembles
in the associative memory.
When the classifier ensemble is active,
it directly inhibits
all of the iDMP state ensembles.

Like the thresholded layer,
the classifier ensemble's
x-intercepts are distributed
near or above a certain threshold;
critically, this threshold
is slightly above the threshold
in the associative memory.
The difference between these two thresholds
results in a slight delay between
when the associative memory
output becomes active,
and when the classifier ensemble becomes active.
The associative memory
must be active long enough to provide
semantic pointer input to the
memory module,
which will maintain a memory of that pointer
once the iDMP states are inhibited
by the classifier ensemble.

Since the classifier directly inhibits
all of the iDMPs, any neural activity in this population
will have a large deleterious effect
on the classification of the current syllable,
and so we aim to completely inhibit
the iDMP once the classifier becomes active.
However, since the classifier inhibits
an ensemble that provides it input,
its activity will quickly cease.
In order to ensure that its short burst of activity
is sufficient to completely reset
the iDMP states,
we change two parameters from their defaults.
First, we use synaptic time constants
that are longer than usual on the connections
from the associative memory to the classifier,
and from the classifier to the iDMPs;
specifically, the time constant is 100 ms,
which is consistent with some NMDA receptors
\citep{sah1990}.
Second, we use a different neuron model.
All other ensembles mentioned in this thesis
are composed of leaky integrate-and-fire neurons.
In the classifier ensemble,
we use an adaptive leaky integrate-and-fire neuron model
\citep{koch1998}.
While a true bursting neuron model
would be better,
the adaptive LIF at least will produce
more spikes given
sporadic input compared to constant current.

It is worth pointing out that the
classifier ensemble does not use the
iDMP reset mechanism to reset the state,
unlike the connection punishing
syllables that are not part of the gesture.
The primary reason is that the reset mechanism
is too slow to fully reset the state
in the span of time that the classifier ensemble
is active.
The punishment mechanism cannot use inhibition
because it must only affect the system state,
not the representation of the observations,
which are represented in the same ensemble.
When resetting the iDMP, however,
the observations are ignored anyhow,
so inhibiting the ensemble
is acceptable.

It is also worth explicitly stating
that this method of temporal classification
accumulates experience and classifies
once the estimated state of one iDMP
passes a threshold.\footnote{
  This type of decision making is reminiscent
  of the well known drift diffusion model
  that has been used to model
  human behavior in two-alternative forced choice tasks
  \citep{bogacz2006}.}
While this classification method
is straightforward to implement,
it may make arbitrary choices
when two syllables are similar,
as the representations resulting
from spiking neural networks are noisy.
More sophisticated classification methods
could be employed using
iDMPs to track estimated system states.
