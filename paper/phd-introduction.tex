\chapter{Introduction}

Speech is arguably the most important
medium through which humans communicate.
As the role of human-computer interaction
in modern society increases,
so does the need for computers
to recognize and synthesize speech.
Currently, state-of-the-art approaches
to computational
speech recognition and synthesis
are based on statistical analyses
of extremely large data sets.
While our understanding
of the neurobiology of speech has advanced
in recent years,
there currently exist no
models perceiving or producing speech
using neurobiological parts,
namely spiking neurons.
Spiking neurons are the basic building blocks
of the brain.
Each neuron takes input from many other neurons,
and with enough input,
experiences a spike in membrane voltage,
which results in neurotransmitter
delivery to downstream neurons.
The activities of many billions
of spiking neurons with trillions
of interconnections are responsible
for producing and processing
complex behaviors including speech.

Recent advances in large-scale neural modeling
enable spiking neuron models of speech.
The Neural Engineering Framework (NEF; \cite{eliasmith2004})
provides a framework to implement
dynamical systems with networks of spiking neurons.
The Semantic Pointer Architecture (SPA; \cite{eliasmith2013})
provides a method to connect these dynamical systems
to symbol-like representations.
In terms of speech,
dynamical systems enable processing of
temporally varying sensory inputs
to effect motor outputs.
The symbol-like representations in SPA
allow us to interpret the states
of these dynamical systems
as linguistically relevant concepts,
which connects the low-level
sensorimotor aspects of speech
with the high-level linguistic aspects of speech.
Taken together, the NEF and SPA
provide the essential tools required
to apply biological methods
to the fields of
speech recognition and synthesis,
which are currently dominated
by computational statistics.

In addition to the application of biological methods,
we aim to provide a framework for
closed-loop speech systems
which perform both recognition and synthesis
with common representations.
Currently, state-of-the-art approaches
to speech recognition and synthesis
are independent;
that is, speech recognition and synthesis
are seen as independent problems,
and are solved by independent systems.
In humans, however,
speech perception and production
overlaps and interacts in a closed-loop manner.
When learning to speak,
we use the perception of our own voice
to improve future vocalizations.
While the role of such a closed-loop system
diminishes over time,
we still regularly monitor
our own speech and can adapt
when our speech perception or production
are perturbed through illness
or other means.\footnote{
  Here, and for the remainder of this thesis,
  we will refer to the sensory aspect
  of speech as ``speech perception''
  when referring to human speech understanding,
  and as ``speech recognition''
  when referring to computer speech understanding.
  Similarly, the motor aspect of speech
  will be referred to as
  ``speech production''
  when referring to humans speaking,
  and as ``speech synthesis''
  when referring to computers producing speech.}

The long-term goal of the line of research
described in this thesis
is a closed-loop speech system
that uses biological methods
implemented in computers
to recognize and synthesize speech.
Such a system could produce its own training data,
and therefore would require
far less hand-labeled real-world data than
current purely statistical systems.
It would also enable interrogation
of how the system recognizes
and synthesizes speech,
and enable synergistic interactions
between theoretical model developers
and experimental researchers
in linguistics and neuroscience.
As the model develops and makes predictions
about the computations required for speech
and how they might be implemented,
linguists and neuroscientists
can test those predictions
and update the model accordingly.

In order to achieve the long-term goal
of a closed-loop system
implemented with biologically realistic neurons
that can learn from itself,
we must first
envision what a large-scale closed-loop system
would look like,
and detail the computations required by that system.
We must also verify that neural implementations
of these computations
are possible with the available tools.
This thesis addresses these two concerns
by proposing a closed-loop speech system
called Sermo
(\underline{S}peech \underline{e}xecution and
\underline{r}ecognition \underline{m}odel \underline{o}rganism),
and constructing neural implementations
of three subsystems of Sermo.
In the conceptual Sermo model,
I have synthesized literature
in linguistics, psychoacoustics, neuroscience,
automatic speech recognition,
and articulatory synthesis
to present a closed-loop system
with well-defined computations.
In the three subsystems I have implemented,
I show that the NEF and SPA
can be successfully applied to
the computations required by the Sermo model,
providing the first steps
toward a neurally implemented closed-loop speech system.

The remainder of the thesis is organized as follows.
In Chapter~\ref{chapt:bg},
I review the relevant background informing
the conceptual Sermo model.
In Chapter~\ref{chapt:model},
I present Sermo,
which provides context for
the three neural models presented
in the subsequent chapters.
In Chapter~\ref{chapt:previouswork},
I review existing approaches
to solving the problems
proposed by Sermo
and addressed by the three neural models.
In Chapter~\ref{chapt:methods},
I provide details on the
mathematical techniques
used by the neural models,
and used to construct the neural models.
In Chapter~\ref{chapt:implementation},
I describe the three neural models.
In Chapter~\ref{chapt:results},
I propose metrics through which
to evaluate those models,
and present the results of collecting those metrics
while varying parameters of the neural models.
In Chapter~\ref{chapt:discussion}
I discuss the results,
summarize the contributions and predictions
of the models,
and discuss avenues for future work.
Finally, I conclude in Chapter~\ref{chapt:conclusion}.\footnote{
  Note that this chapter organization differs
  from a prototypical thesis
  because we make both
  conceptual contributions
  in the Sermo model
  and modeling contributions
  in the three subparts of Sermo
  for which we have made neural models.
  As such, the background chapter is designed
  to provide background for the conceptual model,
  which in turn provides context
  for the neural models.
  The previous work chapter provides
  background for the neural models,
  which are then explained fully
  in the methods and implementation chapters.}
