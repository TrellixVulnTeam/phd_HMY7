\chapter{Design and methodology}

%% ~15-30 pages

%% - continuing from Chapter 2 explain the issues
%% - outline your solution / extension / refutation

Natural speech is not a series of phonemes
recited as clearly as possible
to maximize intelligibility.
Natural speech includes variation in
pitch and volume in order to transfer
non-linguistic information between speakers.
Current automated speech systems
focus solely on the linguistic content
of speech;
by deciphering and reproducing both
linguistic and non-linguistic content,
we aim to produce more natural speech
recognition and production
than the existing state of the art.

\section{Recognition system}

The recognition system is composed of
three layers connected in a feed-forward manner.

???come up with a name maybe

???figure of whole system

\begin{itemize}
\item \textbf{Auditory periphery.} The auditory periphery layer
  takes incoming air pressure waves and converts them
  to a frequency representation,
  mimicking the function of the human ear.
\item \textbf{Auditory preprocessing.} The auditory preprocessing layer
  represents the frequency information from the auditory periphery,
  and through recurrent connections within the layer,
  also represents temporal dynamics of frequency information
  (e.g., time derivatives).
\item \textbf{Features.} The feature layer
  represents speech-relevant features;
  in this work, we represent phonemes, pitch, and volume.
  Features are computed from information
  in the auditory preprocessing layer.
\end{itemize}

This system is similar to many previous systems,
and aside from its feedforward nature,
follows naturally from the organization
of the human speech system (???ref or ???section).
The primary way in which this system
deviates from previous models
is by imposing biological constraints.
Specifically, the system operates
in continuous time,
and only uses locally available information.
Adhering to these constraints differentiates
this system from similar systems,
such as those based on deep learning (???refsection),
in which... ???more, it's discrete
This system is also ???something about NEF

\subsection{Auditory periphery}

???figure of periphery

\subsection{Auditory preprocessing}

\subsection{Features}

NEF stuff

Possible future in which this is learned

\subsection{The role of feedback}

The above system is primarily feed-forward
(though feedback is used extensively in the
auditory preprocessing layer).
It is important to note that
a complete auditory processing system
would match the brain (???refs) and have
feedback connections between all layers
in order to refine each layer's ability
to provide useful output downstream layers.
(??? more stuff from refs)

Feedback is an undeniably important
component of a full system that operates
for long periods of time with sensors
that are constantly changing and degrading.
In this work, however, we assume
that the auditory periphery does not change
its ability to process sounds over a long timescale,
and therefore we do not include corrective feedback
signals between layers in the auditory system.
This type of feedback could be added
to this system in future work.

\subsection{Evaluation}

\section{Synthesis system}

\subsection{Evaluation}

\section{Integrated speech system}

\subsection{Evaluation}
