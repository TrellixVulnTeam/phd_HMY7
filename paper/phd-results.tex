\chapter{Results and Evaluation}

%% ~15-30 pages

%% - adequacy, efficiency, productiveness, effectiveness
%%   (choose your criteria, state them clearly and justify them)
%% - be careful that you are using a fair measure, and that you are
%%   actually measuring what you claim to be measuring
%% - if comparing with previous techniques those techniques
%%   must be described in Chapter 2
%% - be honest in evaluation
%% - admit weaknesses

List of experiments (fill this in as you write other sections!!)

These should maybe be included in the methods?

\section{Recognition system}

\subsection{Metrics}

Phoneme decode error rate

Pitch error rate

Volume error rate

\subsection{Experiments}

1. Do frequency interacting transformation matter?

- generate networks with N random transformations
  in preprocessing layer and collect the metrics

- systematically increase the number of interacting
  transformations and see if it matters

2. Vowels: should we use temporal information or just
   the current moment's info?

3. Vowels: do diphthongs count?

- try having no categories for diphthongs and see if
  we can get the two components of it

- try having explicit diphthong categories and see
  if we get better at it

4. Separate vowel / consonant populations, or just one

- See if error rates differ

5. Consonants vs vowels:

- Vary the number of features and/or neurons and see
  how much is needed for equivalent error rates

  - Should error rates be normalized by the total number
    of possible outcomes?

6. Synthesized vs natural speech:

- How much easier is synthesized speech compared to natural?

7. Relative pitch experiment:

Find something from the literature to evaluate relative pitch...

8. Relative volume expt:

Find something from the literature to evaluate relative volume...

9. The usefulness of noise

- P. 65 of Kollier et al has a bunch of noise added in.
  Try injecting noise, see if it helps.

10. Preprocessing choices

- Pool or don't pool
- Use nonlinear derivative glides, or just pure derivatives

11. Phonemes or gestures

- Try decoding gestures instead of phonemes
- What's better / easier?

\section{Synthesis system}

\subsection{Metrics}

Speech intelligibility

RMSE between recognized and decoded speech?
Is that helpful?

Can use the recognition system to evaluate
this synthesis system relative to other
synthesis systems.
In a sense, this is one of the benefits
of this type of model.

Neural resources used

\subsection{Experiments}

1. Biologically plausible fluctuations.

- Ideal control methods have no variability, they hit things at the same time
- Record when consonantal closures / releases happen, show that there's
  a certain amount of variance
- Hopefully can show that this is similar to biology?

2. Oscillator / trajectory stability

- Show how accurate / fast the coupled oscillators can be
- Contrast to trajectory generation with Aaron's stuff
  (if that's possible)
  - Also contrast to trajectory gen with DMPs?

3. Scalability

- Take the control system and look at how much cortex (neurons, synapses)
  is taken up by each element (word, syllable, phoneme, etc).

- Extrapolate to human sized vocabularies, make sure it'll scale

- Show that if we had a separate oscillator / population
  for each word or syllable that this wouldn't scale

\section{Integrated system}

\subsection{Metrics}

\subsection{Experiments}

1. Shadowing proof of concept

- Show it works

??? more

4. Syllable learning proof of concept

- show it works

5. The effect of speed on syllable learning

- Slow it down. Should be better
