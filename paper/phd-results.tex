\chapter{Results and Evaluation}

%% ~15-30 pages

%% - adequacy, efficiency, productiveness, effectiveness
%%   (choose your criteria, state them clearly and justify them)
%% - be careful that you are using a fair measure, and that you are
%%   actually measuring what you claim to be measuring
%% - if comparing with previous techniques those techniques
%%   must be described in Chapter 2
%% - be honest in evaluation
%% - admit weaknesses

List of experiments (fill this in as you write other sections!!)

These should maybe be included in the methods?

\section{Recognition system}

\subsection{Metrics}

Phoneme decode error rate

Pitch error rate

Volume error rate

\subsection{Experiments}

1. Do frequency interacting transformation matter?

- generate networks with N random transformations
  in preprocessing layer and collect the metrics

- systematically increase the number of interacting
  transformations and see if it matters

2. Vowels: should we use temporal information or just
   the current moment's info?

3. Vowels: do diphthongs count?

- try having no categories for diphthongs and see if
  we can get the two components of it

- try having explicit diphthong categories and see
  if we get better at it

4. Consonants vs vowels:

- Vary the number of features and/or neurons and see
  how much is needed for equivalent error rates

  - Should error rates be normalized by the total number
    of possible outcomes?

5. Synthesized vs natural speech:

- How much easier is synthesized speech compared to natural?
