\chapter{Evaluation and results}
\label{chapt:results}

Here,
we present evaluation metrics for the three models
described in Chapter~\ref{chapt:implementation},
gather those metrics
for several experimental conditions,
and present the results.

\section{Neural cepstral coefficients}

We hypothesize that
neural cepstral coefficients (NCCs)
are a suitable feature vector representation
for traditional ASR systems,
and biologically grounded systems like Sermo.
In order to evaluate
its suitability for speech recognition tasks,
we compare NCCs to the most common
feature vector used currently,
Mel-frequency cepstral coefficients.

\subsection{Evaluation}

While different ASR systems
use the feature vector in different ways,
the most important quality of the feature vector
is that audio samples corresponding
to a particular label
(e.g., a phone or word)
take a similar trajectory
in the feature vector space
as other samples corresponding to the same label.
All samples corresponding to other labels
should take trajectories that are
far enough away in the vector space
to provide a basis for labeling
a test sample correctly.
As such, we compare MFCCs and NCCs
on a purely statistical level,
rather than using them as a frontend
in an ASR system,
which would bias the results
toward the aspects of the
feature vector trajectories
that the ASR system's backend
is most suitable for detecting.

We therefore compare MFCCs and NCCs
using a linear support vector machine (SVM)
that is trained on labeled
examples of feature vector trajectories
corresponding to phones
in the TIMIT training corpus,
and tested on trajectories
corresponding to phones
in the TIMIT testing corpus.

TIMIT \citep{garofolo1993} is a corpus of read speech
with time-aligned phone and word transcriptions.
The corpus contains ten speech samples
from 630 speakers across eight dialects
of American English,
resulting in 6300 utterances in total
(5.4 hours of continuous speech).
The utterances are separated into
a training set
(4620 sentences) and a testing set (1680 sentences),
allowing for apples-to-apples
comparisons of ASR systems.
TIMIT was released in 1990,
and is still a standard data set used
for testing ASR systems
due to the high quality
manual phone transcriptions,
diversity of speakers,
and convenient size,
as it is small enough to be
computationally tractable
but large enough to
provide a meaningful comparison
between ASR approaches.

We will focus on classifying
phones in the TIMIT data set.
Phones are voiced
for a shorter time than words,
and have high variability
as phone sounds can be affected
by the previous and next phones voiced,
making phone classification
a more difficult statistical task
than word classification.
Since we are not solving the full
speech recognition problem,
and are instead classifying
pre-segmented speech samples,
we anticipate that word classification
would be too simple to meaningfully
compare MFCCs and NCCs.

Additionally, classifying phones
is advantageous because similar phones
are voiced for a similar length of time,
unlike words.
Another way in which our statistical comparison
differs from the full speech recognition problem
is that we must stretch
feature vector trajectories
such that all trajectories are the same length,
due to the constraints of SVM classifiers;
therefore,
we aim to minimize the amount
of stretching done.

We stretch feature vector trajectories
to be as long as the longest trajectory
in the training and testing samples
using simple linear interpolation.
An alternative method
for stretching the trajectory
would be to use dynamic time warping
\citep{ratanamahatana2004}.
However, dynamic time warping
involves aligning each speech sample
to a representative example,
which would necessitate manually
determining a suitable example
for each phone for each experiment.
Additionally, the manual effort may not
provide any concrete benefits;
\citeauthor{ratanamahatana2004}
showed that linearly interpolating
time series data
produced classification accuracy rates
statistically indistinguishable
from accuracy rates on time series data
aligned using dynamic time warping.

TIMIT defines a set of 61 phones,
which includes vowels, consonants,
consonantal closures, pauses and silences.
\citet{lee1989} identified allophones and
other situation where two phones
are used interchangeably,
collapsing the full set of phones
to a set of 39 phones,
which are typically used
instead of the full set.
Many studies improve error rates
by further categorizing
the phones with similar properties;
e.g., grouping together all
plosives, fricatives, etc.
\citep{lopes2011}.
We will use the reduced set of 39 phones.\footnote{
  The set of 39 phones includes a ``silence'' phone,
  which we include when using all phones,
  but do not include in either
  the vowel or consonant phone set.}
Because vowels are typically voiced
longer than consonants,
we will investigate whether
separating them is necessary
since we lengthen all samples
to the longest sample.

Support vector machines are a standard
supervised learning technique
often applied to classification tasks,
and are particularly well suited
to high-dimensional data.
The SVM algorithm learns a hyperplane
that separates data
in one class from other classes,
which it can subsequently use
to classify data
by evaluating where the data point
lies in vector space compared to the hyperplane.
When dealing with more than two classes,
hyperplanes can be learned to separate
each combination of two classes
(``one-against-one'' classification)
or each class from all other classes
(``one-vs-the-rest'' classification).
When data is relatively low dimensional,
the kernel trick can be applied,
which projects the input data
into a higher dimensional space
based on a kernel function.

In the case of speech feature vector trajectories,
we have relatively many samples,
classes and features,
so we use a one-vs-the-rest
support vector machine classifier
that does not use the kernel trick
(i.e., a linear SVM).
We use LibLinear's fast implementation
of linear SVMs \citep{fan2008},
exposed to Python through
the scikit-learn package
\citep{pedregosa2011}.

The metric that we collect
in the experiments below
is classification correctness,
either in absolute terms,
or NCC classification correctness
relative to the baseline correctness
obtained with MFCCs.
In the ASR literature,
phone error rates are often
reported on TIMIT.
The phone error rate
takes into account
incorrect classifications
(substitutions),
a lack of classification
when there should be one
(deletions)
and erroneous classifications
where there should not be one
(insertions).
However, since we pre-segment
speech into short samples
corresponding to some phone,
we cannot have deletions or insertions,
only substitutions,
which corresponds to classification correctness.
Since we use correctness
rather than accuracy,
we can only compare our approach
to some reported results
\citep{lopes2011},
but even then the comparison is flawed.
Hence, we generate both MFCCs
and NCCs for the same speech samples
in order to verify that
NCCs generate feature vector trajectories
that are at least as statistically separable
as the equivalent MFCC trajectories.

\subsection{Experiments and results}
\label{sec:results-ncc}

In each experiment,
we isolate the TIMIT speech samples
corresponding to the phones
of interest,
and generate a feature vector trajectory
(i.e., MFCC or NCC)
for that sample.
MFCCs are generated using frames
of 25 ms,
which advance 10 ms on each timestep,
as is standard in many ASR systems.
NCCs are generated
by running the model described in
Section~\ref{sec:ncc-neural} with a 1 ms simulation timestep.
Because NCC generation takes several minutes,
and because the SVM fitting procedure
is quadratic in the number of samples,
we limit our experiments to
one region of the TIMIT dataset
(the ``army brat'' region,
which consists of a training set
of 220 utterances across 22 speakers,
and a test set of of 110 utterances
across 11 different speakers),
but run each experiment
10 times with different random seeds
to generate confidence intervals
for each experimental condition.
In the MFCC case,
the feature vectors are the same
on every trial;
the random seed controls the
stochastic nature of
the supervised learning procedure
used by the SVM.
In the NCC case,
the random seed affects both
the generation of model parameters
that are sampled from random distributions,
and the random seed used by the SVM.

All trajectories are
made to be the same length,
as is required by the SVM algorithm.
The number of frames used
for all samples is
the number of frames in
the longest MFCC trajectory
in the training set.
Samples that are too short
(e.g., most MFCC trajectories)
are lengthened to match the
target number of frames
using linear interpolations.
Samples that are too long
(e.g., all NCC trajectories)
are shortened to match the
target number of frames
by splitting the trajectory
into equally spaced time bins
and taking the average
for each bin.
See Figure~\ref{fig:temp-scaling} for
an illustration of temporal scaling,
and example MFCC and NCC trajectories.

\fig{temp-scaling}{1.0}{Example MFCC and NCC trajectories.}{
  Example MFCC and NCC trajectories and temporally scaled versions.
  (A) MFCCs generated from an utterance of the word
  ``deadline'' from the TIMIT speech corpus
  with 25~ms frames advanced 10~ms per frame.
  Both the raw MFCC and z-scored MFCC are shown.
  (B) MFCCs lengthened through linear interpolation
  to match the same number of frames as the NCC network.
  (C) NCCs generated from the same utterance
  with 1~ms timestep. Both the raw NCC and z-scored NCC are shown.
  (D) NCCs shortened through neighbourhood averaging
  to match the same number of frames as the MFCC model.}

Each trajectory is then flattened
to produce a vector of length
$n_{\text{cepstra}} \times n_{\text{frames}}$.
Each trajectory is associated with
the phone label associated with
the original speech sample,
producing an input-output pair
that is used to train the linear SVM.

Finally, the same procedure is done
with the test set to produce another
set of input-output pairs.
The input is fed to the linear SVM
to generate phone label predictions,
which are compared to the actual phones
to yield a correctness measure.

Wherever not specified,
the parameters used for each model
are as listed in Table~\ref{tab:ncc}.
In most cases, we test for correctness
only on the set of consonant phones
for speed reasons
(a more thorough justification is given
later).

\begin{table}[ht!]
  \begin{footnotesize}
    \begin{center}
      \begin{tabular}{lcl}
        \toprule
        Parameter & Default value & Part of model affected \\
        \midrule
        \texttt{MFCC dt} & 10 ms & MFCC (frame advance) \\
        \texttt{MFCC window\_dt} & 25 ms & MFCC (frame size) \\
        \texttt{n\_fft} & 512 & MFCC \\
        \texttt{deriv\_spread} & 2 & MFCC (derivative) \\
        \texttt{n\_cepstra} & 13 & MFCC and NCC \\
        \texttt{n\_filters} & 32 & MFCC and NCC \\
        \texttt{minfreq} & 0 & MFCC and NCC \\
        \texttt{maxfreq} & 8000 & MFCC and NCC \\
        \texttt{n\_derivatives} & 1 & MFCC and NCC \\
        \texttt{auditory\_filter} & Gammatone & Auditory periphery \\
        \texttt{neurons\_per\_freq} & 8 & Auditory periphery \\
        \texttt{adaptive\_neurons} & False & Auditory periphery \\
        \texttt{cepstra n\_neurons} & 20 & NCC cepstral coefficients \\
        \texttt{deriv\_type} & Feedforward & NCC derivative nets \\
        \texttt{deriv\_n\_neurons} & 20 & NCC derivative nets \\
        \texttt{deriv\_tau\_fast} & 0.005 & NCC FF derivative \\
        \texttt{deriv\_tau\_slow} & 0.1 & NCC FF derivative \\
        \bottomrule
      \end{tabular}
    \end{center}
  \end{footnotesize}
  \caption{Parameters used in the NCC model.}
\label{tab:ncc}
\end{table}

\subsubsection{Derivatives and z-scoring}

Two important choices for a feature vector
are whether to include derivatives,
and whether to normalize the resulting vector.
Typically, ASR systems include at least
one derivative, and perform normalization
through z-scoring
(i.e., removing the mean
and normalizing to unit variance).
We investigated whether these
choices were important for NCCs.

\fig{ncc-zscore}{1.0}{Varying z-score in NCC model.}{
  Results for the NCC experiment in which we vary
  whether the feature vector is z-scored or not.
  (A) Violin plot of the training and testing accuracy.
  Note that a violin plot is a kernel density estimation
  for each column, showing the entire distribution of results.
  (B) Bar plot of the training and testing accuracy.
  Error bars represent bootstrapped 95\% confidence intervals
  from the 10 trials.}

Figure~\ref{fig:ncc-zscore} shows
the training and testing correctness
for ten trials
with and without z-scoring.
For MFCCs,
z-scoring greatly reduces the variance
of the training and testing correctness;
the mean correctness values
may be slightly lower
for consonant phones,
but the high variance is concerning.
For NCCs,
z-scoring siginificantly
raises training correctness,
but lowers testing correctness.
Variance is approximately the same
regardless of whether the vectors are z-scored.

We interpret these results
as meaning that z-scoring is essential
for MFCCs because without it,
the SVM fitting procedure
varies significantly for
different random seeds
(hence the high variability without z-scoring).
NCCs, on the other hand,
have low variance in either condition.
Since a higher test correctness indicates
better generalization,
we believe that z-scoring
does more harm than good for NCCs.
Therefore, for future simulations,
we will z-score the MFCC feature vector,
but not the NCC feature vector.

\fig{ncc-derivatives-acc-b}{0.65}{
  Varying number of derivatives in NCC model.}{
  Results for the NCC experiment in which we varied
  the number of derivatives used in both the MFCC and NCC vectors.
  Error bars represent bootstrapped 95\% confidence intervals
  from the 10 trials.}

Figure~\ref{fig:ncc-derivatives-acc-b} shows
results varying the number of derivatives
used while always z-scoring MFCCs,
and never z-scoring NCCs.
Note that using one derivative means
that the feature vector has
length 26,
where the first 13 elements
are cepstral coefficients,
and the last 13 are derivatives
of the cepstral coefficients;
using two derivatives means that
the feature vector has length 39,
with the derivative of the
derivative of the cepstral coefficient
appended to vector with 26 elements.

For MFCCs, adding a single derivative
improves testing correctness
by almost double
(25\% to 36\% correct),
but adding a second derivative
has very little effect
(36\% correct).
For NCCs, adding derivatives
improves training correctness,
but slightly reduces testing correctness.
Unlike with z-scoring,
we believe it is important
for the MFCC and NCC features to
be the same length,
so for subsequent experiments,
both feature vectors
will contain the first derivative
(and therefore have length 26).

\subsubsection{Number of neurons}

Ideally,
we would like to reduce the amount of time
that the NCC generation process takes.
Reducing the number of neurons
used in the NCC model
can speed it up,
but at the cost of performance.
In the next two experiments,
we vary the number of neurons used
in different layers of the model
in order to see their impact
on speed and performance.

\fig{ncc-periphery-acc-t}{0.65}{
  Varying number of periphery neurons in NCC model.}{
  Results for the NCC experiment in which we varied
  the number of neurons in the auditory periphery.
  Shaded regions represent bootstrapped 95\% confidence intervals
  from the 10 trials.}

Figure~\ref{fig:ncc-periphery-acc-t} shows the results of
varying the number of neurons
in the auditory periphery layer.
In this layer,
each auditory filter projects
to an ensemble of neurons
of the size varied in this experiment.
Since the representation here
is a simple one
(a positive scalar value),
the number of neurons required
should not be very large,
so we vary between
1 and 32 neurons with steps
at each power of two.
The number of neurons used
in the feature layers
is fixed at 20 neurons per feature.
Instead of plotting the absolute correctness,
in this and subsequent experiments,
we plot the correctness of the NCC model
relative to the mean correctness
of the same number
of MFCC model instance accuracies.

As seen in Figure~\ref{fig:ncc-periphery-acc-t},
training correctness
does not change significantly
whether we use
1 or 32 neurons neurons per filter,
though variance is higher
with fewer neurons.
Testing correctness,
however, improves up to
using 8 neurons per filter,
but then plateaus,
or may even decline
(possibly due to overfitting).
Therefore, in all
subsequent experiments,
we use 8 neurons to represent
the output of each auditory filter.

\fig{ncc-feature-acc-t}{0.65}{
  Varying number of feature neurons in NCC model.}{
  Results for the NCC experiment in which we varied
  the number of neurons in the ensembles representing elements
  of the feature vector (i.e., cepstral coefficients
  and the first derivative of the cepstral coefficients).
  Shaded regions represent bootstrapped 95\% confidence intervals
  from the 10 trials.}

Figure~\ref{fig:ncc-feature-acc-t} shows the results
of varying the number of neurons
in the feature layer,
which includes ensembles
representing the cepstral coefficients,
the ensembles representing
the derivatives of those coefficients,
and any additional ensembles required
to compute the derivative.
The number of neurons for each filter
in the periphery is fixed at 8,
as per the previous experiment.

A very different trend is evident
when varying the number of neurons
used in the feature layer.
For very small number of neurons,
training correctness is worse
than MFCC training correctness.
As the number of neurons increases,
correctness improves up to
using 8 neurons per ensemble,
then steadily worsens.
Testing correctness, however,
is highly variable until
we use 8 neurons per ensemble,
at which point performance
steadily increases up to
the highest number of neurons tested,
64 neurons per ensemble.

\fig{ncc-feature-time}{0.65}{Time taken for NCC feature neurons experiment.}{
  Time taken to run a trial for the NCC experiment
  varying the number of feature neurons.
  Error bars represent bootstrapped 95\% confidence intervals
  from the 10 trials.}

The number of neurons in the feature layer
has a big impact on the amount of time
each experiment takes;
see Figure~\ref{fig:ncc-feature-time}.
The time taken for each trial
is the same up to 12 neurons per ensemble,
at which point
the time taken sharply increases.
Since the correctness using 12 neurons per ensemble
is significantly higher than
the mean correctness of the equivalent MFCC feature vector,
we believe that using 12 neurons per ensemble
is sufficiently accurate.

\subsubsection{NCC averaging}

Before the eventual goal of
varying the auditory periphery model used,
we address concerns about the validity
of our comparison.

One concern is that
the NCC model runs at a lower timestep,
giving it some numerical advantages
over the MFCC model;
similarly, the shortening and lengthening
procedures may affect the results.
To examine the validity of these concerns,
we performed an experiment
with a fixed number of frames
per audio sample,
and varied the amount that the frame window
advanced on each timestep.
When the frame window moves at the
same rate as the Nengo simulation timestep,
the two models are manipulated
in the same way by the
shortening algorithm,
which averages over a time window.

\fig{ncc-dt-acc-b}{0.65}{Varying the MFCC frame advance step.}{
  Results for the NCC experiment in which we varied
  the MFCC frame advance step.
  Error bars represent bootstrapped 95\% confidence intervals
  from the 10 trials.}

Figure~\ref{fig:ncc-dt-acc-b}
shows the consonant correctness
as a function of the
frame window advance in seconds.
While using a lower timestep results in
significantly better training correctness,
testing correctness is actually better
with a longer timestep,
indicating that the increased temporal
resolution may cause overfitting.
It is possible that this overfitting
also occurs with the NCC model,
but running with a high simulation timestep
would result in an abundance of
synchronous spiking activity
and lowered decoding correctness.
We therefore conclude that the
differing timesteps and temporal scaling
do not introduce overt biases
in favor of the NCC model.

\subsubsection{Phone groups}

To this point we have
evaluated all of the experiments using
the consonant phones
under the assumptions that they are
more difficult to classify than vowel phones.
However, it may be the case that
using all phones together
is a more advantageous situation
for MFCCs compared to NCCs.
For that reason,
we compared the correctness
of MFCCs and NCCs when using
vowel phones, consonant phones,
and all phones together
(including the silent phone).

\fig{ncc-phones}{1.0}{Testing NCCs with different phone sets.}{
  Results for the NCC experiment in which we varied
  the set of phones used in both training and testing.
  (A) Consonant correctness values for both MFCC and NCC models
  are shown.
  (B) The correctness of the NCC model relative to the mean
  of the MFCC models is shown.
  In all cases, error bars represent bootstrapped 95\% confidence intervals
  from the 10 trials.}

Figure~\ref{fig:ncc-phones} presents the results.
Focusing on absolute correctness,
we observe that using MFCCs
yields slightly worse correctness
when using all phones compared
to vowel or consonant phone sets.
Using NCCs yielded the same
or slightly worse correctness
for consonants compared to all phones.
Using only consonants, therefore,
is more advantageous for MFCCs
than for NCCs.
Looking at the relative correctness values,
it can be seen that for all phone sets,
NCCs perform significantly better
than MFCCs.

\fig{ncc-phones-time}{0.65}{Time taken for NCC phone sets experiment.}{
  Time taken to run a trial for the NCC experiment
  varying the set of phones used.
  Error bars represent bootstrapped 95\% confidence intervals
  from the 10 trials.}

As seen in Figure~\ref{fig:ncc-phones-time},
using all phones increases the
amount of time to run experiments significantly.
Not only do NCCs take a longer time to generate,
the increased number of samples
causes the SVM fitting procedure
to take a long time
for both MFCCs and NCCs.
Interestingly, fitting for NCCs
takes slightly above half
the time as fitting for MFCCs
(approximately 520 seconds
compared to 1110 seconds for MFCCs),
despite the number of samples
and number of features being identical.
This result indicates that it is
easier to find linearly separating hyperplanes
for NCCs than MFCCs.

In sum, we believe that our choice to
test with consonant phones
is justified,
and provides the most difficult
situation for NCCs compared to MFCCs.

\subsubsection{Auditory periphery model}
\label{sec:results-periphmodel}

The final experiment varies the auditory periphery model
used at the beginning of the NCC pipeline.
We looked at two choices in the auditory periphery
and how those choices impacted classification correctness.

The first choice is the auditory filter model.
We ran 20~trials with each of the five
auditory filter models presented
in Section~\ref{sec:periphery-models}
(10 with adaptive neurons, 10 with normal LIF neurons).
While most of the parameters are the same
as those in Table~\ref{tab:ncc},
the current implementation of the Tan Carney model
can only be used for audio signals
with a 50~kHz sampling rate;
for this reason, all TIMIT samples
in these experiments
are upsampled to 50~kHz,
and all models (both the MFCC and NCC models)
use the upsampled audio.
Additionally, the Tan Carney model
has numerical instabilities for auditory filters
with low characteristic frequencies,
so all models use frequencies
evenly distributed over the Mel scale
from 200~Hz to 8000~Hz.

\fig{ncc-periphmodel-racc-b}{0.65}{
  Comparing NCCs produced with five auditory periphery models.}{
  Results for the NCC experiment in which we varied
  the auditory periphery model, relative to the mean MFCC correctness.
  Compressive GC refers to the dynamic compressive Gammachirp filter.
  In all cases, error bars represent bootstrapped 95\% confidence intervals
  from the 20 trials.}

The results are plotted
in Figure~\ref{fig:ncc-periphmodel-racc-b}.
The Gammatone and Tan Carney models
both perform well,
with around 40\% better relative testing correctness
than the MFCC model.
The Compressive Gammachirp model
also performed well,
though statistically worse
than the Gammatone and Tan Carney models.
The Log Gammachirp and Dual Resonance models
both perform significantly worse
than the other three,
achieving only around 20--24\%
more correctness than the MFCC model.

\fig{ncc-periphmodel-time}{0.65}{
  Time taken for NCC periphery model sets experiment.}{
  Time taken to run a trial for the NCC experiment
  varying the auditory periphery model.
  Compressive GC refers to the dynamic compressive Gammachirp filter.
  Error bars represent bootstrapped 95\% confidence intervals
  from the 20 trials.}

As can be seen in Figure~\ref{fig:ncc-periphmodel-time},
the compressive Gammachirp
and Tan Carney models are
more computationally expensive
(3--4 times slower)
than the other three models.
As the least expensive model,
the Gammatone filter
provides by far the best tradeoff
between classification correctness
and computational cost.

\fig{ncc-adaptive-racc-b}{0.65}{Varying neuron type in NCC model.}{
  Results for the NCC experiment in which we vary
  the neuron type.
  Error bars represent bootstrapped 95\% confidence intervals
  from the 50 trials (10 per periphery model).}

We also varied whether we used
normal LIF neuron or adaptive LIF neurons,
as spiral ganglion cells
have facilitating and depressing behavior
that is partly captured in the adaptive LIF neuron.
As shown in Figure~\ref{fig:ncc-adaptive-racc-b},
using adaptive LIF neurons had no impact
on classification correctness.
This result holds true even when
evaluated for each auditory periphery model
individually (not pictured).

\subsection{Scaling}
\label{sec:res-ncc-scaling}

The model used in these experiments
represents a small portion
of the human auditory system;
it is matched primarily to the parameters
commonly used in ASR systems.
However, since our model is
implemented with biologically realistic parts,
it is important to determine
if a scaled up version of the model
matches known neuroanatomical constraints.

The number of spiking neurons used
in the model is
\begin{equation}
  N = N_f |f| + N_c |c| + \sum_{i=0}^{|d|} 2 N_d |c|,
\end{equation}
where $N_f$ is the number of neurons per frequency,
$|f|$ is the number of frequencies modeled,
$N_c$ is the number of neurons per cepstral coefficient,
$|c|$ is the number of cepstral coefficients,
$|d|$ is the number of derivatives,
and $N_d$ is the number of neurons per derivative.

With the parameters used
in the majority of the experiments above,
$N=1036$ neurons.
According to the BioNumbers database,
the human cochlea has 3500 inner hair cells
\citep[BNID~100697]{milo2010}.
Assuming that each inner hair cell
represents a separate auditory filter
(which is not necessarily true,
but a worst case assumption),
we conservatively estimate
that a scaled up version
of the NCC model
would use 20 cepstral coefficients,
1 derivative, and 50 neurons per feature,
resulting in $N=7.3 \times 10^4$ neurons.
If we are somewhat more generous
and use 40 cepstral coefficients,
2 derivatives, and 200 neurons per feature,
we get $N=1.8 \times 10^5$ neurons.
\citet{smiley2013} estimated
that human A1 alone contains
2$\times 10^7$--3$\times 10^7$ neurons,
meaning that the NCC model
fits well within human neuroanatomical constraints.

\section{Syllable sequencing and production}
\label{sec:results-production}

The goal of the syllable sequencing and production model
is to generate a continuous trajectory
of production information
that can be used to control
an articulatory synthesizer,
given a static representation
of a sequence of syllables.
As the quality of generated speech
is difficult to quantify,
we evaluate this model
by generating a gesture score
for the decoded production information trajectory,
and comparing it
to the gesture score that was provided
to the model as input.

\subsection{Evaluation}

While the syllable sequencing and production model
can be used to control any articulatory synthesizer,
we use the VocalTractLab synthesizer
because of its high quality speech
and available API \citep{birkholz2013}.
The vocal tract gesture representation
in VocalTractLab is based on a set of
predefined articular positions.
Most gestures correspond to
a particular shape,
and the overall vocal tract shape
is determined by linearly interpolating
the shapes of all currently active gestures.
Taking into account all of the
possible shapes and glottal parameters,
there are 48 possible vocal tract gestures
(see Table~\ref{tab:vtl-vtg}) despite having only
7 articulator sets
controlling 22 articulators.\footnote{
  There is also an f0 gesture and articulator set
  which is not included here,
  as it controls the pitch of the utterance.
  Since German is not a tonal language,
  the pitch is not required to voice a syllable,
  and instead is used for generating
  varied prosody;
  we do not consider prosody in this model,
  though including it is important future work.}

\begin{table}[ht!]
  \renewcommand{\arraystretch}{0.9}
  \begin{scriptsize}
    \begin{center}
      \begin{tabular}{lll}
        \toprule
        Gesture & Value & Gesture type \\
        \midrule
        \texttt{a} & Binary & Vowel \\
        \texttt{e} & Binary & Vowel \\
        \texttt{i} & Binary & Vowel \\
        \texttt{o} & Binary & Vowel \\
        \texttt{u} & Binary & Vowel \\
        \texttt{E:} & Binary & Vowel \\
        \texttt{2} & Binary & Vowel \\
        \texttt{y} & Binary & Vowel \\
        \texttt{A} & Binary & Vowel \\
        \texttt{I} & Binary & Vowel \\
        \texttt{E} & Binary & Vowel \\
        \texttt{O} & Binary & Vowel \\
        \texttt{U} & Binary & Vowel \\
        \texttt{9} & Binary & Vowel \\
        \texttt{Y} & Binary & Vowel \\
        \texttt{@} & Binary & Vowel \\
        \texttt{@6} & Binary & Vowel \\
        \texttt{aI-begin} & Binary & Vowel \\
        \texttt{aI-end} & Binary & Vowel \\
        \texttt{OY-begin} & Binary & Vowel \\
        \texttt{OY-end} & Binary & Vowel \\
        \texttt{aU-begin} & Binary & Vowel \\
        \texttt{aU-end} & Binary & Vowel \\
        \texttt{a-raw} & Binary & Vowel \\
        \texttt{i-raw} & Binary & Vowel \\
        \texttt{u-raw} & Binary & Vowel \\
        \texttt{ll-labial-nas} & Binary & Lip \\
        \texttt{ll-labial-stop} & Binary & Lip \\
        \texttt{ll-labial-fric} & Binary & Lip \\
        \texttt{tt-alveolar-nas} & Binary & Tongue tip \\
        \texttt{tt-alveolar-stop} & Binary & Tongue tip \\
        \texttt{tt-alveolar-lat} & Binary & Tongue tip \\
        \texttt{tt-alveolar-fric} & Binary & Tongue tip \\
        \texttt{tt-postalveolar-fric} & Binary & Tongue tip \\
        \texttt{tb-palatal-fric} & Binary & Tongue body \\
        \texttt{tb-velar-stop} & Binary & Tongue body \\
        \texttt{tb-velar-nas} & Binary & Tongue body \\
        \texttt{tb-uvular-fric} & Binary & Tongue body \\
        \texttt{breathy} & Binary & Glottal shape \\
        \texttt{pressed} & Binary & Glottal shape \\
        \texttt{open} & Binary & Glottal shape \\
        \texttt{stop} & Binary & Glottal shape \\
        \texttt{modal} & Binary & Glottal shape \\
        \texttt{slightly-pressed} & Binary & Glottal shape \\
        \texttt{slightly-breathy} & Binary & Glottal shape \\
        \texttt{fully-open} & Binary & Glottal shape \\
        \texttt{velic} & Scalar & Velic \\
        \texttt{lung-pressure} & Scalar & Lung pressure \\
        \bottomrule
      \end{tabular}
    \end{center}
  \end{scriptsize}
  \renewcommand{\arraystretch}{1.0}
  \caption[VocalTractLab gestures.]{
    Gestures defined by VocalTractLab.
    All gestures belong to a particular type,
    which determines the gesture sequence that they are a part of,
    and therefore the set of articulators that are modified
    by the gesture.}
\label{tab:vtl-vtg}
\end{table}

As discussed in Section~\ref{sec:impl-prod-overview},
the input to the model
is a semantic pointer
representing a sequence of syllables.
Each syllable has an associated DMP
which is generated using
a gesture score corresponding to
a German syllable;
the gesture scores are provided by
Bernd Kr\"{o}ger.\footnote{
  The gesture scores are freely available at
  \url{http://www.phonetik.phoniatrie.rwth-aachen.de/bkroeger/research.htm}.
  These German syllable gesture scores are among
  the few publicly available gesture scores;
  I was not able to find any gesture scores
  for English syllables.}
Each gesture score in VocalTractLab
is a collection of
gestures that are parameterized by target value,
onset time, gesture length,
and a time constant that determines
how quickly the gesture becomes active.

For each experiment described below,
we chose a random set of syllables
as the frequent syllables
that make up the syllables
in the mental syllabary of the model.
A sequence of syllables of a given length
are randomly chosen and assigned a speed
sampled from $\mathcal{U}(0.8, 3)$~Hz,
which corresponds to syllable trajectories of
333--1250~ms in length.
Depending on the number of syllables
and the randomly sampled lengths,
in each trial we end up with a syllable sequence
generating several seconds
of production information trajectories,
which in the case of these experiments,
are 48-dimensional trajectories
where each dimension is a gesture.

The model is provided the static syllable sequence input,
and is run for the amount of time that the sequence
should take to voice (determined by the
randomly assigned syllable speeds),
plus a small amount of time to account
for variability in each model instance.
The output of the temporal output associative memory
and the production information ensembles
are recorded.

That production information output is then converted
into a gesture score
by estimating the gesture parameters
from the production information trajectories.
Individual gestures are determined by
taking the absolute value of the temporal derivative
across each gesture;
time slices which have high derivatives
are indicative of the onset or offset of a gesture.
We make a new gesture for each
time slice with high derivative
with the beginning of the gesture as
the beginning of the first time slice,
and the end of the gesture
as the end of the next time slice;
the last time slice ends at the end
of the utterance.
The time constant associated
with the gesture
is the length of
the onset slice with high derivative.
The value of the gesture is the
average value between the two
time slices with high derivatives.
Every gesture has a neutral value
representing that the gesture
is not active;
identified gestures with values
sufficiently close to the neutral gestures
are marked as neutral gestures.

The gesture score resulting from the model simulation
is then compared to the original gesture score
on three criteria.
First, we calculate the accuracy
of the reconstructed gesture score
with the same equation as phoneme classification accuracy;
specifically,
\begin{equation} \label{eq:accuracy}
  \text{Acc} = \frac{N_g - S - D - I}{N_g},
\end{equation}
where $N_g$ is the number of gestures
in the target gesture score,
$S$ is the number of substitutions,
$D$ is the number of deletions,
and $I$ is the number of insertions.

Therefore, we must find the number of
substitutions, deletions, and insertions
in our gesture score compared to
a gesture score composed of
the original gestures scores
for the target syllables
concatenated together.
In order to determine the number of
insertions, deletions and substitutions,
we first convert the original
and reconstructed gesture scores
to a sequence of characters
by assigning a character to each gesture
and determining their order
within each articulator set,
then concatenating them.
Neutral gestures are recorded
and denoted with the ``\texttt{0}'' character;
numerical gestures are binarized
and given either ``\texttt{0}'' or ``\texttt{9}''
depending on the numerical value
relative to the gesture's range.
We then do a global pairwise alignment\footnote{
  Using the Needleman-Wunsch algorithm \citep{needleman1970},
  implemented by the nwalign Python package
  available at
  \url{https://pypi.python.org/pypi/nwalign/}.}
of the strings determined
from the target gesture
and the reconstructed gesture,
which is then evaluated
according to Equation~\eqref{eq:accuracy}.

In order to evaluate
the model's accuracy scores
(since we know of no comparable
speech trajectory generation techniques)
we compare the accuracy
of the reconstructed gesture scores
to a baseline calculated as
the mean accuracy of all the
non-matching syllables in the
repository of syllables.
In other words, we calculate
the accuracy of all combinations
of syllables in the syllable repository
(except for matching syllables)
and set the baseline as the mean
of those accuracy measures.
The result of this calculation is 0.5164.
This baseline is relatively high
(chance accuracy is likely much lower than 0.5164);
however, consider that many syllables
have the same form (e.g., CV or CVC)
and are only differentiated by
one or two gestures.
Additionally, all syllables
have a similar profile for
the lung pressure gesture.
If the model can perform better
than this baseline, however,
we can be confident
in saying that it is
following a similar trajectory
as the target gesture score.

Since the accuracy measure only tells us
that the gesture score contains
the right gestures in the right order,
we also aim to quantify
how the temporal characteristics
of the reconstructed gesture score
differ from the target gesture score.
For each correctly aligned gesture
in the alignment computed
for the accuracy measure,
we record the difference
between the gesture durations,
and report both the mean and variance.
The mean tells us about
overall temporal difference,
indicating that gesture durations
differ in predictable ways;
e.g., a high mean difference
indicates that
the reconstructed gesture score
is slower than the target gesture score,
but may still be able to
produce intelligible speech.
High variance, however,
indicates that
gesture durations vary in unpredictable ways,
which is likely to be detrimental
to intelligible speech.

It is critically important
for speech that certain gestures
either begin or end at the same time;
we will call these co-occurring gestures,
though they may end or start at different times.
Therefore, for each pair
of co-occurring gestures in the
target gesture score,
we record the difference
in the reconstructed gesture score's
corresponding co-occurrence pair,
if those gestures exist.
We report the average absolute
difference between the start or end times
of reconstructed gestures
that co-occur in the target gesture,
and compare to a baseline
computed by shuffling all of the gestures
and deteremining if two random gestures
start or end at the same time.

Finally, we synthesize speech samples
for each trial using VocalTractLab,
and perform a qualitative evaluation
of its intelligibility.
Speech generated by the
target gesture score is compared
to that generated by the model.
We do not obscure the label
of the synthesized speech,
as this measure is reported primarily
to reinforce the other two measures;
it is not the primary measure
by which the model should be judged.

\subsection{Experiments and results}
\label{sec:prod-results}

\subsubsection{Basic model operation}

First, we examine the operation
of a model instance in detail
to ensure that the metrics we collect
adequately characterize
good and bad model trials.

Figure~\ref{fig:prod-good} shows
a successful model trial
sequencing and producing the syllables
\ipa{[blA ti dAs]};
the \ipa{[ti]} syllable is voiced
at around 2~Hz while the other
two syllables are voiced around 1~Hz.
The activities in various ensembles
reflect the transformations
and dynamics expected of the model,
and result in a final
production information trajectory
that looks similar
to the desired trajectory.
Accuracy for this trial is 0.93
(2 insertions occur at the end of the trajectory);
the co-occurrence metric is 0.5
(chance is 0.083).

\fig{prod-good}{1.0}{
  Successful example trial of sequencing and production model.}{
  Successful example trial of the sequencing and production model.
  See Figure~\ref{fig:prod-network} for the network structure
  corresponding to the outputs plotted,
  and text for more details.}

\fig{prod-bad}{1.0}{
  Unsuccessful example trial of sequencing and production model.}{
  Unsuccessful example trial of the sequencing and production model.
  See Figure~\ref{fig:prod-network} for the network structure
  corresponding to the outputs plotted,
  and text for more details.}

Figure~\ref{fig:prod-bad}
shows an unsuccessful model trial
sequencing and producing
\ipa{[blA ti dAs]}.
In this instance,
the sequencing mechanism did not properly
switch to the next syllable,
resulting in a repetition of the
\ipa{[ti]} syllable.
Accuracy for this model is 0.76
(1 substition, 2 deletions, and 4 insertions occur);
the co-occurrence metric is 0.55
(chance is 0.083).

\subsubsection{Determining model parameters}

For each trial in the subsequent experiments,
we randomly select a set of syllables
and a syllable sequence
from a repository of 417 German syllables.
Of these syllables,
48 have CCV structure,
184 have CV structure,
162 have CVC structure,
and 23 have V structure;
we do not bias the random sample
toward any of the structures,
meaning that we will generally
have more CV syllables
than any other type in these experiments.
Each randomly selected syllable
has a semantic pointer,
an ensemble in the associative memory,
and a DMP network associated with it.
The intrinsic frequency (speed)
of each syllable is randomly assigned
between 0.8 and 3.0 Hz,
except where noted otherwise.
For each trial,
both the syllable repository
and the length and order
of the syllable sequence
to be produced are randomly generated,
in order to obtain robust metrics
not biased toward certain syllables
or sequences of syllables.
Other model parameters are as listed in
Table~\ref{tab:prod},
unless otherwise noted.

For each experimental condition,
we ran 20 trials using
a random sequence of three syllables
from a repository of three random syllables,
unless otherwise noted.
We allow syllable repetitions to occur,
but due to randomization,
syllable repetitions may not occur.

\begin{table}[ht!]
  \begin{footnotesize}
    \begin{center}
      \begin{tabular}{lcl}
        \toprule
        Parameter & Default value & Part of model affected \\
        \midrule
        \texttt{sequence n\_per\_d} & 30 & SPA modules in Sequence net \\
        \texttt{syllable\_d} & 48 & SPA modules in Sequence net \\
        \texttt{n\_positions} & 7 & SPA vocabulary \\
        \texttt{difference\_gain} & 15 & Working memories in Sequence net \\
        \texttt{threshold\_memories} & True
          & Associative memories in Sequence net \\
        \texttt{sequencer n\_per\_d} & 400 & Ensembles in Sequencer net \\
        \texttt{sequencer tau} & 0.05 & Timer ensemble in Sequencer net \\
        \texttt{sequencer freq} & 1~Hz & Timer ensemble in Sequencer net \\
        \texttt{reset\_time} & 0.85
          & Timer to reset function in Sequencer net \\
        \texttt{reset\_threshold} & 0.5
          & Timer to reset function in Sequencer net \\
        \texttt{reset\_to\_gate} & 0.65
          & Reset to gate function in Sequencer net \\
        \texttt{gate\_threshold} & 0.4 & Gate function in Sequencer net \\
        \texttt{syllable n\_per\_d} & 600 & Syllable DMP ensembles \\
        \texttt{syllable tau} & 0.02 & Syllable DMP recurrent connection \\
        \texttt{syllable freq} & 1~Hz & Syllable DMP recurrent connection \\
        \texttt{prod\_info threshold} & 0.3 & Production information net \\
        \texttt{t\_release} & 0.14 s & Initial input length \\
        \texttt{sequence} & No default & Syllable sequence \\
        \texttt{repeat} & True & Syllable sequence generation \\
        \bottomrule
      \end{tabular}
    \end{center}
  \end{footnotesize}
  \caption{Parameters used in the syllable sequencing and production model.}
\label{tab:prod}
\end{table}

Since the dynamics of the DMP networks
have a significant impact
on the successful operation of the model,
we begin by varying
the time constant on
the DMP state's recurrent connection.

\fig{prod-tau}{0.9}{Varying DMP $\tau$ in production model.}{
  Results for the sequencing and production experiment
  in which we vary the $\tau$ parameter of each recurrent DMP
  state ensemble.
  Plots show the accuracy, number of substitutions/insertions/deletions,
  timing differences, and cooccurrence metric.
  The dotted line in the accuracy plot denotes
  the baseline accuracy value,
  which is the mean of the accuracy between all non-matched
  syllables in the repository of 417 syllables;
  the dotted line in the coocurrence metric
  denotes the metric obtained through random chance.
  Shaded regions represent bootstrapped 95\% confidence intervals
  from the 20 trials.}

As can be seen in Figure~\ref{fig:prod-tau},
accuracy is significantly above baseline
until $\tau > 0.035$~s;
poor accuracy is due to a proliferation
of deletions,
which are likely to due to some syllable DMPs
not activating properly.
Timing variance is generally low
until $\tau \ge 0.04$.
Timing means are negative,
indicating that the model
voices syllables faster
than the original gesture socre.
Cooccurence is significantly
better than chance for all values of $\tau$.

Next, as with the NCC model,
we vary the number of neurons
in different parts of the model
to speed up subsequent experiments.
Specifically, we vary the number of neurons
used in each DMP network and
the number of neurons used in the
timing ensemble in the sequencing network.

\fig{prod-syllneurons}{0.9}{
  Varying number of DMP neurons in production model.}{
  Results for the sequencing and production experiment
  in which we vary the number of neurons in each recurrent DMP
  state ensemble.
  Plots show the accuracy, number of substitutions/insertions/deletions,
  timing differences, and cooccurrence metric.
  The dotted line in the accuracy plot denotes
  the baseline accuracy value,
  which is the mean of the accuracy between all non-matched
  syllables in the repository of 417 syllables;
  the dotted line in the coocurrence metric
  denotes the metric obtained through random chance.
  Shaded regions represent bootstrapped 95\% confidence intervals
  from the 20 trials.}

As shown in Figure~\ref{fig:prod-syllneurons},
the number of neurons in each DMP network
does not have an obvious effect
on the model as long as at least
150~neurons per dimension are used.
Since these network implement
the oscillation, we hypothesized
that using more neurons would
increase accuracy;
however, this hypothesis was not supported.
If anything, accuracy decreased with
high numbers of neurons.
For subsequent simulations,
600~neurons per dimension are used.

\fig{prod-seqneurons}{0.9}{
  Varying number of sequencer timing neurons in production model.}{
  Results for the sequencing and production experiment
  in which we vary the number of neurons in the timing ensemble
  in the sequencer network (see Figure~\ref{fig:prod-network}).
  Plots show the accuracy, number of substitutions/insertions/deletions,
  timing differences, and cooccurrence metric.
  The dotted line in the accuracy plot denotes
  the baseline accuracy value,
  which is the mean of the accuracy between all non-matched
  syllables in the repository of 417 syllables;
  the dotted line in the coocurrence metric
  denotes the metric obtained through random chance.
  Shaded regions represent bootstrapped 95\% confidence intervals
  from the 20 trials.}

Figure~\ref{fig:prod-seqneurons}
shows the results of varying
the number of neurons in the
timer ensemble in the sequencer network.
Unlike the number of neurons
in DMP networks,
increasing the number of neurons
in the timer ensemble has
more clear results.
Using very few neurons
(less than 30 per dimension)
results in performance
significantly worse than baseline.
Increasing the number of neurons
generally increases accuracy,
up to the maximum number tested,
1000~neurons per dimension,
which is the value used
in subsequent experiments.

\subsubsection{Scaling to natural speech}

With a reasonable set of parameters chosen,
we focus on whether the model
can scale to natural speech situations,
which involve
a large repository of syllables
voiced rapidly.

\fig{prod-freq}{0.9}{
  Varying the speed of syllables in the production model.}{
  Results for the sequencing and production experiment
  in which we vary the frequency (i.e., speed) of
  syllables in the syllable repository.
  Plots show the accuracy, number of substitutions/insertions/deletions,
  timing differences, and cooccurrence metric.
  The dotted line in the accuracy plot denotes
  the baseline accuracy value,
  which is the mean of the accuracy between all non-matched
  syllables in the repository of 417 syllables;
  the dotted line in the coocurrence metric
  denotes the metric obtained through random chance.
  Shaded regions represent bootstrapped 95\% confidence intervals
  from the 20 trials.}

First, we investigated the
range of frequencies at which the model
can produce accurate trajectories.
In this experiment, we use a random sequence
of three syllables from a
repository of three random syllables
with the same frequency.
Figure~\ref{fig:prod-freq} shows the accuracy
of the model as a function of the frequency.
Interestingly, the model is least accurate
for both low and high frequencies.
At low frequencies, accuracy is around
the baseline, with relatively high
numbers of deletions,
suggesting that the DMP networks
may not be sufficiently activated
when the sequencer network
is operating at slow speeds.
Accuracy is better than baseline
from around 2.5--6.5~Hz
(i.e., syllables of 150--400~ms in length),
which matches the range of frequencies
often used in normal speech.
The network starts to break down
at higher frequencies,
again with a relatively high number
of deletions, indicating that
the sequencer network
cannot appropirate coordinate
other networks when operating
above 6.5~Hz.

\fig{prod-n_syllables}{0.9}{
  Varying the syllabary size in the production model.}{
  Results for the sequencing and production experiment
  in which we vary the number of syllables
  in the syllable repository.
  Plots show the accuracy, number of substitutions/insertions/deletions,
  timing differences, and cooccurrence metric.
  The dotted line in the accuracy plot denotes
  the baseline accuracy value,
  which is the mean of the accuracy between all non-matched
  syllables in the repository of 417 syllables;
  the dotted line in the coocurrence metric
  denotes the metric obtained through random chance.
  Shaded regions represent bootstrapped 95\% confidence intervals
  from the 20 trials.}

Next, we investigated whether the model
behavior changes when the size
of the syllabary changes.
Since adults have a repository
of hundreds or thousands
of commonly uttered syllables,
the size of the syllabary should have
little effect on the model.
However, as is shown in Figure~\ref{fig:prod-n_syllables},
model accuracy decreases
as the size of the syllabary increases,
only being more accurate than baseline
for up to four syllables,
due to a sharp increase
in the number of deletions.
We have not yet determined the source
of these deletions.

\fig{prod-sequence_len}{0.9}{
  Varying the sequence length in the production model.}{
  Results for the sequencing and production experiment
  in which we vary the number of syllables
  voiced in an experimental trial.
  Plots show the accuracy, number of substitutions/insertions/deletions,
  timing differences, and cooccurrence metric.
  The dotted line in the accuracy plot denotes
  the baseline accuracy value,
  which is the mean of the accuracy between all non-matched
  syllables in the repository of 417 syllables;
  the dotted line in the coocurrence metric
  denotes the metric obtained through random chance.
  Shaded regions represent bootstrapped 95\% confidence intervals
  from the 20 trials.}

In a similar vein,
we investigated whether model behavior
depends on the length of the syllable sequence,
with the size of the syllabary fixed.
In this case, we know from \citet{choo2010}
that sequences of length five or greater
tend to have difficulty with elements
in the middle of the sequence.
As shown in Figure~\ref{fig:prod-sequence_len},
accuracy drops off after only four syllables.
Interestingly, accuracy is also low
for sequences of length one;
we believe that the increased deletions
for sequences of length one occur
because the network is allowed some
extra grace time after the syllable should
be finished, in case the model
is slower than the correct sequences.
However, for sequences of length one,
this grace period is likely too long,
resulting in the syllable
being repeated,
inserting extra gestures.

\fig{prod-repeat}{0.9}{
  Comparing repetitive and unique sequences in the production model.}{
  Results for the sequencing and production experiment
  in which we allow syllables to repeat,
  or constrain the sequence to contain only unique syllables.
  Plots show the accuracy, number of substitutions/insertions/deletions,
  timing differences, and cooccurrence metric.
  The dotted line in the accuracy plot denotes
  the baseline accuracy value,
  which is the mean of the accuracy between all non-matched
  syllables in the repository of 417 syllables;
  the dotted line in the coocurrence metric
  denotes the metric obtained through random chance.
  Error bars represent bootstrapped 95\% confidence intervals
  from the 20 trials.}

Finally, to this point we have
allowed syllables to repeat
under the assumption that our network design
(specifically the choice to use rhythmic DMPs)
results in similar performance when
repeating syllables compared to
a sequence of unique syllables.
Therefore, we investigated whether
performance is impacted when
sequences of length three
are constrained to be unique.
As can be seen in Figure~\ref{fig:prod-repeat},
the model performs the same
whether syllables are allowed to repeat or not,
verifying our assumption.
The only statistically significant difference
between the model with repetitions
versus no repetitions
is that substitions are more likely
to occur when repetitions are allowed.

\subsubsection{Speech intelligibility}

While manually checking the synthesized speech
for each experimental trial
would take a significant amount of time,
it is nevertheless important
to ensure that the gesture scores
reconstructed from the model outputs
can yield intelligible speech.

To test this, we manually evaluated
one set of experimental trials
(specifically, those associated with
the experiment varying the number of
sequencer neurons,
with 1000~neurons used).
I and one other person evaluated
the set of 20 utterances independently,
and reevaluated utterances
in which our evaluations differed.
For the 20 trials,
we found that
almost all of them were cut off;
that is, not all of the syllables
in the sequence were present
in the synthesized speech.
This could be because the model
was not run for a sufficient length of time,
or because the last syllable DMP in the sequence
did not activate
despite the network running
for a sufficient length of time.
Regardless, since all trials
contained sequences of three syllables,
we compared the target synthesized syllables
to the syllables that were present
in the speech generated by the model.
We found that
\begin{itemize}
  \item 7 utterances (35\%) were as intelligible as the target,
  \item 8 utterances (40\%) were not as intelligible as the target,
    but still resembled it, and
  \item 5 utterances (25\%) were not intelligible.
\end{itemize}

The model, therefore,
is capable of producing intelligible speech
with syllable sequences of varying speed.

\subsection{Scaling}
\label{sec:res-prod-scaling}

The number of neurons in the
sequencing and production model is
\begin{equation}
  N = N_{\text{SPA}} + N_S + N_{\text{DMP}} + N_O,
\end{equation}
where these variables
are the number of neurons in
a subset of the model.
$N_{\text{SPA}}$ is the number of
neurons for all SPA modules,
which is
\begin{equation}
  N_{\text{SPA}} = 5 D(\text{SYLL}) N_{D(\text{SYLL})} +
      |DFT_{\text{SYLL}}| N_{D(\text{SYLL})}
      + 2 N_{\text{AM}} (3 |\text{SYLL}| + 1),
\end{equation}
where $D(\text{SYLL})$ is the dimensionality
of syllable semantic pointers,
$N_{D(\text{SYLL})}$ is the number of neurons
per syllable dimension,
$|DFT_{\text{SYLL}}|$ is the number of
discrete Fourier transform coefficients
used by the binding operator,
$N_{\text{AM}}$ is the number of neurons
per associative memory ensemble,
and $|\text{SYLL}|$ is the number of
syllables in the syllabary.

$N_S$ is the number of neurons
in the sequencer network,
which is
\begin{equation}
  N_S = 4 N_D_S + 140,
\end{equation}
where $N_D_S$ is the number of neurons
per dimension in the sequencer network,
and 140~neuron come from ensembles
of fixed size.

$N_{\text{DMP}}$ is the number of neurons
in DMP networks, which is
\begin{equation}
  N_{\text{DMP}} = 3 |\text{SYLL}| N_{|\text{DMP}|} + 40,
\end{equation}
where $N_{|\text{DMP}|}$ is the number of neurons
per DMP dimension,
and 40~neurons come from ensembles of fixed size.

$N_O$ is the number of neurons
in production information ensembles, which is
\begin{equation}
  N_O = 48 N_{|O|}
\end{equation}
where 48 is the number of gestures,
and $N_{|O|}$ is the number of neurons
per production information ensemble.

With the parameters used
in the majority of the experiments above,
$N=22180$ neurons,
with most of those neurons
in the SPA populations
($N_{\text{SPA}}=14200$)
and the DMP ensembles ($N_{\text{DMP}}=5520$).
Fortunately, these parts of the model
grow at a linear or sublinear rate
as additional syllables are added
to the syllabary.
If we assume a conservative syllabary
with around 1000 syllables,
and increase the dimensionality
of each syllable to 256,
the model uses
$N=2.2 \times 10^6$ neurons.
If we assume a more generous estimate
with 2000 syllables
and use 512 dimensions
per syllable semantic pointer,
then the model uses
$N=4.4 \times 10^6$ neurons.
Given that there are approximately
$2.7 \times 10^4$ neurons per mm$^3$ (on average)
in human cortex \citep[BNID 112050]{milo2010},
a human scale syllabary would take up
approximately 163.7 mm$^3$ of cortex
(0.1637 cm$^3$)
in the generous case,
which is a very small portion of cortex,
especially considering that this model
would be distributed over
several cortical areas.

\section{Syllable recognition}

The goal of the syllable recognition system
is to temporally classify syllables
given a continuous trajectory
of production information.
We will evaluate the system
based on its ability to emit
correct and timely syllable classifications,
and whether those classifications
are available in working memory
between classifications.

\subsection{Evaluation}

The infrastructure for evaluating
the recognition model
is similar to that of the production model.
We use the same library
of German syllable gesture scores,
from which we create
48-dimensional gesture trajectories.
On each trial,
we generate a trajectory
by concatenating the trajectories
of a randomly chosen sequence
of trajectories together
(i.e., we do not use the
syllable production model
to generate input for the model).
That trajectory is provided
to all of the iDMPs as input,
and we record the output of the
production information network
and the memory module.

From the data collected,
we compute three metrics
to evaluate performance
in different situations.
First, we calculate
classification accuracy
using the same formula
as is used to calculate
accuracy in the production model;
specifically,
we generate a list of all of the
syllables classified by the model,
and compare that list to the
originally specified list
using the formula
\begin{equation}
  \text{Acc} = \frac{N_S - S - D - I}{N_S},
\end{equation}
where $N_S$ is the number of syllables
in the trajectory,
$S$ is the number of substitutions,
$D$ is the number of deletions,
and $I$ is the number of insertions.
A substitution occurs when
the syllable is misclassified.
A deletion occurs when
a syllable trajectory has been followed,
but the model does not emit a classification.
An insertion occurs
when the model emits a classification
when a corresponding trajectory has not been followed
(e.g., it classifies the same syllable twice).

Second, we record the times at which
syllable classifications occur,
and compare those times
to the end of the actual syllable trajectories.
Here, we are attempting to determine
if the model can classify syllables
while they are being voiced,
or if the entire syllable must complete
before the classification can be made.

Third, we analyze the contents
of the memory module
in between syllable presentations.
Specifically,
we find the semantic pointer
with the maximum similarity
to the memory representation
at the midpoint between
the start and the end of a syllable
for all syllables in the trajectory
(except the first,
which cannot be remembered until it occurs).
The memory representation of the trial
is perfect if the
previously classified syllable
is in memory at the midpoint
of the next syllable.

\subsection{Experiments and results}

\subsubsection{Basic model operation}

First, we examine the operation
of a model instance in detail
to ensure that the metrics we collect
adequately characterize
good and bad model trials.

Figure~\ref{fig:recog-good} shows
a successful model trial
recognizing the syllables
\ipa{[blA ti dAs]};
all syllables are voiced
at around 2~Hz.
The activities in various ensembles
reflect the transformations
and dynamics expected of model,
and result
perfect classification and memory accuracy.

\fig{recog-good}{0.68}{
  Successful example trial of recognition model.}{
  Successful example trial of the syllable recognition model.
  For iDMPs, the dimension representing the system state
  is emphasized to illustrate state tracking.
  See Figure~\ref{fig:recog-network} for the network structure
  corresponding to the outputs plotted,
  and text for more details.}

\fig{recog-bad}{0.68}{
  Unsuccessful example trial of sequencing and production model.}{
  Unsuccessful example trial of the syllable recognition model.
  For iDMPs, the dimension representing the system state
  is emphasized to illustrate state tracking.
  See Figure~\ref{fig:recog-network} for the network structure
  corresponding to the outputs plotted,
  and text for more details.}

Figure~\ref{fig:recog-bad}
shows an unsuccessful model trial
recognizing \ipa{[blA ti dAs]}.
In this case, while the initial
\ipa{[blA ti]} syllables
are recognized successfully,
an erroneous additional
\ipa{[ti]} classification occurs.
Because of this erroneous classification,
the iDMPs are reset when
the \ipa{[dAs]} syllable begins,
causing it to not be classified.
Therefore, there are two correct
classifications and one substitution
(the \ipa{[dAs]} is replaced with the
erroneous \ipa{[ti]})
resulting in 66.7\% accuracy.

\subsubsection{Determining model parameters}

The basic experimental setup
in the following experiments
is similar to that of the production model.
However, instead of using
a sequence of syllables
as input and producing
a production information trajectory,
the recognition model
uses a production information trajectory
as input and produces
a sequence of syllable classifications.
Again, for each trial in each experiment,
we randomly select a set of syllables
and a syllable sequence
from which to generate a trajectory.
The random selection process
is the same as in the production model.
The randomly selected
gesture scores are combined
into a single trajectory,
and that trajectory
is provided to the network
through a Nengo node.
The intrinsic frequency (speed)
of each syllable is randomly assigned
between 1.3 and 1.5~Hz,
except where noted otherwise.
For each experimental condition,
we ran 20 trials using
a trajectory made from
three random syllables
in a repository of three syllables,
meaning that syllable repetition can occur,
but is not guaranteed.
Wherever not specified,
we use parameters are defined in
Table~\ref{tab:recog}.

\begin{table}[ht!]
  \begin{footnotesize}
    \begin{center}
      \begin{tabular}{lcl}
        \toprule
        Parameter & Default value & Part of model affected \\
        \midrule
        \texttt{dimensions} & 64 & Associative memory \\
        \texttt{threshold} & 0.9 & Associative memory \\
        \texttt{wta\_inhibit\_scale} & 3 & Associative memory \\
        \texttt{memory neurons\_per\_dimension} & 60 & Memory \\
        \texttt{memory feedback} & 0.83 & Memory recurrent connection \\
        \texttt{reset\_th} & 0.9 & Classifier ensemble \\
        \texttt{inhib\_scale} & 1 & Classfier connections to iDMPs \\
        \texttt{syllable n\_per\_d} & 1000 & Syllable iDMP ensembles \\
        \texttt{syllable tau} & 0.05 & Syllable iDMP recurrent connection \\
        \texttt{similarity\_th} & 0.82 & Syllable iDMP recurrent connection \\
        \texttt{scale} & 0.71 & Syllable iDMP recurrent connection \\
        \texttt{reset\_scale} & 2.5 & Syllable iDMP diff connection \\
        \texttt{trajectory} & No default & Input trajectory \\
        \texttt{repeat} & True & Input trajectory generation \\
        \bottomrule
      \end{tabular}
    \end{center}
  \end{footnotesize}
  \caption{Parameters used in the syllable recognition model.}
\label{tab:recog}
\end{table}

\fig{recog-similarity}{0.9}{
  Varying the similarity threshold in the recognition model.}{
  Results for the syllable recognition experiment
  in which we vary the similarity threshold
  in the function computed through the recurrent connections
  on iDMP state ensembles.
  Plots show the accuracy, number of substitutions/insertions/deletions,
  timing differences, and memory accuracy.
  The dotted line in the memory accuracy plot denotes
  the chance accuracy value which depends on the syllabary size.
  Shaded regions represent bootstrapped 95\% confidence intervals
  from the 20 trials.}

In the case of the recognition model,
two parameters other than the number of neurons
have the most significant effect on model behavior.
First is the similarity threshold
in Equation~\eqref{idmp-dynamics1}.
As can be seen in Figure~\ref{fig:recog-similarity},
classification accuracy peaks with
similarity threshold around 0.8.
Up to 0.8, the number of insertions
is high, while after 0.8
the number of deletions increases.
Since insertions do not always harm
the memory representation,
memory accuracy is high
up to the peak of around 0.82,
at which point it sharply decreases.
Insertions are difficult to avoid;
however, deletions can be dealt
with through other parameters,
so we choose a similarity threshold of
0.82 for subsequent experiments.

\fig{recog-scale}{0.9}{
  Varying the scale in the recognition model.}{
  Results for the syllable recognition experiment
  in which we vary the scale term
  in the function computed through the recurrent connections
  on iDMP state ensembles.
  Plots show the accuracy, number of substitutions/insertions/deletions,
  timing differences, and memory accuracy.
  The dotted line in the memory accuracy plot denotes
  the chance accuracy value which depends on the syllabary size.
  Shaded regions represent bootstrapped 95\% confidence intervals
  from the 20 trials.}

The next parameter of interest is the
scale on the iDMP system state dynamics
($\alpha$ in Equation~\eqref{imdp-dynamics1}).
As can be seen in Figure~\ref{fig:recog-scale},
with similarity threshold of 0.82,
the scale makes a smaller difference
in classification and memory accuracy
than does the smilarity threshold.
Deletions are likely up to
a scale value to 0.8,
after which insertions become more likely.
Again, since deletions can be handled
through other means,
we err on the side of avoiding insertions,
and choose a scale value of 0.71
for subsequent experiments.

\fig{recog-syllneurons}{0.9}{
  Varying the number of iDMP state neurons in the recognition model.}{
  Results for the syllable recognition experiment
  in which we vary the number of neurons per dimension
  in the iDMP state ensembles.
  Plots show the accuracy, number of substitutions/insertions/deletions,
  timing differences, and memory accuracy.
  The dotted line in the memory accuracy plot denotes
  the chance accuracy value which depends on the syllabary size.
  Shaded regions represent bootstrapped 95\% confidence intervals
  from the 20 trials.}

Finally, we vary the number of neurons
used in each iDMP network.
As shown in Figure~\ref{fig:recog-syllneurons},
the number of neurons in each iDMP network
has a significant impact on the performance
of the overall model.
Classification accuracy is relatively low
until around 300 neurons per dimension,
and continues to increase
to the maximum size attempted,
1000 neurons per dimension.
As the number of neuron increases,
the probability of deletions decreases,
as an ensemble with more neurons
can better approximate Equation~\eqref{idmp-dynamics1}.
Memory accuracy with 1000 neurons per dimension
is especially high (nearly 90\%),
so we use 1000 neurons per dimensions
in subsequent experiments.
Note that using more than 1000 neurons per dimension
may give slight increases,
but with 1000 neurons per dimension
the number of deletions is already
as low as the number of insertions
and substitutions,
implying that further improvements
would be modest.

\subsubsection{Scaling to natural speech}

As in the production network,
we next vary experimental parameters
to examine how the model scales
to more realistic speech.

\fig{recog-freq}{0.9}{
  Varying the trajectory frequency in the recognition model.}{
  Results for the syllable recognition experiment
  in which we vary the frequency
  of the syllables that make up the input production information
  trajectory.
  Plots show the accuracy, number of substitutions/insertions/deletions,
  timing differences, and memory accuracy.
  The dotted line in the memory accuracy plot denotes
  the chance accuracy value which depends on the syllabary size.
  Shaded regions represent bootstrapped 95\% confidence intervals
  from the 20 trials.}

First, we investigated the
range of frequencies in which the model
can successfully classify trajectories.
In this experiment, we use a random sequence
of three syllables from a
repository of three random syllables
with the same frequency.
Originally, we hypothesized that the
model would be especially weak in this experiment,
as the values for the similarity threshold
and scale were chosen based on
experiments with 1.3--1.5~Hz.
As can be seen in Figure~\ref{fig:recog-freq},
our hypothesis was correct in the sense that
the recognition model
is more sensitive to frequency changes
than the production model.
Classification accuracy is initially poor;
however, this is due to a large number
of insertions
(likely insertions of the correct syllable).
Maximal accuracy around 75\%
occurs at 1.8~Hz,
just above the frequency that we
optimized for earlier.
After 1.8~Hz, accuracy decreases
as deletions become more common
than insertions.
the variance of the timing difference
(i.e., when the classification occurs
relative to the end of the syllable)
is very small from about 1.4 Hz and above.
The mean timing difference
is negative in for all frequencies,
indicating that classifications
occur before the syllable trajectory
is complete.

\fig{recog-n_syllables}{0.9}{
  Varying the syllabary size in the recognition model.}{
  Results for the syllable recognition experiment
  in which we vary the size of the syllabary
  from which we select three syllables to recognize.
  Plots show the accuracy, number of substitutions/insertions/deletions,
  timing differences, and memory accuracy.
  The dotted line in the memory accuracy plot denotes
  the chance accuracy value which depends on the syllabary size.
  Shaded regions represent bootstrapped 95\% confidence intervals
  from the 20 trials.}

We then investigated whether the model
behavior changes when the size
of the syllabary changes.
As shown in Figure~\ref{fig:recog-n_syllables},
accuracy drops off quickly when
more than five syllables
are part of the syllabary.
Unlike in the production system,
this effect is predictable
from the structure of the network,
due to the competitive nature
of the iDMPs;
all iDMPs are constantly tracking
the system state,
except when a classification has occurred
and the iDMP networks are being inhibited.
As the number of syllables in the repository increases,
the chance of substitutions increases,
as can be seen in Figure~\ref{fig:recog-n_syllables}.
Memory accuracy, however, remains significantly
above chance in all cases.

\fig{recog-sequence_len}{0.9}{
  Varying the sequence length in the recognition model.}{
  Results for the syllable recognition experiment
  in which we vary the number of syllables
  selected from the syllabary and concatenated together
  to make up the input trajectory.
  Plots show the accuracy, number of substitutions/insertions/deletions,
  timing differences, and memory accuracy.
  The dotted line in the memory accuracy plot denotes
  the chance accuracy value which depends on the syllabary size.
  Shaded regions represent bootstrapped 95\% confidence intervals
  from the 20 trials.}

The length of the sequence,
on the other hand,
has little effect on classification accuracy,
as can be seen in Figure~\ref{fig:recog-sequence_len}.
In this case, the trajectories
of the three syllables in the syllabary
are repeated several times.
The iDMP networks continue to operate
at the same accuracy rate
regardless of how long they have been active,
which is a benefit of this method
over HMMs, which must be renormalized
and reset often.
However, as seen in Figure~\ref{fig:recog-sequence_len},
it may be the case that memory accuracy
would decrease if the sequence length
is increases further,
due to the fact that the memory
is implemented with a simple decaying integrator.
If longer sequences prove to be problematic,
it may be necessary to replace the memory
with an input-gated working memory module,
as is used in the production system
for representing syllable positions.

\fig{recog-repeat}{0.9}{
  Comparing repetitive and unique sequences in the recognition model.}{
  Results for the syllable recognition experiment
  in which we allow syllables to repeat,
  or constrain the trajectory to contain only unique syllables.
  Plots show the accuracy, number of substitutions/insertions/deletions,
  timing differences, and memory accuracy.
  The dotted line in the memory accuracy plot denotes
  the chance accuracy value which depends on the syllabary size.
  Error bars represent bootstrapped 95\% confidence intervals
  from the 20 trials.}

Finally, we verified that the model
operates well in situations with
sequences of unique or repeated syllables.
As shown in Figure~\ref{fig:recog-repeat},
the overall accuracy is indistinguishable
whether repeated syllables are allowed or not.
All other measure are within confidence intervals as well.

\subsection{Scaling}
\label{sec:res-recog-scaling}

The number of neurons in the
syllable recognition model is
\begin{equation}
  N = N_{\text{in}} + N_{\text{SPA}} + N_{\text{iDMP}} + N_C,
\end{equation}
where these variables
are the number of neurons in
a subset of the model.

$N_{\text{in}}$ is the number of neurons
in trajectory input ensembles, which is
\begin{equation}
  N_{\text{in}} = 48 N_{|\text{in}|}
\end{equation}
where 48 is the number of gestures,
and $N_{|\text{in}|}$ is the number of neurons
per input ensemble.

$N_{\text{SPA}}$ is the number of
neurons for all SPA modules,
which is
\begin{equation}
  N_{\text{SPA}} = 3 |\text{SYLL}| N_{\text{AM}}
    + D(\text{SYLL}) N_{D(\text{SYLL})},
\end{equation}
where $|\text{SYLL}|$ is the number of
syllables in the syllabary,
$N_{\text{AM}}$ is the number of neurons
per associative memory ensemble,
$D(\text{SYLL})$ is the dimensionality
of syllable semantic pointers,
and $N_{D(\text{SYLL})}$ is the number of neurons
per syllable dimension.

$N_{\text{iDMP}}$ is the number of neurons
in iDMP networks, which is
\begin{equation}
  N_{\text{iDMP}} = |\text{SYLL}| \left( \left(\overline{D}_{\text{iDMP}}
  + 2 \right) N_D_{|\text{iDMP}|} + 20 \right),
\end{equation}
where $\overline{D}_{\text{iDMP}}$ is the average
number of gestures in each syllable,
$N_D_{|\text{iDMP}|}$ is the number of neurons
per iDMP dimension,
and 40 neurons come from ensembles with fixed size.
$N_C$ is the number of neurons
in the classifier,
which is $N_C = 20$.

With the parameters used
in the majority of the experiments above,
$N\approx 30000$ neurons,
with most of those neurons
in the iDMP populations
($N_{\text{iDMP}}\approx 22000$).
Note that these values are approximate
because the number of gestures per syllable
varies depending on the syllable,
which is randomly selected.\footnote{
  If we did not sparsify the syllable representation,
  then we would be using all 48 gesture dimensions.
  Typically, $\overline{D}_{\text{iDMP}}$ is around 6--7,
  which is a significant savings in terms
  of neural resources used.}
Like the production model,
the number of resources
grows linearly with the number of syllables.
However, it should be noted that the
linear growth is scaled by
$N_{D(|\text{iDMP})}$,
the number of neurons per iDMP dimension,
which we chose as 1000 through the experiments above.
If we assume a conservative syllabary
with around 1000 syllables,
and increase the dimensionality
of each syllable to 256,
the model uses
$N \approx 7.86 \times 10^6$ neurons,
which is more than the generous
estimate for the production model.
If we assume a more generous estimate
with 2000 syllables
and 512 dimensions
for syllable semantic pointers,
then the model uses
$N \approx 1.64 \times 10^7$ neurons.
The estimate represents approximately
606 mm$^3$ of cortex (0.606 cm$^3$),
which is still a small amount
considering the size of the brain areas
involved in sensorimotor integration,
but represent poorer scaling
than the other two models presented in this thesis.
