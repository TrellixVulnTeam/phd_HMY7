% For hyperlinked PDF, suitable for viewing on a computer, use this:
\documentclass[letterpaper,12pt,titlepage,oneside,final]{book}

% For PDF suitable for double-sided printing, use this:
%\documentclass[letterpaper,12pt,titlepage,openright,twoside,final]{book}

\newcommand{\package}[1]{\textbf{#1}} % package names in bold text
\newcommand{\cmmd}[1]{\textbackslash\texttt{#1}} % command name in tt font
\newcommand{\href}[1]{#1}

% This package allows if-then-else control structures.
\usepackage{xifthen}
\newboolean{PrintVersion}
\setboolean{PrintVersion}{false}

\usepackage{amsmath,amssymb,amstext}
\usepackage[pdftex]{graphicx}
\usepackage[dvipsnames,hyperref]{xcolor}
\usepackage{microtype}
\usepackage[round]{natbib}
\usepackage{rotating}
\usepackage{multicol}
\usepackage{tipa}
\let\ipa\textipa

% Set figure directories
\graphicspath{{../figures/background/}
              {../figures/model/}
              {../figures/previouswork/}
              {../figures/methods/}
              {../figures/implementation/}
              {../figures/results/}}

% Includes a figure. Usage:
%
% \fig{filename}{width}{caption}{shortcaption}
%
% - Filename will be used as label
% - Width is in proportion of column width
% - Shortcaption is optional
\newcommand{\fig}[4]{
  \begin{figure}[ht!]
    \centering
    \includegraphics[width=#2\columnwidth]{#1}
    \ifthenelse{\isempty{#4}}{\caption{#3}}{\caption[#3]{#4}}
    \label{fig:#1}
  \end{figure}}

% \newcommand{\scalefigtwo}[4]{
%   \begin{figure}[ht!]
%     \centering
%     \includegraphics[width=#3\columnwidth]{#1}
%     \includegraphics[width=#3\columnwidth]{#2}
%     \caption{#4}
%     \label{fig:#1}
%   \end{figure}}

% Font stuff
\usepackage{heuristica}
\usepackage[heuristica,vvarbb,bigdelims]{newtxmath}
\usepackage[T1]{fontenc}

% argmin and argmax aren't normal?
\DeclareMathOperator*{\argmin}{arg\,min}
\DeclareMathOperator*{\argmax}{arg\,max}
% typing mathbf over and over again is garbage
\newcommand*{\V}[1]{\mathbf{#1}}%
% binding
\newcommand*{\bind}{\circledast}
% We got room; big summations please!
\everymath{\displaystyle}

% For fancy boxes
\usepackage{tikz}
\usetikzlibrary{shapes,decorations}
\tikzstyle{roundbox}=[draw=black!80, very thick,
    rectangle, rounded corners, inner sep=10pt, inner ysep=20pt]
\tikzstyle{fancytitle}=[fill=black!80, text=white]
\newcommand{\probbox}[1]{%
  \begin{center}
    \begin{tikzpicture}
      \node [roundbox] (box){%
        \begin{minipage}{0.8\textwidth}
          #1
        \end{minipage}
      };
      \node[fancytitle, right=10pt, rounded corners] at (box.north west) {%
        \textbf{Problem statement}
      };
    \end{tikzpicture}
\end{center}}
\newcommand{\inoutbox}[2]{%
  \begin{center}
    \begin{tikzpicture}
      \node [roundbox] (box){%
        \begin{minipage}{0.9\textwidth}
          \begin{multicols}{2}
            \begin{center}
              #1
              \vfill
              \columnbreak

              #2
            \end{center}
          \end{multicols}
        \end{minipage}
      };
      \node[fancytitle, right=7.0em, rounded corners] at (box.north west) {%
        \textbf{Inputs}
      };
      \node[fancytitle, left=7.1em, rounded corners] at (box.north east) {%
        \textbf{Outputs}
      };
      \draw[very thick] (box.north) -- (box.south);
    \end{tikzpicture}
\end{center}}

% Fix error using \copyright with T1 fontenc
\renewcommand*\copyright{{\usefont{OT1}{lmr}{m}{n}\textcopyright}}

% Do this last
\usepackage[pdftex,pagebackref=false]{hyperref}
\hypersetup{
    plainpages=false,
    unicode=false,
    pdftoolbar=true,
    pdfmenubar=true,
    pdffitwindow=false,
    pdfstartview={FitH},
    pdftitle={Biologically inspired methods in speech recognition and synthesis: closing the loop},
    pdfauthor={Trevor Bekolay},
    pdfsubject={Biologically inspired speech models},
    pdfkeywords={speech} {spiking neural networks},
    pdfnewwindow=true,
    colorlinks=true,
    linkcolor=BrickRed,
    citecolor=PineGreen
    % urlcolor=cyan
}
\ifthenelse{\boolean{PrintVersion}}{
  \hypersetup{
    citecolor=black,
    filecolor=black,
    linkcolor=black,
    urlcolor=black}
}{}

% --- FIXME replace with geometry

\setlength{\marginparwidth}{0pt} % width of margin notes
% N.B. If margin notes are used, you must adjust \textwidth, \marginparwidth
% and \marginparsep so that the space left between the margin notes and page
% edge is less than 15 mm (0.6 in.)
\setlength{\marginparsep}{0pt} % width of space between body text and margin notes
\setlength{\evensidemargin}{0.125in} % Adds 1/8 in. to binding side of all
% even-numbered pages when the "twoside" printing option is selected
\setlength{\oddsidemargin}{0.125in} % Adds 1/8 in. to the left of all pages
% when "oneside" printing is selected, and to the left of all odd-numbered
% pages when "twoside" printing is selected
\setlength{\textwidth}{6.375in} % assuming US letter paper (8.5 in. x 11 in.) and
% side margins as above
\raggedbottom

% --- end replace with geometry

% The following statement specifies the amount of space between
% paragraphs. Other reasonable specifications are \bigskipamount and \smallskipamount.
\setlength{\parskip}{\medskipamount}

% The following statement controls the line spacing.  The default
% spacing corresponds to good typographic conventions and only slight
% changes (e.g., perhaps "1.2"), if any, should be made.
\renewcommand{\baselinestretch}{1}

% Force each section of the front pages to start on a recto page.
% Also ensure a page number is not printed on an otherwise blank verso page.
\let\origdoublepage\cleardoublepage
\newcommand{\clearemptydoublepage}{%
  \clearpage{\pagestyle{empty}\origdoublepage}}
\let\cleardoublepage\clearemptydoublepage

\begin{document}

% suppress the page number and headers/footers.
\pagestyle{empty}
\pagenumbering{roman}

\begin{titlepage}
  \begin{center}
    \vspace*{1.0cm}

    \Huge
    {\bf Biologically inspired methods in speech recognition and synthesis: \\
    closing the loop}

    \vspace*{1.0cm}

    \normalsize
    by \\

    \vspace*{1.0cm}

    \Large
    Trevor Bekolay \\

    \vspace*{3.0cm}

    \normalsize
    A thesis \\
    presented to the University of Waterloo \\
    in fulfillment of the \\
    thesis requirement for the degree of \\
    Doctor of Philosophy \\
    in \\
    Computer Science \\

    \vspace*{2.0cm}

    Waterloo, Ontario, Canada, 2016 \\

    \vspace*{1.0cm}

    \copyright\ Trevor Bekolay 2016 \\
  \end{center}
\end{titlepage}

% no headers, but yes page numbers (starting from ii.)
\pagestyle{plain}
\setcounter{page}{2}

\cleardoublepage
\phantomsection
\addcontentsline{toc}{chapter}{Author's Declaration}

\noindent
I hereby declare that I am the sole author of this thesis. This is a
true copy of the thesis, including any required final revisions, as
accepted by my examiners.

\bigskip

\noindent
I understand that my thesis may be made electronically available to
the public.

\cleardoublepage

\begin{center}\textbf{Abstract}\end{center}

Current state-of-the-art approaches
to computational
speech recognition and synthesis
are based on statistical analyses
of extremely large data sets.
It is currently unknown how these methods
relate to the methods that the human brain
uses to perceive and produce speech.
In this thesis,
I present a conceptual model, \textit{Sermo},
which describes some of the computations that
the human brain uses to
perceive and produce speech.
I then implement three large-scale brain models
that accomplish tasks
theorized to be required by Sermo,
drawing upon techniques
in automatic speech recognition,
articulatory speech synthesis,
and computational neuroscience.

The first model extracts features
from an audio signal by
performing a frequency decomposition
with an auditory periphery model,
then decorrelating the information
in that power spectrum
with methods commonly used in
audio and image compression.
I show that the features produced
by this model implemented
with biologically plausible spiking neurons
can be used to classify phones in
pre-segmented speech with significantly
better accuracy than the features
typically used in
automatic speech recognition systems.
Additionally, I show that this model
can be used to compare auditory periphery models
in terms of their ability to
support phone classification of pre-segmented speech.

The second model
uses a symbol-like neural representation
of a sequence of syllables
to generate a trajectory of premotor commands
that can be used to control
an articulatory synthesizer.
I show that the model
can produce trajectories
up to several seconds in length
from a static syllable sequence representation
that result in
intelligible synthesized speech.
The trajectories reflect the
high temporal variability
of human speech,
and smoothly transition between
successive syllables,
even in rapid utterances.

The third model
classifies syllables
from a trajectory of premotor commands.
I show that the model is able to
classify syllables online
despite high temporal variability,
and can produce the same
syllable representations
used by the second model.
These two models can be connected
in future work in order to implement
a closed-loop sensorimotor speech system.

Unlike current computational approaches,
all three of these models are implemented
with biologically plausible spiking neurons,
which can be simulated with neuromorphic hardware,
and can interface naturally with artificial cochleas.
All models are shown to scale
to the level of adult human vocabularies
in terms of the neural resources required,
though limitations on their performance
as a result of scaling will be discussed.

\cleardoublepage

\begin{center}\textbf{Acknowledgements}\end{center}

I would like to thank all the little people who made this possible.

\cleardoublepage

\begin{center}\textbf{Dedication}\end{center}

For Emily, who is the best.


\cleardoublepage

\renewcommand\contentsname{Table of Contents}
\tableofcontents
\cleardoublepage
\phantomsection

\addcontentsline{toc}{chapter}{List of Tables}
\listoftables
\cleardoublepage
\phantomsection

\addcontentsline{toc}{chapter}{List of Figures}
\listoffigures
\cleardoublepage
\phantomsection

% To include a Nomenclature section
% \addcontentsline{toc}{chapter}{\textbf{Nomenclature}}
% \renewcommand{\nomname}{Nomenclature}
% \printglossary
% \cleardoublepage
% \phantomsection
% \cleardoublepage

\chapter*{Typographical Conventions}
\addcontentsline{toc}{chapter}{Typographical Conventions}

??? IPA stuff
% https://www.tug.org/TUGboat/tb17-2/tb51rei.pdf

??? Also figure out a convention for things in Nomenclature?
Bold or italic?

\cleardoublepage
\phantomsection

% Change page numbering back to Arabic numerals
\pagenumbering{arabic}

\chapter{Introduction}

Speech is arguably the most important
medium through which humans communicate.
As the role of human-computer interaction
in modern society increases,
so does the need for computers
to recognize and synthesize speech.
Currently, state-of-the-art approaches
to computational
speech recognition and synthesis
are based on statistical analyses
of extremely large data sets.
While our understanding
of the neurobiology of speech has advanced
in recent years,
there currently exist no
models perceiving or producing speech
using neurobiological parts,
namely spiking neurons.

Recent advances in large-scale neural modeling
enable spiking neuron models of speech.
The Neural Engineering Framework (NEF; \cite{eliasmith2004})
provides a framework to implement
dynamical systems with networks of spiking neurons.
The Semantic Pointer Architecture (SPA; \cite{eliasmith2013})
provides a method to connect these dynamical systems
to symbol-like representations.
In terms of speech,
dynamical systems allow
the processing of
temporally varying sensory inputs and motor output.
The symbol-like representations in SPA
allow us to interpret the states
of these dynamical systems
as linguistically relevant concepts,
which connects the low-level
sensorimotor aspects of speech
with the high-level linguistic aspects of speech.
Taken together, the NEF and SPA
provide the essential tools required
to apply biological methods
to the fields of
speech recognition and synthesis,
which are currently dominated
by computational statistics.

In addition to the application of biological methods,
another motivation for this thesis
is the long-term goal of a closed-loop
speech system.
Currently, state-of-the-art approaches
to speech recognition and synthesis
are independent;
that is, speech recognition and synthesis
are seen as independent problems,
and are solved by independent systems.
In humans, however,
speech perception and production
overlaps and interacts in a closed-loop manner.
When learning to speak,
we use our perception of our own voice
to improve future vocalizations.
While to role of such a closed loop system
diminishes over time,
we still regularly monitor
our own speech and can adapt
when our speech perception or production
are perturbed through illness
or other means.\footnote{
  Here, and for the remainder of the thesis,
  we will refer to the sensory aspect
  of speech as ``speech perception''
  when referring to human speech understanding,
  and as ``speech recognition''
  when referring to computer speech understanding.
  Similarly, the motor aspect of speech
  will be referred to as
  ``speech production''
  when referring to humans speaking,
  and as ``speech synthesis''
  when referring to computers producing speech.}

The long-term goal of the line of research
described in this thesis
is a closed-loop speech system
that uses biological methods
to recognize and synthesize speech
with computers.
Such a system could produce its own training data,
and therefore would require
far less hand-labeled real-world data than
current purely statistical systems.
It would also enable interrogation
of how the system recognizes
and synthesizes speech,
and enable synergistic interactions
between researchers in linguistics
and neuroscience.
As the model develops and makes predictions
about the computations required for speech
and how they might be implemented,
neuroscientists can test those predictions
and update the model accordingly.

In order to achieve the long-term goal
of a neurally implemented closed-loop system
that can learn from itself,
we must first
envision what a large-scale closed-loop system
would look like,
and detail the computations required by that system.
We must also verify that neural implementations
of these computations
are possible with the available tools.
This thesis addresses these two concerns
by proposing a closed-loop speech system
called Sermo
(\underline{S}peech \underline{e}xecution and
\underline{r}ecognition \underline{m}odel \underline{o}rganism),
and constructing neural implementations
of three subsystems of Sermo.
In the conceptual Sermo model,
I have synthesized literature
in linguistics, neuroscience,
automatic speech recognition,
and articulatory synthesis
to present a closed-loop system
with well-defined computations.
In the three subsystems I have implemented,
I show that the NEF and SPA
can be successfully applied to
the computations required by the Sermo model,
providing the first steps
toward a neurally implemented closed-loop speech system.

The remainder of the thesis is organized as follows.
In Chapter~\ref{chapt:bg},
I review the relevant background informing
the conceptual Sermo model.
In Chapter~\ref{chapt:model},
I present Sermo,
which provides context for
the three neural models presented
in the subsequent chapters.
In Chapter~\ref{chapt:previouswork},
I review existing approaches
to solving the problems
proposed by Sermo
and address by the three neural models.
In Chapter~\ref{chapt:methods},
I provide details on the
mathematical techniques
used by the neural models,
and used to construct the neural models.
In Chapter~\ref{chapt:implementation},
I describe the three neural models.
In Chapter~\ref{chapt:results},
I propose metrics through which
to evaluate those models,
and present the results of collecting those metrics
while varying parameters on the neural models.
In Chapter~\ref{chapt:discussion}
I discuss the results,
summarize the contributions and predictions
of the models,
and discuss avenues for future work,
and finally,
conclude in Chapter~\ref{chapt:conclusion}.\footnote{
  Note that this chapter organization differs
  from a prototypical thesis
  because we make both
  conceptual contributions
  in the Sermo model
  and modeling contributions
  in the three subparts of Sermo
  for which we have made neural models.
  As such, the background chapter is designed
  to provide background for the conceptual model,
  which in turn provides context
  for the neural models.
  The previous work chapter provides
  background for the neural models,
  which are then explained fully
  in the methods and implementation chapters.}

\chapter{Background}
\label{chapt:bg}

??? this thesis draws upon and makes contributions
to what would traditionally be thought of a disparate fields:
phonetics and phonology from linguistics,
psychoacoustics from psychology,
knowledge representation
and artificial neural networks from computer science,
spiking neural networks and
synaptic plasticity rules from computational neuroscience,
dynamical systems and control theory
from systems design engineering,
concretely implemented in software
using principles from software engineering.
As such, not all relevant background can
be given equal coverage.
In this chapter, we assume
that the reader's background
is in computer science or engineering.
We therefore spend more time
reviewing background in linguistics
and neuroscience as they fall outside
of the domain of computer science and engineering,
and so can be considered the problem domain
that we will apply our assumed knowledge to.
% ???yikes reword

% First, we review relevant background
% for the problem domain of interest,
% human speech,
% which we organize
% into background relevant to speech recognition,
% speech production, and sensorimotor integration.
% Then, we review existing models
% of speech recognition and production,
% focusing on those using
% biologically inspired or constrained methods,
% and concluding with other integrated approaches
% that have incorporated speech recognition
% and production in a single system.

\section{Human speech recognition}

We limit our coverage
of human speech recognition
to the topics drawn upon
in the model that will be presented
in Chapters~\ref{chapt:model}
and \ref{chapt:implementation}:
basic ear physiology,
psychoacoustics,
and neurobiology.
Basic ear physiology is necessary
foundation for understanding
human speech recognition.
Psychoacoustics provides a quantitative account
of how the human auditory system
responds to incoming air pressure levels.
As such, it provides a method to verify
that our model responds to sounds
as humans do.
A primary assumption behind this work
is that the human auditory system
has evolved to be adept at processing speech;
psychoacoustics describes many of the ways
in which the human auditory system
manipulates sound,
which we aim to reproduce. % ???ugh this needs work
Auditory neurobiology describes both
the physical substrate and organization
of the system that we aim to emulate.
We do not aim to fully replicate
biological neurons,
but by using a simple approximation
of biological neurons,
we constrain our algorithmic choices
to those that could be implemented
in a biological system.
Similarly, by examining the organization
of the auditory brain structures,
we constrain the space of possible
network topologies
to those that match
a network that we know
to be successful.\footnote{One notable omission
  from our discussion
  of the human auditory system
  is the integration of information
  coming from both ears.
  There are binaural effects at many levels,
  and while a full human auditory model
  would incorporate these effects,
  at this stage of research
  we only aim to build
  a monaural speech system.}

\subsection{Ear physiology}

The human ear transduces fluctuations in
air pressure level to neural signals
that we interpret as sounds.
It does this by mechanically
separating the air pressure level fluctuations
into instantaneous frequency components,
which is the basic representation
that the brain receives.
Figure~\ref{fig:an-overview} illustrates
the major structures that will be discussed.

\fig{an-overview}{0.8}{??? caption}{??? shortcap}

The outer ear (pinna) funnels air
into the ear canal.
Functionally, it selectively boosts
air pressure fluctuations at
around 3 kHz (see Figure~\ref{fig:ear-tf}),
and modifies the air pressure wave
such that the directionality
of sound can be determined.

% \fig{ear-tf}{0.7}{??? caption}{??? shortcap}

In the middle ear,
air pressure fluctuations
cause the eardrum (tympanic membrane)
to vibrate;
these miniscule vibrations
cause three tiny bones,
the malleus, incus, and stapes to move.
The stapes sits on the oval window
of the cochlea,
such that when the eardrum vibrates,
the stapes moves in and out of the oval window,
causing disturbances in the liquid
in the cochlea
(see Figure~\ref{fig:cochlea}).

\fig{cochlea}{0.5}{??? caption}{??? shortcap}

The inner ear transduces vibrations
of the liquid in the cochlea
to electrical signals transmitted
to the brain.
As liquid in the cochlea
moves past the basilar membrane,
it causes it to deflect
based on the width and thickness
of the membrane at that point
in the cochlea.
The basilar membrane
is shaped such that
the base of the membrane
(near the stapes)
is deflected when
incoming vibrations have power
at high frequencies,
up to 20,000 Hz.
The membrane becomes sensitive
to lower and lower frequencies
going further down the membrane
to the apex
in the center of the cochlea,
which is deflected when incoming vibrations
have power as low as 20 Hz.

The basilar membrane's surface is covered
with hair cells that transduce
the deflections of the membrane
to electric current.
Stereocilia on the top of the hair cell
rests on or near the tectorial membrane
on the outside of the cochlea.
When the basilar membrane deflects,
the stereocilia's orientation
relative to the tectorial membrane changes,
which mechanically opens receptors
on top of the stereocilia,
allowing positive charged ions
(mostly potassium and calcium)
to enter the hair cell.
Inner hair cells synapse with
spiral ganglion cells,
which accumulate the continuously
changing voltage in the inner hair cells
and send action potentials
down the auditory nerve
conveying how much power
is present at the current moment
at the frequency characteristic
of that section of the basilar membrane.
Outer hair cells play a more nuanced
role in transduction.
Their activity is tuned
more broadly in both space and time,
and seems to act as a dynamic amplifier,
amplifying quiet sounds and
attenuating loud sounds.
This dynamic amplification allows
the human ear to have a wide dynamic range,
able to safely hear sounds from 0--130 dB,
which represents 13 orders of magnitude
of absolute air pressure.

\subsection{Psychoacoustics}
\label{sec:psychoacoustics}

While the human ear can be thought of
as a spectrum analyzer for air pressure waves,
there are important differences between
a straightforward Fourier analysis
and how the ear and early auditory brain regions
respond to air pressure waves.
Assuming that the human
auditory system has evolved to be
well-suited for speech,
it stands to reason that
capturing these differences
is important for building
a frontend for speech recognition.
Therefore, the psychoacoustical findings
discussed in this chapter
are presented as motivation for the
auditory periphery models described
in Section~\ref{sec:periphery-models}.

% ??? Either here or in previous section,
% introduce the idea of formants
% and format frequencies...

\subsubsection{Spatial psychoacoustical effects}

Sound pressure levels can be objectively measured
and expressed in terms of the logarithmic
decibel (dB) scale.
Human perception of loudness can help
determine the transfer functions
at different parts of the auditory system.
Early studies investigated
subjective loudness in response
to pure tones,
as tones of equal objective volume
are perceived as louder for higher frequencies.
The exact relationship between
sound pressure level and
loudness was originally proposed
by \cite{fletcher1933},
and further standardized
by the international standards organization
(ISO-226:2003);
see Figure~\ref{fig:el-curves}.
In general, perceived loudness,
as expressed in the phon scale,
increases for higher frequency sounds
up to around 7000 Hz.

% \fig{el-curves}{0.8}{??? caption}{??? caption}

Another scale, the sone,
quantifies the relative differences
to pure tones played with
different sound pressure levels.
Specifically, the sone scale
is linear with perceived loudness;
a sound played with twice the sone value
should be perceived as twice as loud.
The sone for a particular frequency is
\begin{equation}
  \text{sone} \approx 2^{\frac{\text{phon} - 40}{10}},
\end{equation}
above signals of 40 phon or louder.
% However, while it is easy to compare
% the relative volumes of two sound presented
% in close succession,
% absolute loudness is difficult to perceive.
% Kollmeier suggests we perceive only
% seven levels of loudness?

The difference in perceived loudness
across the frequency spectrum,
along with the organization
of the inner ear,
suggests that a better model
of the ear (instead of a spectrum analyzer)
is as a bank of narrowband filters.
Support for this model was provided
in \cite{zwicker1957},
who showed that when two tones
have frequencies that are close enough,
their perceived loudness sums,
suggesting that the two tones
are within the bandwidth of one of the
auditory system's filters.
\cite{moore1983}
summarized other attempts to
determine the bandwidth
of auditory filters
in a simple equation
describing the equivalent
rectangular bandwidth (ERB)
of auditory filters:
\begin{equation}
  \text{ERB} = 6.23 f^2 + 93.39 f + 28.52,
\end{equation}
where $f$ is the center frequency
of the auditory filter of interest.
See Figure~\ref{fig:erb} for a visualization of
this measure for frequencies of relevance
for the human auditory system.
Auditory filters are not rectangular,
but for simplification this equation
treats them as such;
despite this simplification,
the ERB measure has been productive and is still used
to determine the bandwidths
of filters in auditory periphery models.

% \fig{erb}{0.8}{???}{???}

Like volume,
humans perceive
the frequency of incoming sounds
in subjective ways.
The subjective perception of frequency
is referred to as pitch.
\cite{stevens1937} quantified
the relationship between
frequency and pitch
with a simple equation,
\begin{equation}
  \text{mel} = 2595 \log_{10} \left(1 + \frac{f}{700}\right),
\end{equation}
resulting in the mel scale
(see Figure~\ref{fig:mel}).
As with volume, humans are adept at
noticing when two tones differ in pitch;
however, absolute perceptions of pitch
have far less granularity
(musical notation, for example,
uses seven degrees of pitch per octave
and seven degrees of loudness).
Octaves are spaced from one frequency
to double that frequency;
harmonic tones are perceptually similar,
so pitches can only
be easily differentiated
relative to their position in an octave.

% \fig{mel}{0.8}{???}{???}

\subsubsection{Temporal psychoacoustical effects}

To this point, we have discussed
spatial psychoacoustical effects,
as they depend on the content of
the sound pressure level
at a moment in time.
There are also several temporal effects,
which are important for speech.
The first is that the length
of a sound affects its perceived volume.
Specifically, short sounds are harder to hear
(see Figure~\ref{fig:short-sounds}).
As a result, humans find
consonant sounds more difficult
to recognize in situations with background noise
than vowel sounds,
as consonants are shorter than vowels
\cite[Chapter~3]{everest2001}.
There is evidence that sounds with constant volume
are perceived as being monotonically louder
up to approximately 200 ms;
this finding suggests that
at some point in the auditory system,
the power for a given characteristic frequency
is integrated over a time window
of approximately 200 ms
\cite[p.64]{kollmeier2008}.

\fig{short-sounds}{0.4}{???}{???}

Another temporal effect is forward and backward masking.
When two sounds are made in quick succession,
the louder sound of the two can
``mask'' the quieter sound,
rendering it inaudible,
even though it would be audible
were it made in isolation.
When the masked (quieter) sound
precedes the masker (louder sound),
it is called backward masking;
the reverse is called forward masking.
Spatial masking also occurs
when two sounds differ in loudness,
and are within the bandwidth
of an auditory filter.

Our perception of echoes can also tell us
something about how the auditory system
responds to temporally transformed sounds.
We perceive all sounds arriving within
a certain time window as occurring at essentially
the same time;
for example, in an enclosed room,
we hear our own speech as we voice it,
and shortly thereafter when it is reflected
back to us off of the walls.
Yet, we perceive this as a single utterance,
a phenomenon known as auditory fusion.
\citeauthor{haas1972} found that auditory fusion occurs
best when similar sounds are separated by roughly
20 to 30 milliseconds.
When similar sounds are separated by 50 to 80 milliseconds,
we perceive them as discrete echoes.

Together, these temporal psychoacoustical effects
suggest that the incoming frequency information
from the filters implemented by the auditory periphery
are further temporally filtered
by auditory regions of the brain.

\subsection{Neurobiology}
\label{sec:recog-neurobio}

Brain structures in the primate auditory system
are organized in several parallel
hierarchical pathways.
Information from the auditory periphery
arrives at the brainstem. % ???more.
That information is relayed to the cortex
through the medial geniculate nucleus (MGn)
in the thalamus.
The MGn sends information to a part
of cortex in the temporal lobe
called the core.
In the context of auditory cortex,
the core is a small portion of cortex
in Heschl's gyrus (???)
% (???size in mm2)
that serves as the first stage
in the mostly feedforward hierarchy
of auditory cortical areas.
The core appears to be made up of
three subareas, ??? A1, R, and RT.
The core is surrounded by a set of areas
called the belt,
which primarily receives input from
the core.
% As such, there are ???three
% distinct areas in the belt,
% ???.
The belt provides input
to the parabelt.
% which is made up of ???.
See Figure~\ref{fig:corebelt} for a summary of
the neuroanatomical structures and connections
in the auditory hierarchy.

\fig{corebelt}{0.9}{???}{???}

The parabelt provides input
to several areas,
including brain areas
associated with language
in the superior temporal gyrus,
and frontal cortex which is associated
with working memory and executive function.
Historically, the most
salient linguistic area
in the superior temporal gyrus
was Wernicke's area (Brodmann area 22),
as lesions in this area result in
a type of aphasia in which
speech can be produced fluently,
but lacks meaning.
However, more recent hypotheses
concerning how auditory information
becomes discrete linguistic information
assigns roles to much larger portions
of the superior temporal gyrus
and the superior temporal sulcus.
We will discuss this in more detail
in Section~\ref{sec:sm-neurobio}.

While we do not yet know how auditory information
is transformed through the connections
described above,
we get some clues by investigating
how various auditory brain areas
respond to sounds with different
spatial and temporal properties.
The first clue
(which was used to determine the boundaries
between auditory cortical regions)
is that all of the areas
up to the parabelt are
tonotopically organized;
that is, each neuron in these areas
responds preferentially to
a part of frequency spectrum,
and cells close to one another
are likely to be sensitive
to the same frequency band.
In general, lower level regions
like the MGn and core
are more tightly tuned
to a smaller part of the frequency spectrum
(sometimes even more tightly tuned
than the auditory filters in the cochlea),
while higher level regions
like the parabelt
are more broadly tuned
to larger parts of the frequency spectrum.
% ???more, cites
See Figure~\ref{fig:aud-tonotopy} for visualizations
of the tonotopic organization
of primary auditory cortical regions.

\fig{aud-tonotopy}{0.6}{???}{???}

Neurons at various levels in the auditory hierarchy
respond to their selected frequency range
with different temporal characteristics.
In general, we can express the temporal
relationship between the incoming
frequency spectrum and the resulting
neural activity with a causal filter.
Considering that we also
have referred to the spatial tuning
of cells as a filter,
many auditory researchers
summarize the spatial and temporal
tuning properties of auditory neurons
using a single filter,
based on the
Spectral-Temporal Receptive Field (STRF).
STRFs summarize how a neuron responds
to incoming stimuli both spatially
and temporally \cite{aertsen1981}.
Unfortunately, despite many decades
of determining STRFs in
primate ??? cite and
human auditory cortex ???cite
the exact temporal properties of
each area are not known.
In general, STRFs have been found that
show delays of 5--250 ms
(see, e.g., \cite[Supplementary material]{mesgarani2014}).
See \cite{kaas1999,kaas2000,scott2003,semple2003}
for reviews.

It should be noted
that the information gleaned from
spectro-temporal tuning curves
still describes a linear relationship
between the incoming frequency spectrum
and how a particular neuron responds.
\citeauthor{tian2004}
suggest that belt neurons
in rhesus monkeys respond nonlinearly
to frequency-modulated sweep stimuli,
but \cite{kowalski1995}
found that cat A1 responses were linear.
It is difficult to evaluate
the responses of cells
as traditionally the STRF
is determined using simple stimuli
like white noise;
however, see \cite{theunissen2000},
for a method using complex sounds.
\cite{escabi2002}
found nonlinear responses
in approximately 40\%
of the midbrain inferior colliculus
neurons they recorded,
suggesting that nonlinearities
are widespread
throughout the auditory hierarchy.
Regardless, it is clear that
nonlinearities are more prevalent
the higher up you go in the auditory hierarchy,
though the types of nonlinearities
at each processin level are still unclear.
% we aim to provide some insight into
% nonlinearities needed in the auditory system
% in the model presented in this thesis.

In sum, we interpret
the auditory neurobiological literature
to support a model in which
the MGn and other lower brain areas
transmit frequency information
and delayed frequency information
to the core through direct connections,
and connections through intermediate
recurrently connected ensembles.
The core transmits nonlinearly transformed
frequency information to the belt,
which pools the responses from
a wider band of frequencies than the core.
The parabelt performs a similar transformation
and broadening of frequency selectivity.
Finally, the representations from the parabelt
are used by higher-level regions to
do extract discrete
acoustically and linguistically
important information.

\subsection{Automatic speech recognition}

Automatic speech recognition (ASR) has been
an active field of research
in artificial intelligence since 1952,
when \citeauthor{davis1952}
presented a system for recognizing
spoken digits by a single individual
with accuracy above 97\%.
This system was implemented
with special purpose hardware,
matching the spectral properties of
the sound to known spectral patterns.
From these humble beginning,
modern speech recognition systems
can transcribe speech
in realistic environments
with accuracy high enough
to be deployed commercially.
Two significant advancements
have enabled progress in ASR.

The first significant advancement
was the development of
Hidden Markov Models (HMMs)
and related techniques
for speech recognition.
Briefly,
HMM-based speech recognition models
attempt to find the word sequence
with maximum probability given
an audio waveform.
They do this through
a pipeline in which
short time-slices of the audio waveform
(called ``frames'')
are analyzed to yield lower dimensional
feature vectors
through signal processing techniques.
A sequence of feature vectors
is then decoded into words
using an HMM-based acoustic model
that uses sub-phonemic states
to emit phoneme labels,
which are then composed into words
using large corpora of
lexical and linguistic information
(see Figure~\ref{fig:asr} and
\ref{rabiner1989,gales2008} for reviews).
The first part of the pipeline
that produces feature vectors
is often called the ``frontend;''
the second part that uses
statistical methods to infer labels
is called the ``backend.''

\fig{asr}{0.6}{???}{???}

HMM-based systems
were among the first systems
successful enough to be
used commercially
(see \cite{huang2014}).
In 1989, \citeauthor{lee1989}
achieved a phone accuracy rate of % ??? look up: phone or word accuracy?
66.08\% on a subset of the
TIMIT speech corpus.
Four years later,
\citeauthor{lamel1993}
used a more sophisticated HMM-based system
to achieve 72.9\% phone accuracy on
all of the TIMIT speech corpus.
Improvements from then until
the mid-2000s were modest
(see \cite{lopes2011}).

Since 2006, a series of learning algorithms
and network structures in artificial neural networks (ANNs)
that are now referred to as ``deep neural networks'' (DNNs)
have been applied to
the domain of automatic speech recognition
with considerable success
both in academic research
and in commercial application.
In 2013, \cite{graves2013}
achieved
phone classification accuracy on TIMIT
of 82.3\%,
significantly better than the most sophisticated
HMM-based model's error rate of 72.9\%.  % ??? check!
% look at zen2007 zen2009

The architecture of DNN-based systems
is simpler than that of HMM-based systems
(see Figure~\ref{fig:dnn}
and \cite{hinton2012}).
Instead of distinct steps in which
an explicit statistical acoustic model
is evaluated and then combined with language models,
DNNs map directly from the
feature vector input representation
to the output representation of choice
(typically phonemes or word vectors),
which can be further refined with a language model.
This simpler architecture may be
responsible for much of DNN's success,
as its internal representations
are not locked to specific representations
like triphones.
However, the simpler architecture
does not necessarily mean that
using DNNs is straightforward;
the effort in developing ASR models
with DNNs shifted from
developing complicated architectures
to developing complicated learning algorithms
that have many parameters
which must be carefully tuned.

% \fig{dnn}{0.5}{???}{???}
% ??? DNN architecture
% something like slide 8 of Vanhoucke2013
% or check hinton2012

One additional benefit of DNN-based approaches
is that they have analogies to
how human recognize speech.
DNNs leverage simple computational units
(artificial neurons),
which operate in parallel
and communicate through unidirectional
weighted connections.
While the computations done by these neurons
and the learning algorithms that adjust
the connection weights
may not be directly implementable
under certain biological constraints,
mappings between DNNs
and biologically plausible spiking neural networks
are currently being developed
\cite{hunsberger2015}.

\subsection{Auditory periphery modeling}

Most ASR frontends do an ideal
power spectral analysis
with variants of the Fourier transform,
mimicking the function
of the human auditory periphery.
However, as discussed in Section~\ref{sec:psychoacoustics},
the human ear's frequency decomposition
is far from ideal.
Some differences in how the ear
analyzes the frequencies in sound
are due to the continuous nature
of the real world;
it is not possible for the ear
to maintain a perfect history
of the past 50 milliseconds of sound.
Some differences are due to
the nature of biological computation,
which is distributed, analog, and noise robust,
but not easily emulated with digital computers.
Finally, some differences
occur because they are advantageous
for some aspect of hearing
(possibly speech),
and have evolved over time through natural selection.
Unfortunately, it is difficult to know
the reason for each difference.
Historically, auditory periphery models
have attempted to reproduce
the auditory periphery as closely as possible,
whether or not that difference is advantageous
for speech recognition.

Several approaches to auditory periphery modeling
have progress in parallel over the past century.
Detailed mechanical models
like those reviewed in \cite{ni2014}
aim to model ear physiology
as accurately as possible.
Unfortunately, efficient implementations
of these models are not readily available.
Phenomological models, on the other hand,
aim to reproduce the output
of the human ear
as closely as possible
using simplified mathematical models;
efficient implementations of these
models are freely available
\cite{fontaine2011}.
We will focus only on phenomenological
models of the auditory periphery.

Generally, phenomenological auditory periphery models
are made up of a cascade of linear and nonlinear filters.
Together, the filters can model
how the sound pressure level
is transformed through the middle ear,
how it deflects the basilar membrane,
how those deflection are transduced
into electrical signals
by the inner hair cells,
and how those signals result in
action potentials traveling down
the auditory nerve.
Not all models include all of these stages.
For reviews of prominent
auditory periphery models and
the auditory-nerve response characteristic
reproduced by those models,
see \cite{lopez2005}
and \cite{lyon2010}.
The auditory periphery models used
in this thesis will be discussed
in further detail in
Section~\ref{sec:periphery-models}.

\section{Human speech production}

We limit our coverage
of human speech production
to the topics drawn upon
in the model that will be presented
in Chapters~\ref{chapt:model}
and \ref{chapt:implementation}:
basic vocal tract physiology,
phonetics and phonology,
and neurobiology.
Phonetics and phonology
provide insight into the basic units of speech.
While a full conversational system
would draw upon all areas of linguistics,
we believe that phonetics and phonology
provides the most relevant insights
when assessing speech from the level of
incoming air pressure levels.

\subsection{Vocal tract physiology}

The human vocal tract performs
the opposite role,
translating electrical brain activity
into muscle activations
which cause fluctuations in air pressure.

% \fig{vt}{0.4}{???}{???}
% % Generate from Kroger synth, label

The vocal tract consists of
the layrngeal cavity,
the pharynx, the oral cavity,
and the nasal cavity
(see Figure~\ref{vt}).
Air expelled from the lungs
is transformed
by these four structures
in turn to form the
specific air pressure fluctuations
that we interpret as speech.
Major changes to airflow
occur in the layrngeal cavity,
which contains the glottis.
The glottis is made up of
the vocal folds,
which are muscles that are able to vibrate rapidly,
and the opening between the vocal folds.
When the glottis is completely open,
air passes through mostly undisturbed,
resulting in low-frequency breathy sounds.
When the glottis narrows,
air passes through more turbulently,
resulting in higher frequency breathy sounds,
as in the /h???/ phoneme
in the English word /had???/.
In all voiced phonemes (e.g., all vowels),
the vocal folds vibrate,
resulting in a ``buzzing'' quality;
compare voicing ``ah'' to
forcing air out of your mouth
in the same vocal tract shape.
The exact action of the glottis
is responsible for many
of the subtleties in human speech,
such as intensity (speaking versus shouting),
frequency (low versus high pitch),
and quality (e.g., harsh, breathy, murmured, creaky).

The pharynx, oral cavity, and nasal cavity
can be moved into many different shapes
by the muscles of the vocal tract.
The shape of these remaining portions
of the vocal tract cause further turbulence
of the air passing through,
resulting in specific patterns
that we interpret as phonemes.
The areas of the vocal tract
that can be moved in order to
effect a linguistically relevant sound
are called articulators.
As will be discussed further
in subsequent sections,
the positions of these articulators
relative to each other
determines the phoneme
that will be voiced
when air passes through
the vocal tract.
The seven articulators are
as follows.

\begin{enumerate}
\item The \textbf{pharynx} is a tube at the back of the throat.
  It carries air from the larynx to the oral and nasal cavities.
\item The \textbf{velum} or soft palate permits or restricts
  access to the nasal cavity. It is also a constriction target
  for the tongue, as in the phonemes /k???/ and /g???/.
\item The \textbf{hard palate} is the hard surface
  at the roof of the mouth. It cannot move, but serves
  as a constriction target for the tongue.
\item The \textbf{alveolar ridge} is the area between
  the hard palate and the top front teeth. Like the hard palate,
  it serves as a constriction target for the tongue.
\item The upper and lower \textbf{teeth} are at the front of the mouth.
  While they primarily serve as a constriction target,
  the lower teeth are under muscular control,
  though few sounds utilize the motion of the lower teeth.
\item The \textbf{tongue} is a large, flexible, muscular structure
  that can reach several different constriction targets.
  The tongue is often divided into the tip, blade, dorsum, and root,
  though there are no clear dividing lines between these areas.
\item The \textbf{lips} are another important flexible, muscular
  articulator that is involved in many speech sounds.
  The lips can move toward other articulators to constrict airflow,
  or can become rounded to change the overall quality of a sound.
\end{enumerate}

Other parts of the vocal tract are important to speech;
for example, some sounds require air
to pass through the nasal cavity.
However, air is routed to the nasal cavity
through constrictions of the seven articulators above;
there are few little linguistically relevant
movements that can be done with the nose.
Exact definitions of the articulators
do vary between phoneticians,
and play an important role
in the design of a speech production system.
We will discuss this in further detail
in Section~\ref{sec:results-production}.

\subsection{Phonetics and phonology}
\label{sec:phonology}

Our treatment of phonetics and phonology
focuses on the hierarchical levels
in human speech,
paying particular attention to two
sets of characteristics in each level
and in each mapping between two levels:
temporal patterns and constraints,
and the existence of rule-like regularities.
For example,
if our model were to deal with sentences,
it would be important to know that,
temporally, sentences occur sequentially,
and cannot co-occur.
There are grammatical rules
about what words can be in a sentence,
and in what order they can be arranged,
but there are few rules placing constraints
between sentences---any sentence \textit{could}
follow any other sentence,
though there may be semantic issues.
Similar constraints exist at
the lower levels of language that
phonetics and phonology examine,
and these constraints will
inform the structure of the model
we present in Chapter~\ref{chapt:model}.

The specific hierarchy that we will
present in this section is pictured
in Figure~\ref{fig:prod-hierarchy}.
We will arbitrarily begin at
the lowest (rightmost???) level and work upwards
(though a presentation starting at
the highest level would be equally valid).
For clarity, we will use English for examples,
except when demonstrating that a particular
quality differs across languages.\footnote{
  When not explicitly cited,
  the information in Section~\ref{sec:phonology}
  should be considered
  basic phonetics and phonology
  that would be covered in an
  introductory undergraduate level
  class or textbook
  (I used \cite{roach2010} as a general reference).
}\footnote{The International Phonetic Alphabet (IPA)
  will be used this and subsequent sections.
  See Section~\ref{typography} for more details.}

% \fig{prod-hierarchy}{0.5}{???}{???}
% box-arrow diagram with the stuff from this sect

\subsubsection{Vocal tract gestures}

According to many articulatory phoneticians,
the fundamental unit of speech production
is the ``gesture,''
which describes some movement
of the articulators in the vocal tract
\cite{browman1989}.
Gestures are pre-linguistic,
in that a vocal tract gesture
does not necessarily produce
a linguistically interesting sound;
a set of possibly overlapping
vocal tract gesture can produce a phoneme,
which is smallest linguistically relevant unit.
There is evidence that children
learning to say their first words
often use the correct gestures,
but take time to order them correctly
to produce the desired utterance,
indicating that gestures develop
before language,
and are leveraged when
developing language
\cite{browman1989}.

Gestures are stereotyped movements
of sets of articulators in the vocal tract.
Each gesture has its own temporal dynamics;
that is, a gesture is a function defined
over space (what muscles are contracted)
and time (when contractions occur).
In general, a gesture can be thought of
as a point attractor
in which the position
of one or more vocal tract
articulators is parameterized.
A minimal gesture is parameterized by
the set of articulators engaged
by that gesture,
and the degree to which
the articulators are constricted;
for example,
widening and narrowing the velic aperture
is a gesture consisting of
the articulator involved
(the velum)
and degree to which it is constricted
(low degree for narrow, high degree for wide).
Other gestures require constriction location
and constriction shape parameters
to disambiguate them from other similar gestures;
for example, the tongue tip
is involved in gestures
that depend on location
(e.g., constriction at
the teeth, alveolar ridge, or hard palate)
and shape
(e.g., tongue tip can be straight
or curved backwards).
All gestures are also affected by
global parameters affecting all gestures,
most notably ``stiffness,''
which defines how quickly
the gesture attempts to reach
the point of attraction.

Gestures are rarely discussed in isolation
as phonemes only occur when certain
sets of gestures overlap
or occur in tight succession.
Therefore, gestures are often
grouped into gestural scores
(see Figure~\ref{fig:gs})
which embody the spatiotemporal
coordination of several gestures
to produce a linguistically
relevant sound.
The vertical axis of a gestural score
is the articulator set,
as although each set
can be used for multiple gestures,
it can only produce one gesture
at a time.
The horizontal axis of
a gestural score represents time,
though the time axis may be
stretched depending on
the global parameter, stiffness.
Gestural scores can also
be visualized as a graph,
as in Figure~\ref{fig:gs},
which highlights that
certain gestures must co-occur
or have tight temporal couplings
such that one gesture
must start as another gesture ends.

\fig{gs}{0.5}{???}{???}

As will be discussed in detail
in Section~\ref{sec:model-motorcontrol},
there is a mapping
between gestures and vocal tract articulators;
in human speech, the mapping links
gestures and vocal tract muscle activations.
In models of human speech,
the mapping depends on what articulators
are available in the articulatory synthesizer,
and how those articulators can be manipulated.
Unfortunately, few models make a clear distinction
between the set of gestures
and the mapping from gestures
to articulator trajectories,
resulting in there being
no agreed upon set of vocal tract gestures
in human speech.
Similarly, several approaches to
temporal sequencing of gestures
have resulted in different
definitions of gesture sets,
with different graph structures as
the one shown in Figure~\ref{fig:gs}.
We will discuss our interpretation
of gestures and gesture sets
in Section~\ref{sec:results-production}.
% Development of better human vocal tract models
% will help researchers
% determine a complete set of
% vocal tract gestures.

% ??? In a sense, vocal tract gestures
% can be thought of a subset of all
% motor synergies ???cite

\subsubsection{Phonemes}

The smallest linguistically relevant unit
is the phoneme, which describes
a short sound produced by one or more
vocal tract gestures
on the order of tens of milliseconds.
Phonemes are noted by their ability
to change the meaning of some word
when swapped in speech;
for example, the only difference between
the word ``bad'' and ``mad'' is the
consonant sound at the beginning,
and therefore those two consonant sounds
are each separate phonemes.
Even though different individuals
voice each phoneme differently,
the important quality is that
the particular sound is recognized
as a particular phoneme
in the context of an utterance.
A helpful analogy can be made between
phonemes in speech
and alphabetical letters in writing.
With no knowledge of English,
seeing the sentence,
``A bird has a wing,''
one might think that ``A'' and ``a''
are different letters.
However, with enough examples,
one could surmise that in every situation
where ``A'' is used, it appears at the
start of a sequence of letters,
and it could have instead
been replaced by ``a''
had it not appeared at the start of the sequence.
Therefore, ``A'' and ``a'' represent
the same underlying ``letter.''
Similarly, despite individual differences
in the pitch, speed, volume, and quality
of how one voices a particular phoneme,
that phoneme is still considered the same
if it plays the same role
in a linguistically relevant sequence of phonemes.

Generally, there are two types of phonemes:
vowels and consonants.
Vowels are longer sounds
made when the vocal tract is mostly open.
Consonants are shorter sounds
made when some part of the vocal tract
is constricted or transiently closed.
In almost all languages,
there are more consonant phonemes
than there are vowel phonemes,
though pronunciation varies significantly
between dialects,
and transcribing the full set of
phonemes in a language is
not a purely objective exercise.
For example, while most dialects
of English recognize 24 consonant phonemes,
General American English has been transcribed
as having 16 vowel phonemes,
while Received Pronunciation English
has been transcribed as having 25 vowel phonemes
\cite{wells1982}.

Vowel phonemes occur when air is freely
moving through the open vocal tract.
The shape of the vocal tract determines
the quality of the sound.
Three factors influence vowel vocal tract shape:
openness, backness, and roundedness.
Openness refers to the general position
of the jaw (open or closed) and tongue (low or high);
backness refers to the position of the
tongue relative to the back of the mouth;
and roundedness refers to
whether the lips are rounded.
Roundedness can vary independently
of the other factors;
therefore, each vowel sound has a rounded
and unrounded variant.
Openness and backness are partially coupled,
such that when the vocal tract is open,
it must be mostly (but not completely) back.
The three possible extremes, then,
are /a???ipa/ (open, back),
/i???/ (closed, front),
and /u???/ (closed, back).
Most of the remaining vowel sounds
can be expressed as being
a blend of one of these three vowel sounds
(see Figure~\ref{fig:vowels} for a visualization).

Pure vowel phonemes (also called monophthongs)
occur when the glottis phonates
and the vocal tract stays
in one of the positions already described.
Diphthongs phonemes occur
when the vocal tract
transitions or ``glides''
from one vocal tract position
to a second vocal tract position
during phonation;
e.g., the English ???
in ??? is a diphthong.
Triphthong phonemes,
in which three vocal tract positions
are visited in sequence,
also occur in some languages;
e.g., in RP English
the word ???
is often voiced with the
triphthong ???.

% \fig{vowels}{0.8}{???}{???}
% should be able to make with TIPA

Consonant phonemes occur when some point
of the vocal tract is constricted.
The place and manner of constriction
determines the consonant that will be uttered.
Place refers to the location in the vocal tract
that becomes constricted.
For example, in bilabial consonants,
both lips come together,
as in /m???/ and /b???/;
in velar consonants,
the tongue moves toward the velum
(i.e., soft palate),
as in /k???/ and /g???/.
Manner refers to how the vocal tract
is constricted in that location.
For example, in a plosive,
the vocal tract is completely closed
at the place of articulation;
air compresses behind the place of constriction,
and is then released,
producing the sound recognized
as a plosive, like /t???/ and /k???/.
In fricatives,
the articulators move close together
such that there is a narrow channel
for air to pass through.
The narrow channel causes
the hissing sounds associated
with phonemes like /s???/ and /f???/.
Several other places and manners exist;
those defined by the IPA are shown
in Figure~\ref{fig:consonants}.

% \fig{consonants}{0.8}{???}{???}
% also from TIPA;
% http://www.ling.ohio-state.edu/events/lcc/tutorials/tipachart/tipachart.pdf

While no phoneme's pronunciation is consistent,
consonant pronunciations can vary more than vowels
because consonant sounds are produced in the context
of the vocal tract position for
the prior or upcoming vowel sound.
In the word /bat???/ for example,
the /b???/ and /t???/ sounds
occur in the context of /a???/.
\cite{kroger1993} theorizes that
each consonant sound is composed
of speech gestures which force some
articulators to change in a particular way,
but allow other articulators,
like those involved in the vocalic sound,
to change freely.

The vowels and consonants described so far
represent the most common phonemes used
in daily speech.
However, as with most aspects
of phonetics and phonology,
there are many examples that do not
match the convenient criteria listed above.
For example, /w???/, as in ``weep''
is a phoneme consonant that is articulated
with a mostly open vocal tract,
like a vowel;
for this reason, it is sometimes called a semivowel.
Some languages use clicks,
in which inward suction releases a complete constriction
resulting in a loud consonantal sound.
However, we will not simulate
these and other phonemes
that are either uncommon or not present in English;
instead, we will state when a part of our model
will require further work
in order to handle these phonemes,
and when the model can be easily
adapted to handle these phonemes.

\subsubsection{Syllables}

Syllables are groups of one or more phonemes
with a well-specified structure.
They consist of a loud vocalic center,
with optional quieter consonantal components
before and after the center.
Unlike other levels of organization,
the phonemes in a syllable
may not be strictly sequential;
some phonemes may co-occur
when voicing a syllable.

The phonemes that make up a syllable
are typically grouped into
the onset and rime,
where the onset is a cluster of
consonant phonemes,
and the rime is a vowel phoneme
(called the nucleus)
and an optional cluster of consonant
phonemes (called the coda).
In English, the onset is also optional,
though this is not true in all languages.
Figure~\ref{fig:syllables} summarizes this grouping.
Note that other ways to describe
syllables exist
(e.g. the \textit{mora}; \cite{otake1993}),
but we adopt this method
as it is the most common
for Germanic languages like English.

\fig{syllables}{0.6}{???}{???}

Phonetically, syllables are a useful
level of organization because
the higher-level aspects
of an utterance---rhythm, prosody and stress,
for example---are easier to analyze
in terms of their effects on
sequences of syllables rather than
on sequences of phonemes;
it is easy to distinguish
changes in pitch and volume
within and between two syllables,
but not between two phonemes,
because they occur quickly
and can co-occur.

One aspect of the stress of an utterance
is the weight of a syllable.
``Heavy'' or ``strong'' syllables
have a branching rime,
or a branching nucleus,
meaning that they
end in a consonant,
or contain a long vowel or diphthong,
respectively.
``Light'' or ``weak'' syllables
do not have branching rime or nucleus,
and in general are shorter
and quieter.
In English, most vowel phonemes
can only appear in either
heavy or light syllables;
for example, the schwa, /e???/,
can only appear in light syllables,
and diphthongs like /ae???/
can only appear in heavy syllables.
The weight of a syllable determines
whether or not it can be stressed
in an utterance;
specifically, only heavy syllables
can be stressed.

Like all levels of phonetics and phonology,
the above description is a useful simplification
of a more complex phenomenon.
In terms of syllable production,
the onset-nucleus-coda grouping
is not always sufficient.
In English, syllables that may have once
been typical VC syllables have morphed
into ``syllabic consonants,'' in which
the vowel is no longer present;
for example, ``bottle'' is often
pronounced /bottl???/,
where /l???/ is a syllabic consonant.
In terms of syllable recognition,
agreement between native English speakers
is surprisingly low when
asked to segment an utterance
into its component syllables.
Yet, the meaning of the utterance
is understood by all subjects.
Therefore, while the syllable
may be a useful level of organization
for producing utterances,
its role in recognizing them
may be limited.

% ??? Here or somewhere else, summarize
% the ``mental syllabary'' from Levelt and Wheeldon, 1994
% also Burki 2015

\subsubsection{Tone-units}

??? move to discussion

The final level of organization
that we will examine is the tone-unit.
The tone-unit allows us to examine
suprasegmental aspects of speech.
A tone-unit is made up of a serially ordered
sequence of syllables.\footnote{Many
  phoneticians consider a tone-unit to be
  composed of ``feet,'' where a foot is
  a single unit of rhythm.
  However, feet are mostly used when describing
  non-conversational utterances,
  such as those found in music and poetry.
  In this thesis we focus on speech
  as a means of conveying linguistic information,
  and therefore ignore the concept of feet.}

The structure of a tone-unit is similar
to a syllable, except its component parts
are syllables instead of phonemes,
and its components are serially ordered.
A tone-unit must contain a tonic syllable
(sometimes also called the nucleus),
and can optionally contain one or more syllables
in a pre-head, head, or tail section.
The pre-head consists of all syllables
before the first stressed syllable
in a tone-unit.
The head consists of all syllables from
the first stressed syllable
to the tonic syllable.
The tonic syllable is the most significant
syllable in the tone-unit because
pitch changes in the tonic syllable
will occur relative to the tonic syllable;
the tonic syllable is not necessarily
the loudest or most prominently stressed
syllable in the tone-unit,
though it does always contain
a stressed (and therefore heavy) syllable.
The tail consists of all syllables
following the tonic syllable.

% ??? tone-unit structure figure
% maybe don't worry for now

Not all heavy syllables are necessarily stressed;
stress is hypothesized to occur when
more muscle activation is used
to voice a particular heavy syllable
???cite?.
Stress is perceived as a heavy syllable
that is louder, longer, and with
a different pitch or quality compared
to other syllables.
In some languages, rules govern
which syllables receive stress;
for example, in French,
the last syllable in a word is
always stressed.
In English, each word defines
its own stress pattern,
and each utterance can add
additional stresses
that emphasize some words over others.
Additionally, stress is not a binary quantity;
English is typically thought to contain
three stress levels
(primary stress, secondary stress, and unstressed).

Intonation, on the other hand,
is a more straightforward phenomenon to model.
Intonation is the use of pitch changes
to 1) express emotions and attitudes,
2) impart prominence on stressed syllables,
3) exaggerate grammar and syntactic structure,
and 4) clue listeners into what information
is novel and what is thought to be already known.
While intonation can be thought to also include
body language and other prosodic characteristics,
we will focus only on pitch changes.
The tone of a tone-unit
is the overall trajectory of pitch
during the tone-unit.

There are a limited set of possible pitch trajectories
(i.e., tones) in English.
While different sources identify different trajectories,
we will adopt the conventions of ???below
which note six possible pitch trajectories:
high fall, low fall, high rise,
low rise, fall-rise, and rise-fall
(see Figure~???).
The same sequence of syllables
can change its meaning dramatically
by using a different pitch trajectory,
or by changing the position of the
tonic syllable within that pitch trajectory.
The meaning of each pitch trajectory
changes depending on the utterance in question;
we will not investigate meanings further
in this work, as we focus on
sound reproduction rather than on
linguistic meaning.

??? Quick plot of pitch trajectories
% again, deal with later

% cite https://notendur.hi.is/peturk/KENNSLA/25/IPBE/SHELL/25/nucleus.html
% or Cruttenden

It is important to note that when talking about
the pitch trajectory of an utterance,
pitch is always relative to the
upper and lower range of a particular speaker.
In the pitch trajectory plot in Figure~???,
horizontal lines show the upper and lower ranges.

These six trajectories in Figure~???
interact with the pre-head and tail
in predictable ways.
The pre-head is usually low,
but can be high in front of a low stressed syllable.
The tail trajectory can be predicted
by the trajectory of the tonic syllable;
for example, if the tonic syllable falls,
the tail remains low;
if the tonic syllable rises,
the tail continues to rise.
The head, on the other hand,
is independent of the tonic syllable;
it may remain at a high or low level,
or it can rise or fall like
the trajectories associated with the tonic syllable.
The combinations of head and tonic syllable
pitch trajectories also contributes
to the varying meanings conveyed
by the pitch trajectory of a tone-unit.

Finally, one note about
the absence of ``words'' in our treatment
of phonetics and phonology.
While words are clearly an important concept
that we will use colloquially in this thesis,
they are a linguistic construct
rather than a phonological one.
However, as directly applied
to speech recognition and synthesis,
words are more readily recognized
than sequences of syllables with no meaning.
Fortunately, the pioneering work
of ???allen94 which summarizes ???
gives a direct relationship between
the recognition rates of nonsense syllables and words,
which can be used to adjust
recognition rates for
non-linguistic speech.

\subsection{Neurobiology}
\label{sec:prod-neurobio}

Traditionally, speech production
was tightly linked to Broca's area,
a sizable region of the dominant
(usually left, for right-handed individuals)
hemisphere of the frontal lobe,
the inferior frontal gyrus.
Paul Broca made this area famous
through study of two patient
with profound difficulty producing language;
one patient could only produce the word ``tan,''
while the other had only a vocabulary
of five words which he would utter repetitively.
% While speech neurobiology is
% difficult to study in depth
% (invasive studies are typically deemed unethical
% in species that can express themselves
% through language),
While we now have a much better understanding
of the brain regions
involved in speech production,
how these regions are connected
and what information they communicate
to connected areas is still
the subject of active research,
which we hope to contribute to
through generating predictions
from our model.

Modern understanding of
the neurobiology of speech production
views Broca's area among several
areas in the inferior frontal gyrii
of both brain hemispheres
that temporally controls a series
of tightly timed activations
of motor cortex.
\cite{flinker2015}
showed through Granger causality analysis
that Broca's area is activated
before motor cortex,
and while the motor cortex is
engaged in producing vocal tract movements,
Broca's area is relatively silent.
We will discuss its role in
transforming incoming linguistic
information from sensory areas
to representations useful for production
in Section~\ref{sec:sm-neurobio},
and instead focus here on
how motor areas effect vocal tract movements.

Controlling vocal tract muscles
in order to produce recognizable speech
can be thought of as a special case of
general motor control,
in which we aim for relatively
low dimensional targets
(i.e., position in three-dimensional space)
with high dimensional controls
(i.e., muscle contractions).
Despite the number of tightly coordinated
muscle activations that come about
when we make an intentional action,
we do not have to explicitly
think about those individual muscle activations,
which has led to the concept
of motor synergies.
Motor synergies are learned sets of
muscle activations that result
from activation of relatively few
neurons in motor cortex.
Synergies can be flexibly weighted
and combined;
for example, if one has a synergy
for reaching forward and one for
reaching to the left,
activating both synergies
equally could effect
a reach diagonally forward-left.
Many different variations
on the idea of motor synergies exist,
and neurobiological support
for these variants
can be found if certain
analysis techniques are used
(see \cite{tresch2009}
for a review in general
and \cite{smith2004,smith2006}
for evidence of speech synergies).
We limit our discussion
of motor synergies to the abstract concept,
and do not choose a specific variant.
As such, our primary interest
is in what brain areas correspond
to the level of the speech hierarchy
(see Figure~\ref{fig:prod-hierarchy}).

Motor synergies, then,
can be thought of as the highest level
of the human speech motor hierarchy.
There is evidence that these motor synergies
exist in the ventral sensorimotor cortex (vSMC).
\cite{bouchard2013} found that activations
of vSMC during the voicing of a syllable
represent activation of vocal tract articulators,
and that the vSMC is organized somatotopically
(i.e., distinct regions of the vSMC
influence different vocal tract articulators).
\cite{breshears2015} built on this work
by directly stimulating vSMC in humans
who were undergoing brain surgery.
vSMC stimulation in different subsections
resulted in movement of vocal tract articulators.
They also found that the somatotopic organization
of vSMC varied for each subject,
but all subjects' vSMCs were organized
in dorsal-ventral articulator order.

\cite{dewolf2010} integrated neurobiological research
into a model of the motor hierarchy
that involves the interaction of
supplementary motor areas,
basal ganglia, cerebellum,
primary motor cortex,
and nuclei in the brain stem and spinal cord
(see Figure~\ref{fig:noch}).
We assume that vocal tract movements
use the same hierarchy,
with vSMC as the top level of that hierarchy
(i.e., it is the supplementary motor area
for the vocal tract).
\cite{wildgruber2001} provides support
for the role of basal ganglia
and cerebellum in speech,
and \cite{brown2009}
provides support for
the role of primary motor cortex
in speech.

% \fig{noch}{0.4}{???}{???}
% % just steal

The motor synergies for vocal tract movements
provide a method for implementing
vocal tract gestures,
but there remains a link
between the phonemic
speech hierarchy in Figure~\ref{fig:prod-hierarchy}
and the motor hierarchy
in Figure~\ref{fig:noch}.
Neurobiologically,
that missing link appears to be
implemented by the anterior insula,
which is located in the lateral sulcus
separating the temporal lobe from
the parietal and frontal lobes.
As reviewed in \cite{ackermann2004},
the insula seems to be responsible
for the coordination of the up to 100 muscles
that shape the vocal tract during speech
and non-speech vocalizations.
They (and others like \cite{ivry1998})
hypothesize that the mapping from
from vocal tract gestures to articulator movements
is lateralized such that
the left anterior insula operates
discretely on a fast phoneme-level timescale,
while the right anterior insula
operates continuously on a slower
syllable or tone-unit level timescale,
suggesting that these two timescales
would be represented
with separate neural resources
in a neural model of the speech system.

\section{Sensorimotor integration}

Unlike speech recognition and production,
speech-specific sensorimotor integration
is difficult to interrogate through
traditional psychological or linguistic means,
as it is most directly observable
during development and other forms of learning.
We briefly review leading theories
of how we develop and learn speech,
then look at neurobiological studies
of sensorimotor integration.

\subsection{Development and learning}

There are two findings in the
speech development literature
that are relevant to our model.
The first is that in
early developmental phases,
sensorimotor learning
takes place rapidly
and in ways that are difficult
to achieve later in development;
these phases are sometimes referred
to as ``critical'' or ``sensitive'' periods.
Babies up to around a year of age
have a sensitivity for phonemes;
they become increasingly sensitive
to the phonemes that they hear regularly,
and increasingly less sensitive
to phonemes that they hear rarely or never
\cite{kuhl2008}.\footnote{
  Interestingly, it has been shown that
  remaining sensitive to phoneme
  that are not regularly encountered
  is predictive of poor language performance
  at two years of age \cite{kuhl2008}.}
Up until puberty,
children have a sensitivity for
acquiring language in general;
after puberty, formal instruction
is less effective,
and speaking foreign languages
without an accent is more difficult
\cite{hurford1991}.
This finding suggests that
parts of our model
that include learning
through changes in synaptic connection weights
may have qualitative or large quantitative
differences at different developmental stages.

The second finding is that
auditory learning
and speech understanding precede
speech production.
While this may seem obvious,
analogies between how humans acquire speech
and how birds acquire songs suggests that
the vocal babbling of children
may be directed toward the goal of
matching speech templates learned
through auditory input
\cite{bolhuis2010}.
This finding suggests that
we develop a repository of speech templates
that we attempt to match when
learning to produce speech.

Despite the prominent role of sensorimotor integration
in development,
it is not the case that this system
plays no role in the adult speech system.
One influential study by \citeauthor{houde1998)
investigated sensorimotor learning
by manipulating adult speech
through a device that recorded
a subject's voicing a syllable,
adjusted the sound such that the syllable's vowel sound
was adjusted in a predictable way,
and played the adjusted sound back through headphones.
Subjects were able to compensate
for the adjustment such that
subsequently uttered syllables
more closely matched the desired syllable,
and that adjustment generalized
to other syllables that had not yet
been tested.

\subsection{Neurobiology}
\label{sec:sm-neurobio}

Unlike recognition and production,
in which we have a sensor and actuator to measure,
sensorimotor neurobiology
can only be investigated
in neural recordings from cortical areas.
While we do not yet
have a full understanding of
how the brain's speech recognition
and production systems are organized
and interact,
two influential theories have provided
a fruitful framework for many studies.

The first theory
is the dual-stream model of speech processing
proposed by \cite{hickok2007}.
This model is conceptually similar
to the dual-stream model of visual processing,
in which the ventral stream
is involved in object recognition
and the dorsal stream is involved
in guiding actions.\footnote{The dual-stream
  model for visual processing has traditionally
  been though of as the ventral ``what'' pathway,
  and the dorsal ``where'' pathway,
  assigning the dorsal stream
  the role of determining object locations
  \cite{ungerleider1982}.
  However, more recent studies have made
  strong links between dorsal stream
  activity and motor actions
  \cite{andersen1997,rizzolatti1997,rizzolatti2003}.}
In the speech processing case,
the ventral stream
is involved in lexical processing
(i.e., the meanings of incoming sounds),
and the dorsal stream
is involved in sensorimotor action mapping.
In this thesis, we develop
a model of parts of the dorsal pathway,
but developing an
interacting ventral pathway
would be a logical extension.

\cite{hickok2007}
suggest that the dorsal stream
is strongly involved in
speech development,
as development requires auditory feedback
to guide the creation and fine-tuning
of motor pathways to effect speech;
the role of the dorsal pathway may
be reduced over time as motor pathways
reliably produce recognizable speech
\cite{schmidt1975,doyon2003}.
Anatomically, the dorsal stream begins with
the dorsal superior temporal gyrus (STG; i.e., Wernicke's area)
and the superior temporal sulcus (STS),
which both project to
a parietal-temporal boundary area in the Sylvian fissure
(called area Spt).
They propose that the Spt is a sensorimotor interface,
which also takes in input from other sensory modalities
(i.e., visual and somatosensory information).
Spt then projects to higher-level articulatory areas,
including the posterior inferior frontal gyrus (pIFG; i.e., Broca's area),
premotor cortex (PM) and the anterior insula.
As was discussed in
Sections~\ref{recog-neurobio} and \ref{prod-neurobio},
the STG has clear roles in auditory processing,
and the anterior insula has clear roles in
motor processing;
area Spt, therefore, is a strong candidate
for where sensorimotor integration might occur.

\cite{hickok2009}
provide human fMRI evidence that
area Spt is significantly active
during both sensory and motor aspects
of a speech task.
However, \cite{cogan2014}
performed electrocorticography
in human epilepsy patients
and found electrodes that activated
for both sensory and motor aspects
of a speech task across all areas
of the dorsal stream
(including STG, middle temporal gyrus, PM,
and inferior frontal gyrus).
Clearly, there is a large overlap
between the information required for
sensory processing and the information required
for motor processing.

The second influential theory
in speech neurobiology is the
motor theory of speech perception.
The central tenet of this theory
is that perceiving speech
is perceiving vocal tract gestures
\cite{galantucci2006}.\footnote{
  The motor theory of speech perception has a long history
  and has made some claims that are not supported by evidence.
  We will ignore these claims, and focus only on
  the central ideas which have empirical support.}
This makes a hypothesis about
the internal representations of speech,
and is therefore difficult
to interrogate directly,
but several experiments appear
to support this hypothesis.
First, we are able to integrate
gestural information from multiple modalities;
the direct perception of gestures
influences how we perceive the sound.
A well-known demonstration of
this phenomenon is the McGurk effect
\cite{mcgurk1976},
in which the same audio waveform
in perceived as representing a different syllable
when accompanied by a video
showing the lip gesture that produces
the audio waveform.\footnote{
  The McGurk effect can be experienced firsthand through
  a video demonstration at
  \url{https://www.youtube.com/watch?v=G-lN8vWm3m0}.}
Second, in a speech reaction task,
humans react $\sim$26 ms quicker
when imitating a cue syllable
compared to when reacting
to the cue syllable
with a predetermined syllable
\cite{fowler2003}.\footnote{
  For clarity, in the task, subjects
  shadowed a recording voicing the ??? /a/ phoneme,
  which switched to either /pa/ /ta/ or /ka/??? at an
  unspecified time.
  Subject either voiced the syllable that they heard,
  or one of /pa/ /ta/ or /ka/??? which was
  kept constant regardless of what was heard.}
Typically, reaction time experiments
in which the subject produces the same reaction
regardless of the cue (e.g., pressing a button)
achieve faster reaction times
than experiments in which the subject
has a choice of reactions
and the cue is arbitrarily related
to the reaction.
If the cue is nonarbitrarily related
to the response
(e.g., the cue appears on the right side
when a button on the right is to be pressed)
then reaction times are closer to those
of the simple experiments.
The fact that imitating the syllable
(i.e., having a choice of three possible responses,
but receiving non-arbitrary cues)
is faster than a simple reaction
implies that the the information gathered
from the cue is directly related
to the motor response;
since vocal tract gestures are thought
to be the basic unit of speech motor action,
then it is likely that vocal tract gestures
are also an early unit
in speech recognition as well.

Neurobiologically,
the motor theory of speech recognition
cites similar empirical evidence as
the dual-stream model;
i.e., there are brain areas (like Spt)
that are selectively activated
when engaged in both speech recognition
and speech production tasks,
so it is likely that they share
a representation which might be vocal tract gestures.
Indeed, while much of the auditory literature
attempts to decode phonemes
from these brain areas
(e.g., ??? cite),
neural activities representing vocal tract gestures
would necessarily be capable of
decoding phonemes due to the fact that
vocal tract gestures are responsible
for effecting movements that produce
those phonemes.
While they do not make this link directly,
\cite{mesgarani2014}
showed support for the motor theory of speech recognition
by showing that in the STG,
which is typically thought of
as an auditory-specific brain region,
there is topographic organization
based on the acoustic properties
of the phoneme being listened to
(see Figure~\ref{fig:stg-phones}).
These acoustic properties are
directly linked to the articulators
producing the phoneme
(e.g., labial versus dorsal consonant phonemes)
and therefore it is plausible that
the STG represents vocal tract gestures.

\fig{stg-phones}{0.8}{???}{???}

Given that neurobiological evidence can
support both theories,
we therefore adopt the view that
the dorsal auditory stream
represents vocal tract gestures explicitly,
though this may or may not be the case
in the ventral stream.
Our model, which purports to replicate
functions of the dorsal stream only,
will reflect these two theories
by explicitly representing vocal tract gestures.

\subsubsection{Automatic speech recognition with production information}
\label{sec:asr-prod}

In terms of higher-level biological organization,
almost all speech recognition systems
can be mapped most directly
to the ventral stream of the human speech system,
as they attempt to decode
lexical information in order to
transcribe speech to text.
In this thesis, we aim to model
the dorsal stream of the human speech system,
which we hypothesize follows the
motor theory of speech recognition,
meaning that it uses vocal tract gestures
as the basic unit of representation.

Several labs have investigated
using measurements
related to speech production
as part of a speech recognition system
(see \cite{king2007} for a review).
The primary advantage of this approach
is that speech production information
is not locked to each phoneme,
so it is not as sensitive to the
differing acoustic realizations
of the same phoneme.
The existence of large datasets
with continuous acoustic
and articulator position recordings
(e.g., \cite{westbury1990,wrench2000,steiner2012})
has spurred development of
systems that use both
acoustic and articulatory information
as input to an ASR system.
\cite{zlokarnik1995}
used continuous acoustic and articulatory
information with an HMM-based system
and reduced word error rates
by more 60\% (relative).
\cite{eide2001}
augmented the feature vector
in a standard HMM system
articulatory features and
reduced word error rates
by 34\% (relative).
Similar studies have been done
with traditional artificial neural networks
(e.g., \cite{kirchhoff2002})
and dynamic Bayesian networks
(e.g., \cite{stephenson2000,stephenson2004}).

Deep neural networks have been used for
a related task,
acoustic-articulatory speech inversion.
Here, articulatory information
is decoded from acoustic information;
that is, the input of the network
is still an acoustic feature vector,
but the desired output
is articulator positions.
\cite{uria2011}
obtained a root mean square error
of 0.95 mm on the MNGU0 dataset,
which is 0.04 mm better than prior
efforts using Guassian mixture models
\cite{richmond2009}.
The extracted articulatory information
could be used by another ASR system
to lower phone or word error rates.

The aforementioned results
use continuous articulator positions.
Some work has also been done
relating this work to
vocal tract gestures.
For example,
\cite{zhuang2008,zhuang2009}
showed that the vocal tract gesture score
can be estimated
from the continuous articulator positions,
which can then be used
to do word classification with
over 80\% accuracy on a synthesized dataset.
\cite{nam2010,nam2012}
showed that a gesture score
can be estimated from natural speech
through a landmark-based time alignment procedure;
however, this model assumed that the
words in the utterance were known \textit{a priori},
and applied their technique primarily
for annotating data sets with time-aligned gesture scores.
Surprisingly, we have been unable to find any literature
attempting to extract vocal tract gesture scores
from acoustic information directly,
nor through a pipeline in which
articulator positions are decoded,
and then gesture scores are estimated
from those articulator positions.

\subsection{Integrated recognition and production systems}
\label{sec:bg-diva-kroger}

For the purposes of this thesis,
we consider a system to be integrated
if speech recognition and production
share internal representations
(i.e., representations other than
the incoming audio waveform or text),
and are connected such that
the output of one system
can be used to improve
the performance of the other system.
By this definition,
the most widely used
speech recognition and production systems
are not considered integrated
even though they can be used
in a conversational manner;
virtual agents like Apple's Siri
can both recognize and produce speech,
but recognized speech is converted
to text before speech synthesis occurs,
and the speech production system
cannot be modified over time
as it relies on a large corpus
of recorded speech.

The most well known integrated system
is called DIVA
(Direction Into Velocities of Articulators;
see Figure~\ref{fig:diva};
\cite{guenther1995,guenther2004,guenther2006,guenther2006a}).
DIVA consists of an articulator synthesizer,
and several interconnected artificial neural networks
that drive the articulatory synthesizer
and process auditory and somatosensory feedback
in order to improve future synthesized speech.
DIVA is not pre-programmed with a repository
of programs to effect various sounds;
instead, it mimics human speech development
by learning to control the synthesizer
through an initial babbling phase
in which random articulator movements
are associated with auditory
and somatosensory feedback.
DIVA has been used to explain
speech production phenomena
like motor equivalence, contextual variability,
and anticipatory and carryover coarticulation
\cite{guenther1995,guenther2003,nieto2005}.
Additionally, all of the neural networks
in the model have been mapped
to brain regions that are hypothesized
to perform similar functions
as the neural networks in the model.

\fig{diva}{0.7}{???}{???}

DIVA is a productive research tool
for studying speech development.
However, I do not believe that DIVA
in its current form
can scale to maintaining conversations
with adult vocabularies.
Currently, DIVA is focused on speech production,
so while it can incorporate acoustic feedback
to learn better speech production,
there is no clear path
to incorporate speech recognition
and higher-level linguistic capabilities.
Additionally, DIVA currently represents
stored speech plans with one artificial
neuron per speech plan,
where a speech plan can correspond to
a single phoneme, a syllable, or one or more words.
In \cite{guenther2006a},
they note that ``it is expected that
premotor cortex sound maps in the brain
involve distributed representations
of each speech sound.''
Using such distributed representations,
as we do in our model,
would require several changes to DIVA's
structure and learning algorithms.
While these changes are possible,
they would represent a significant
modification that may affect
empirical results obtained
with the current version of DIVA.

In addition to using highly localist representations,
DIVA has other biological plausibility issues.
It represents auditory feedback using the first
three formant frequencies,
but does not provide an account of
how the brain might extract those
frequencies from the incoming air pressure levels.
The artificial neural networks
consist of homogeneous neurons
with non-spiking linear activation functions,
though the learning procedure
uses a local Hebbian learning rule.
Auditory and somatosensory feedback
is provided to the model with no delay,
making the learning procedure
much easier than a learning procedure
that must deal with delayed feedback,
as happens in the real world.

\cite{kroger2009} proposed
a large integrated model
similar to DIVA in that
it is composed of
several artificial neural networks
and learns to produce syllables
through a babbling stage
in which feedback from
auditory and somatosensory systems
tunes the weights between
neural populations
(see Figure~\ref{fig:kroger}).
The primary improvements in
Kr\"{o}ger's model compared to DIVA
are the introduction of a phonetic map,
and a motor planning network.

\fig{kroger}{0.7}{???}{???}

In DIVA, auditory and somatosensory feedback
is directly associated with a cell
in the speech sound map,
meaning that each sound must be learned separately.
This approach does not generalize across
similar sounds, nor will it scale
to large vocabularies of speech sounds.
Kr\"{o}ger's model overcomes this limitation
by explicitly decoding phonetic information
from auditory feedback.
It then uses that phonetic information
to generate explicit motor plans.
Motor plans act as parameterizations
of the possible vocal tract gesture scores
that can be produced by the model;
each motor plan is defined by
five parameters which define
the vocalic state,
the gesture-performing articulator,
and the location of articulation.
These five parameters are used
to generate vocal tract gestures,
which are then mapped onto
movements of articulators
in a three-dimensional vocal tract model.

In Kr\"{o}ger's neurocomputational model,
phonetic information is extracted
in an unsupervised manner
using self-organizing maps
\cite{kohonen1982,kohonen2007},
which cause
local clusters of neurons
in the phonetic map to become active
when incoming auditory information
(in the form of the first three formants)
is similar to certain patterns
seen during training.
The maps are able to
discriminate between three vowel phonemes
and three consonant phonemes;
it is not clear whether
maps trained in this way
can scale to discriminating
between the number of phonemes
in a realistic language (over 40).
However, it is clear that an approach
which uses a finite set of intermediate representations
(i.e., phonemes)
between auditory input
and learned motor sequences
will scale better than DIVA's approach
that does not use an intermediate representation.

% ??? Update Kroger section for Cao and 2015 paper

In contrast to these two models,
the model that we present
in the subsequent chapters does not focus on
modeling speech development,\footnote{
  The methods used by DIVA and Kr\"{o}ger's
  neurocomputational model to emulate
  speech development could be adapted
  and added to our model.
  However, since these problem have existing
  solutions in these two models,
  we have focused on other aspects
  of speech recognition and production
  as being of greater scientific interest.}
and instead implements
portions of the human speech system
not modeled by these two models,
and aims to be more biologically realistic.
Biological realism is achieved
by more sophisticated neural network models
that use distributed representations
that can scale to the level
of adult vocabularies,
and by using a human auditory model
to process incoming sounds
rather than preprocessing the sound
to obtain the first three formants.
In addition, while Kr\"{o}ger model
improves on the DIVA model by
introducing phoneme representations,
we follow the motor theory
of speech representation by
explicitly decoding vocal tract gestures
from incoming sound,
which makes integration between
the perception and production aspects
of the model more straightforward.

\chapter{Conceptual model}
\label{chapt:model}

The long term goal of the research
presented in this thesis
is a complete neurally realized model
of the human speech system,
which is able to maintain
natural conversations.
Implementing such a system
requires the collaborative efforts
of many domain experts.
In this chapter, I present
a conceptual model of the human speech system
that I believe can be implemented
in biologically plausible spiking neural networks.

The model, dubbed Sermo
(\underline{S}peech \underline{e}xecution and \underline{r}ecognition
\underline{m}odel \underline{o}rganism),
is a synthesis of the background material
summarized in the previous chapter.
The goal of Sermo
is to break the larger
problem of human speech
into concrete subproblems
that have existing partial solutions,
or provide a reasonable challenge
for machine learning
and computational neuroscience researchers.
Pragmatically for this thesis,
Sermo provides context for
the subproblems for which
we have implemented concrete solutions
in subsequent chapters.

\section{Sermo description}
\label{sec:sermo}

\fig{sermo}{0.9}{Architecture of the Sermo model.}{
  Architecture of the Sermo model.
  The goal of the model is to explicitly define
  computations needed
  for speech recognition and synthesis.
  Solid lines indicate connections that
  are necessary for Sermo to function.
  Dashed lines indicate connections that
  we hypothesize to exist,
  but are not essential for Sermo to function.
  See text for more details.}

Sermo is split into five interacting systems,
as summarized in Figure~\ref{fig:sermo}.
\begin{itemize}
  \item \textbf{Auditory feature extraction}
    converts incoming audio waveforms into
    features suitable for linguistic processing
    and sensorimotor integration.
    This system is analogous to the frontend
    of an ASR system,
    and to the auditory periphery and
    early auditory pathways in the human speech system.
  \item \textbf{Non-auditory sensory modalities}
    gather relevant information
    from non-auditory sources
    to inform or correct other systems.
  \item \textbf{Sensorimotor integration}
    uses sensory information to
    recognize and predict the motor intentions
    of speakers, including one's self.
    This system is analogous to the
    dorsal stream in the human speech system.
  \item \textbf{Linguistic processing}
    converts sensory and sensorimotor information
    into lexical representations
    suitable for high level linguistic
    and cognitive reasoning.
    This system is analogous to
    the backend in an ASR system,
    and to the ventral stream
    in the human speech system.
  \item \textbf{Speech motor control}
    uses sensorimotor information
    to generate motor commands
    that drive an articulatory synthesizer,
    which produces audio waveforms.
\end{itemize}

Further details on each of these
systems follow in the subsequent sections.

In addition to the overall organization
of subsystems and how they interact,
we also impose the following constraints
on Sermo in order to make explicit
that it is modeled after a biological system,
and should be able to interact
with biological systems naturally.

\begin{itemize}
  \item Subsystems must operate in a continuous, online fashion.
  \item Subsystems must be implementable in biologically plausible
    spiking neurons.
  \item Learning rules may only use local error information,
    and must compute error signals with spiking neurons.
\end{itemize}

Another way to summarize these constraints
is to emphasize that Sermo
is a mechanistic model
of the human speech system.
We are not only concerned with
recognizing and synthesizing speech
with human-like accuracy,
we also wish to gain insight
into what information is required
to recognize and synthesize speech.

Solutions to subproblems that do not
meet these criteria
(e.g., statistical models)
are still a crucial component
of progress in this model.
Some parts of the model are
necessarily statistical,
and the information being transformed
may not be easily explained
by a label like ``lexical representation.''
In these situations,
however, it is still a critical research activity
to adapt statistical models that work in
discrete time, with rate based neuron models,
or with idealized learning rules like backpropagation
to meet the above criteria,
for the following reasons.

\begin{itemize}
  \item Models meeting these criteria can be compared directly
    to experimental data at multiple levels,
    from single unit activity recordings to behavior.
  \item Models meeting these criteria can be implemented
    in neuromorphic hardware.
\end{itemize}

Direct experimental comparisons enable
iterative development of the model.
As Sermo subsystems are implemented,
they will be used to make testable predictions.
As these predictions are tested,
the model can be invalidated
or updated to incorporate new empirical results.

Neuromorphic hardware
presents a plausible avenue
to real time interaction
between humans and Sermo instances.
Many of the subsystems already implemented
run at many times slower than real time,
despite the use of recent general purpose
CPUs and GPUs,
and reasonably efficient algorithms.
While incremental advances
in these technologies may
bring subsystems closer to real time speeds,
a fully integrated system
with all subsystems interacting online
is intractable for the foreseeable future.
Neuromorphic systems that are designed
to simulate millions of spiking neuron models
in real time
are currently under development
(e.g., \citealt{furber2013,benjamin2014}),
and may make a full Sermo instance
possible in the next few decades.

A final note about
the general architecture of Sermo
and its subsystems
is that we will focus on an
``adult'' version of Sermo
that would be used in a conversational setting
that does not require learning.
We treat each subsystem as a separate
module that is defined primarily
by the input it expects,
and the output it provides;
however, during development
and learning,
there would be a proliferation of
interconnections between modules
providing error information.
Detailing these connections
in each subsystem and across subsystems
would be an important area
for further developing the model,
but we believe that presenting
a static endpoint for learning
is important as a first step
before considering
how that endpoint might come about.
Regardless, we will provide a sketch of
how learning can be incorporated in Sermo
in Section~\ref{sec:syllable-learning}.

\subsection{Auditory feature extraction}
\label{sec:model-ncc}

\inoutbox{Audio waveform}{Auditory feature vector}

The auditory feature extraction system
preprocesses audio waveforms
in order to support
linguistic and sensorimotor analysis.
As has been shown through
the gap between
automatic speech recognition
and speech perception,
the temporal characteristics
of speech
make these problems
difficult to solve
with traditional computers.
Digital computation is precise and discrete;
human speech is noisy and continuous.
We hypothesize that the
parallel and distributed nature
of biological computation
is central to its
ability to process temporal information.

Sermo defines three subproblems
that must be solved
by the auditory feature extraction system:
\textit{spectral analysis},
\textit{decorrelation},
and \textit{temporal transformation}.
In spectral analysis,
the incoming auditory signal
is projected into frequency space
by analyzing the spectral density
of the signal in the recent past.
The spectral density of the signal
is expected to have high
correlation between nearby frequencies.
Generally, these correlations
make feature extraction more difficult,
so the decorrelation step
projects the spectral density
onto a set of basis functions
that are closer to orthogonal.
Typically, fewer basis functions
are used than frequency components,
making decorrelation also a form
of dimensionality reduction.
Finally, while the information
in the spectral densities
and decorrelated vectors
contain information from
a certain time window,
the temporal characteristics of speech
may require that more
temporal information is available
at the current moment.
Temporal transformations
maintain past information
to filter spectral
and decorrelated spectral information
for better linguistic
and sensorimotor information decoding.

\subsubsection{Spectral analysis}

We theorize in Sermo that spectral analysis
is performed by a model
of the auditory periphery.
An auditory periphery model
emulates the human ear
up to the signals
traveling down the auditory nerve.

There are many existing
models of the auditory periphery
that meet the constraints
imposed by Sermo (see Section~\ref{sec:sermo}).
Each auditory periphery model
captures certain psychoacoustic phenomena
at a particular computational cost.
In general, these two characteristics
are traded off;
the more accurately the ear is modeled,
the most costly the model is to simulate.

It is difficult to assert \textit{a priori}
which psychoacoustic phenomena
are important for speech.
It would be reasonable to assume that
the majority of what the ear does
is important for an individual's survival;
however, we process many types of sound,
not just speech.
An ideal auditory filter model
for the spectral analysis subsystem
in Sermo
would capture all of the phenomena
that are important for speech,
but no additional phenomena
that increase the computational cost
of the model.

\subsubsection{Decorrelation}

Decorrelation techniques like
linear filters and the discrete cosine transform
are widely used in digital computing
for compression \citep{khayam2003}.
In compression, a signal is projected
onto a set of basis functions
and only the coefficients are stored.
An approximation of the original signal
can later be recovered by linearly combining
the basis functions weighted by those coefficients.
The coefficients are much smaller
to store than the original signal.
A similar argument can be made
to explain why automatic speech recognition systems
decorrelate spectral information in their frontends:
by operating on the coefficients of a set
of basis functions,
the set of parameters that must be optimized
by the backend is reduced.

While this line of reasoning
does not necessarily imply that
the brain also explicitly decorrelates
spectral information
provided by the auditory periphery,
it is clear that biological systems
aim to minimize energy expenditure whenever possible.
Learning through synaptic plasticity
takes more energy than does
typical signal transmission.
Since much of the speech system
is developed and refined
through synaptic plasticity,
using decorrelated representations
as early in the speech system as possible
reduces the brain's energy expenditure
over its lifetime.
Therefore, we hypothesize
that a system implemented
with a biologically plausible substrate,
like Sermo,
decorrelates spectral information
provided by the auditory periphery.

\subsubsection{Temporal transformation}

The extent to which the recent past
is used to perceive speech
at the current moment
is of crucial importance,
yet is difficult to study systematically.
In the brain, spectrotemporal receptive fields
have been measured using
tone ramps and other temporally varying signals,
revealing neurons that appear to be active
hundreds of milliseconds
after activity at
the cell's characteristic frequency.
However, it is not clear if those
neurons are involved in speech,
nor is it clear what transformation they are implementing.
In automatic speech recognition,
the length of the audio frame
and how much it increments on each timestep
is often chosen arbitrarily,
or uses values known to work.

Until recently,
the most successful ASR systems
were based on Hidden Markov Models (HMMs),
which are based on the Markov assumption:
only information in the current state
is required to reason about the next state.
Despite the fact that speech
is highly temporally correlated,
these systems have been relatively successful,
due in large part to
a great deal of engineering effort
applying HMMs to speech,
primarily because of its temporal nature
(see e.g., \citealt{bahl1983,rabiner1989,lee1989}).
One such improvement for HMM-based ASR systems
is to include the first and second temporal derivatives
in the state representation.
Current state-of-the-art systems
are based on hierarchies of bidirectionally connected
recurrent neural networks (RNNs),
which are able to maintain information about
previous states through recurrent connections.
It is reasonable to assume that
a more flexible representation
of the recent past
is part of why these networks
fare better than HMM-based systems
which have had decades of incremental improvements.

Further study into how much temporal information
is maintained by deep RNNs,
and how that temporal information is transformed
would benefit Sermo
and other speech systems.
In particular, expressing deep RNNs'
temporal transformations as
linear time-invariant (LTI) filters
would be of particular interest,
as LTI filters can be efficiently implemented
with spiking neural networks
(see \citealt{eliasmith2004}).
Alternatively, LTI filters
commonly used in engineering applications
can be experimented with
to investigate whether decoding accuracy
is improved in downstream
linguistic and sensorimotor systems.

The system described above
is primarily feed-forward.
It is important to note that
a complete auditory processing system
would match the brain,
which has feedback connections
between almost all layers
in order to refine each layer's ability
to provide useful output downstream layers
(see Figure~\ref{fig:auditory}).
Feedback is an undeniably important
component of a full system that operates
for long periods of time with sensors
that are constantly changing and degrading.
In this work, however, we assume
that the auditory periphery does not change
its ability to process sounds over a long timescale,
and therefore we do not include corrective feedback
signals between layers in the auditory system.
This type of feedback could be added
to this system in future work.

\subsection{Non-auditory sensory modalities}

Biological systems are remarkably adept
at integrating information
from different sensory modalities
in order to guide behavior.
In speech,
we use visual information
from other speakers to improve perception,
and somatosensory information
from vocal tract articulators
to improve production.
A fully realized version of Sermo
would include these non-auditory sensory systems.

\subsection{Linguistic processing}
\label{sec:sermo-linguistic}

\inoutbox{Auditory feature vector \\ Non-auditory sensory information \\
  Production information \\ Phoneme strings}{Lexemes \\ Syllable targets}

As summarized in Section~\ref{sec:sm-neurobio}
the ventral stream of the human speech system
represents high-level linguistic features of speech.
In Sermo, we hypothesize that
the linguistic processing system
\textit{decodes lexical information}
from sensory and production information,
and \textit{provides syllable targets}
to be voiced to the sensorimotor integration system.
Much of these processes require
higher level linguistic processing,
which would be implemented by
models of higher cortical areas.

Decoding lexical items from speech
has been the primary topic of research
in ASR systems for decades.
As such, we are confident that
the methods used in ASR systems
can be implemented in a mechanistic model
suitable for incorporation with Sermo,
as has been done for similar
visual classification systems
(e.g., \citealt{hunsberger2013,hunsberger2015}).
Once classified, lexical items
will be used in higher level linguistic
systems that will associate
semantics (as in \citealt{blouw2013,blouw2015})
and syntax (as in \citealt{stewart2014,stewart2015})
to these representations.
As these higher level systems
have been developed with
the Semantic Pointer Architecture
(SPA; see Section~\ref{sec:methods-spa}),
we hypothesize that the linguistic processing system
produces lexical representations
in the form of semantic pointers.
However, we do not further constrain these representations
as constraints will come from the needs of
higher-level linguistic and cognitive systems,
which are not currently part of Sermo.

The output of these higher-level systems
is constrained by the needs
of Sermo's sensorimotor integration system.
Specifically, Sermo requires that
the output of the
linguistic processing system
be in the form of syllable targets.
Therefore, for the portion
of the linguistic system connected
to the sensorimotor system,
semantic pointers representing
sentences or words must contain
syllable representations that can be
queried through unbinding.

We follow the theoretical work
embodied in the WEAVER++ model
\citet{roelofs2000,roelofs2008,roelofs2014}
explaining how syllable targets might be generated
in the linguistic processing system.
Higher cortical areas are responsible
for providing a sequence of phonemes,
which are composed into syllables on the fly
through a prosodification process.
Not included in the WEAVER++ model
is the notion of syllable sequencing,
which we believe is necessary.
The syllables constructed by the
prosodification process
may or may not be constructed
at the same rate as syllables
are being uttered.
Specifically, it is likely that
the next syllable,
or the next several syllables,
is conceptualized well before it is uttered.
For this reason, a process for
temporally sequencing syllables
is necessary.
Syllable targets are therefore provided
to the sensorimotor integration system
at the appropriate times.

\subsection{Sensorimotor integration}
\label{sec:model-sm}

\inoutbox{Auditory feature vector \\ Non-auditory sensory information \\
  Syllable targets}{Syllable classifications \\ Production information}

As summarized in Section~\ref{sec:prod-neurobio},
the dorsal stream of the human speech system
represents both sensory and motor information
when perceiving and producing speech.
In Sermo,
we hypothesize that
the sensorimotor integration system
is responsible for decoding production information
from sensory information,
and maintaining two types of associations.
First, sensory input,
both auditory and non-auditory,
is used to \textit{decode the production information}
that generated the sensory input.
Second, that decoded production information
is associated with
static syllable representations;
we call this \textit{trajectory classification},
as each syllable requires
a time series of sensory features
on the order of hundreds of milliseconds
to be reliably classified.
Finally, syllable targets are associated with
trajectories of speech production information;
we call this \textit{trajectory generation},
as a time series of vocal tract movement commands
are necessary to voice a syllable.

As in the auditory feature extraction system,
a great deal of effort is involved in
dealing with the temporal nature of speech.
Were the sensory and motor representations static,
the associations in the sensorimotor system
could be accomplished
with simple associative memories
(see \citealt{eliasmith2012,eliasmith2013}).
Instead, we are effectively associating
sensory trajectories with motor trajectories.
The key insight
that enables these associations in Sermo
is to use an intermediate syllable representation
that can be transmitted between
brain areas and queried
for the full trajectory.
A convenient side effect of
the intermediate syllable representation
is that it also enables
top-down influence from
other systems,
most notably the linguistic processing system.
It is not reasonable to expect
linguistic areas to communicate
production information directly
to motor cortex;
however, it is reasonable to expect
that the lexical representations
in the linguistic system
can be used to generate syllable targets
which are used by sensorimotor areas.

Note that Sermo follows
the theoretical and empirical
speech production work
in \citet{levelt1994,levelt1999,cholin2004},
which says that the
speech production is organized
at the level of syllables.
While this literature
is sufficient reason to use syllables,
Sermo also allows us to
provide two additional reasons
for organizing speech production
at the level of syllables:
enabling transfer learning,
and efficient use of neural resources.

Syllables enable flexible speech
because we can use learning to
fine-tune the trajectory generated by any
one syllable representation individually,
and that fine-tuning transfers
to other words using that syllable.
For example,
if Sermo can improve its ability
to voice the syllable \ipa{[bA]},
then the quality of all words
containing the syllable \ipa{[bA]}
improves.
Had we chosen word representations,
then this transfer learning
would not be possible.
On the other hand,
had we chosen phoneme representations,
then it would be difficult
to fine tune trajectories at all,
as the motor trajectory of a phoneme
is dependent on the
previous and next phonemes,
and we therefore could not fine tune
the trajectory of each phoneme individually.

Syllables minimize the expenditure of neural resources
because syllables are mostly independent of one another,
and because there are a finite number of syllables.
To illustrate why these reasons are important,
let us assume that a certain amount of
neural resources (neuron and synapses)
are associated with the basic unit of representation.
We will denote this as $n$,
and assume that this amount is the same
regardless of the representation used.
If we choose phonemes as the basic representation,
then we have a small number of basic units;
for the sake of argument,
say around 50 phonemes,
which is slightly above the amount
in English.
Because the motor trajectory
of a phoneme is dependent on the
phonemes before and after it,
we need neural resources not only for
each phoneme, but in the worst case,
also for all possible
permutations of three phones;
i.e., we would need $50^3n$ or 125,000$n$ resources.
On the other hand, if we chose
words as the basic unit of representation,
we would not have to worry about
between word interactions,
but the number of words in a language
is much larger than the number of syllables
in a language.
Using a conservative estimate
for the number of English words
and syllables
\citep{schiller1996},
word representations would use approximately
30,000$n$ resources,
compared to just 500$n$ resources for syllables.
Syllable representations are clearly the most tractable
given these assumptions.

So far, we have not been explicit
about our representation of production information.
As will be discussed in more detail
in Section~\ref{sec:prev-classification},
we could use a discrete representation
of production information
(i.e., vocal tract gestures)
or a continuous representation
of production information
(i.e., vocal tract articulator positions).
Unlike in the choice of
using syllable representations,
the number of possible vocal tract gestures
and the number of possible articulator positions
is similar,
assuming a reasonable parameterization
of the vocal tract articulators
as is done in all articulatory synthesizers.
Therefore, we consider the choice
of production information representation
to be an empirical question
that we aim to explore with Sermo.

\subsubsection{Trajectory generation}

In trajectory generation,
the syllable target
is used to generate
a time series of production information.

There are several criteria
that a trajectory generation system
must meet in order to be
useful in Sermo's sensorimotor integration system.
First, the system must be able to produce
trajectories with high degrees of freedom.
There are at least twenty degrees of freedom
for both vocal tract gestures
and articulator positions.
Despite the high dimensionality,
the neural resources required
for each syllable trajectory
must be relatively low,
as there may be up to a few thousand
of these trajectory generators
in an adult speech system.

Second, the trajectory generators
must be flexible with respect to
the timescale at which
they generate a trajectory.
There is significant variability
in how we voice syllables.
Syllables are often elongated
or shortened for emphasis,
and during a stressful situation,
all of a speaker's syllables may be
voiced quickly.
A trajectory generator must be
able to produce appropriate
production information
at speeds that vary
between and within syllables.

Third, the trajectory generators
must be able to operate
in tight succession,
and possibly even overlap in time.
Speech is remarkably continuous;
to robustly segment speech into
syllables often requires
speech to be slowed down
or overenunciated.
Therefore, if our notion of production
requires representation at the level of syllables,
then the motor trajectories
resulting from those syllables
must be able to blend naturally into one another
without obvious pauses between
different syllables,
or multiple utterances of the same syllable.

\subsubsection{Trajectory classification}

In trajectory classification,
we are essentially solving the inverse problem
of trajectory generation.
Given some time series of production information,
we must infer the syllable
that would produce that trajectory.

As an inverse problem,
the trajectory classification system
has the same challenges
as the trajectory generation system.
The trajectories to be classified
will have relatively high degrees of freedom;
the trajectories may not advance uniformly
through time;
and subsequent syllables will follow
in quick succession,
even possibly overlapping.

\subsubsection{Decoding production information}

In the trajectory classification system,
we assume that the trajectories we are classifying
are trajectories of production information.
It may also be possible to directly classify
sensory information into syllables;
however, we follow the motor theory of speech perception
\citep{liberman1985}
which theorizes that production information
is directly inferred from auditory information.
One theoretical reason
why we assume that production information
is decoded from sensory information
is that it allows us
to frame trajectory generation and
classification as inverse problems
of each other.
In doing so, there may be ways
in which the classification and generation systems
can correct each other's errors.

A more practical justification
for decoding production information
is that it provides a site in which to
aggregate production information
gathered from multiple sensory modalities.
The McGurk effect discussed
in Section~\ref{sec:sm-neurobio}
indicates that we use both auditory
and visual information to
determine which syllable is being voiced.
If part of our syllable classification system
is informed by a decoding of production information,
then both auditory and visual systems
can contribute to that decoding.
The speech development systems
embodied in DIVA and the Kr\"{o}ger model
also learn to associate sensory information
with one's own production information
through babbling and imitation stages.

Additionally,
it is possible to vocally imitate
novel syllables.
Novel syllables that are made up of sounds
(and therefore vocal tract gestures)
that are frequently encountered
in one's language can be
repeated readily.
Novel syllables that feature
novel vocal tract gestures
are more difficult to repeat readily.
Since both types of novel syllables do not have
a learned production information trajectory
(i.e., are not in the mental syllabary),
it is reasonable to hypothesize that
the syllable's production trajectory
is decoded from the speech information,
and passed through to the
speech motor control system unchanged.
Repeated experience with the novel syllable
would lead to a production trajectory
being learned and consolidated.
Both trajectory decoding and
learning are likely
to use existing elements
in the mental syllabary
that closely resemble the
infrequent syllable to be learned
(e.g., a similar CV syllable could
be used to learn a different CV syllable,
following the frame/content theory
of \citet{macneilage2001}).

\citet{uria2011} has shown that it is possible to
do this decoding,
which some call auditory-articulatory inversion
(see Section~\ref{sec:asr-prod}).
Currently, this has only been done
with continuous articulator positions;
determining if discrete vocal tract gestures
can be decoded from speech
would guide further development of Sermo.
We do not solve this problem
in this thesis;
however, we are able to use Sermo
to produce a corpus of time-locked
synthesized auditory information
and vocal tract gestures,
which can serve as a
benchmark data set.

\subsection{Speech motor control}
\label{sec:model-motorcontrol}

\inoutbox{Production information}{Audio waveform}

The speech motor control system
uses a continuous trajectory
of speech production information
to drive an articulatory
speech synthesis system,
which produces audio waveforms.
It is composed of two components.
The first is the \textit{motor expansion} system,
which projects production information
into the vector space that
directly controls the articulatory synthesizer.
The second is
the \textit{articulatory synthesizer} itself.

\subsubsection{Motor expansion}

To this point, we have discussed
production information abstractly,
either as vocal tract gestures
or vocal tract articulator positions.
In the neurobiological case,
the vocal tract model used
by each individual corresponds to
their own physical vocal tract,
an internal model of which
would be learned through
sensorimotor associations.
In Sermo, however,
there is no perfect choice
for an articulatory synthesizer.
Indeed, it is a long-term goal of Sermo
to provide a unified control system
that can be used to
compare the speech quality
of different articulatory synthesizers
by providing a control system
that can be used for any synthesizer.
In order to reach that goal,
we must choose a canonical
production information representation.
However, since different synthesizers
have different control parameters,
a motor expansion system
is necessary to translate
Sermo's production information representation
to control signals for a specific synthesizer.
It allows the rest of Sermo
to be designed agnostic
to the articulatory synthesizer chosen
because the motor expansion system
will handle the mapping
from production information
to control signals.

One important first step, then,
is to choose a canonical
production information representation,
so that only one mapping is required
for each articulatory synthesizer.
Currently, it is not clear
what production representation
should become that standard.
Previously, we have argued
vocal tract gestures are a
good choice for representing production information.
However, we have not yet determined
if vocal tract gestures can be decoded
from acoustic information
(recall that \citealt{uria2011}
has done this for vocal tract parameters),
nor do we have empirical evidence
that vocal tract gesture trajectories
can be generated and classified robustly.
Additionally, there are unfortunately
at least two systems that employ
a disparate set of vocal tract gestures.
The researchers that first proposed
the task dynamics framework
and the HLSyn \citep{hanson1999}
and CASY synthesizers \citep{iskarous2003}
use vocal tract gestures
which can be expressed
as a vector of eight scalars
\citep{zhuang2008,zhuang2009}.
\citet{kroger2014} and \citet{birkholz2006,birkholz2013},
on the other hand,
use 11 scalar gestures in their 2D synthesizer,
and 46 binary gestures and 3 scalar gestures
in the VocalTractLab synthesizer,
respectively.
If a vocal tract gesture representation
is shown to be advantageous
for the operations required by Sermo,
then standardizing on a single
vocal tract gesture representation
will be a critical next step.
Similarly,
if it is determined that
continuous production information
in the form of articulator positions
is a more advantageous representation
for Sermo,
then a sufficiently expressive
set of control points
will have to be chosen.

While the motivation for the
motor expansion could be interpreted
as meaning that it does not
have an analogous system in the brain,
a brain area playing a similar role
as the motor expansion system
may be useful for
compensating for vocal tract perturbations.
When our vocal tracts are modified in some way,
perhaps by illness or injury,
or by a deliberate manipulation
as in \citet{houde1998},
we can adapt our speech
such that changes in how we voice a phoneme
transfer to other syllables containing that phoneme.
Since it is unrealistic to think that
each syllable trajectory is modified
when adapting to the perturbation,
it is reasonable to assume that
there is an intermediate representation
between the production information
produced by the sensorimotor integration system
and the control signal that is sent
to the motor system.
In Sermo, the representations used
in the motor expansion system
would play that role.

\subsubsection{Articulatory synthesizer}

Articulatory synthesizers
generate audio waveforms
by simulating the airflow
through a model of the human vocal tract.
The general organization of an
articulatory synthesizer
was reviewed in Section~\ref{sec:art-synth}.
As with auditory periphery models,
Sermo uses existing solutions
to articulatory synthesis,
and aims to be a testbed
in which articulatory synthesizers
can be compared to one another
using the same control system.
From the perspective of Sermo,
the details of the vocal tract
and acoustic models
should be abstracted away
and only considered
in the motor expansion system.
Primarily, we are concerned with
how each synthesizer is controlled,
and whether the synthesizer
can be used online.
However, even synthesizers
that cannot be used online
can still be tested with Sermo
by generating the control signals online
and providing them to the synthesizer offline.
These synthesizers would not be suitable
for real time conversation without
being adapted for online use.

\section{Evaluation}

As a conceptual model,
Sermo outlines what we believe a fully integrated
speech system would look like macroscopically.
The model provides context
motivating research of each
of the subproblems that must be solved
for the full model to work in real time.
It is important, therefore,
to consider how to evaluate
subsystems in Sermo,
as the fully integrated model
will take many years to develop.

Progress in the auditory feature extraction system
is difficult to measure
because it is most useful
in the context of other systems.
It is not readily apparent
what impact
an improved decorrelation network
might have on the sensorimotor integration
system's ability to decode production information,
for example.
Although we can assume that
an output representation with
less autocorrelation
will be easier to decode,
the gains from this improvement
may be modest compared to
changing the time constants on temporal transformations,
or even increasing the number of neurons
used to represent certain pieces of information.

Fortunately, the decoding systems
that are the primary use
of the feature extraction system
can be evaluated directly
through phoneme and word error rates,
in the case of linguistic decoding,
and through the mean squared error
in the case of production information decoding.
There are several existing data sets
with full training and testing
input-output pairs,
such as TIMIT for linguistic decoding
\citep{garofolo1993},
and the X-ray microbeam database \citep{westbury1990}
and MNGU0 \citep{steiner2012}
for continuous sensorimotor decoding.
Currently, there are no data sets
for discrete sensorimotor decoding
(i.e., inferring vocal tract gestures
from acoustic information).
We plan to synthesize a
vocal tract gesture
data set to catalyze progress
in this area.

These data sets not only offer a method
to evaluate decoding methods,
but allow us to evaluate choices made
in the feature extraction system.
In general, we see the strength
of large integrated systems like Sermo
as allowing for candidates solutions
to different subproblems
to be evaluated in the context of
known best solutions
to subproblems in connected systems.
In other words,
one of the primary contributions of Sermo
is to propose standardized input and output formats
required by each subsystem.
We will show a proof of concept
that Sermo supports such comparisons
in Section~\ref{sec:results-periphmodel}
in which we compare
five auditory periphery models
on a phoneme classification task.

The strongest evaluation
of the sensorimotor integration
and speech motor control systems
is for listeners to perceptually evaluate
synthesized speech.
If a desired sequence of syllables
is identified by all listeners
as the intended sequence,
then these systems can be considered successful.
However, en route to such an experiment,
the trajectory generation and classification systems
can compare their outputs to known trajectories.
While a version of Sermo
that explains development
would require many presentations
in order to generalize trajectories
associated with syllables,
the current system
is provided a desired trajectory
for each syllable.
Therefore, we can directly
test the sensorimotor integration system's
ability to classify
and generate known trajectories.
Classification is either done
correctly or incorrectly,
though some classification methods
may provide confidence information
that can be used to further evaluate the system.
Generation, on the other hand,
cannot be easily evaluated
as correct or incorrect.
One method of evaluation would be to
measure how much the generated trajectory
deviates from the target trajectory
(comparing, for example,
the squared jerk of the trajectories;
see \citealt{hogan2009}).
However, a stronger evaluation
is to use the generated trajectories
to drive the speech motor control system,
and perceptually evaluate generated audio.
We will present other evaluation methods
for the implemented subsystems
in their respective sections
in Chapter~\ref{chapt:results}.

\section{Relation to other models}

As discussed in Section~\ref{sec:bg-diva-kroger},
the models most similar to Sermo
are the DIVA model by \citeauthor{guenther1995}
and the Kr\"{o}ger model by \citeauthor{kroger2009}.
Unlike Sermo, both of these models
focus on the speech production development,
rather than attempting to model
an adult speech system directly.
Nevertheless, there are significant overlaps
between these models and Sermo.

DIVA (see Figure~\ref{fig:diva}) primarily models
the trajectory generation step
in Sermo's sensorimotor integration system,
but employs feedback signals to learn
how to generate trajectories.
Activation of a cell in the ``speech sound map''
generates feedforward control signals
that drive an articulatory synthesizer.
It also includes auditory somatosensory feedback systems
analogous to Sermo's auditory feature extraction
and non-auditory sensory systems.

The primary strength of DIVA
that is not currently captured in Sermo
is an account of the development
of feedforward control signal trajectories
through repeated trials in which
auditory and somatosensory feedback
corrects parts of the trajectory
in which the synthesized sound differs
from the expected sound.
We believe that Sermo can be adapted
to learn trajectories online in a similar way,
using biologically plausible learning rules
(e.g., \citealt{macneil2011,bekolay2013a}).

There are several critical differences
between how Sermo and DIVA view
the overlapping systems.
First, while Sermo asserts that
the motor system is organized at the level
of syllables in order to ensure
realistic scaling to adult vocabularies,
DIVA allows its speech sound cells to
correspond to phonemes, syllables, words,
or even short phrases.
Additionally, in DIVA
the motor trajectory is initiated
by activating a single speech sound cell,
and it is not clear how sequences
of sound cells would be temporally coordinated.
In Sermo, considerable effort is taken
to create reasonable representations
for the speech target
(which in Sermo are always syllables),
and to temporally coordinate
sequences of syllables.

Second, the form of auditory feedback
differs significantly.
Sermo sees auditory processing as part of
the model, and therefore uses
a biologically inspired auditory filter
and statistical methods that can be
performed by spiking neurons.
DIVA does not constrain
how it processes audio waveforms,
and provides the model with
representations that are well suited
for the operations performed by the model;
specifically, they provide the model with
the first three formant frequencies
at each moment in time,
though they have used log formant ratios
and wavelet-based transformations
with similar results \citep{guenther2006a}.
Regardless, this ideal auditory processing
gives DIVA a significant advantage
in terms of being able to
associate auditory feedback
with control signals,
which is the primary function of the model.

Finally, despite positioning itself as
a biologically plausible neural network model
of speech production
(the ``most thoroughly defined and tested''
as of 2006; \citealp{guenther2006a})
it performs many operations
that are not easily implementable
in a biological system.
Most notably,
while simulated neurons communicate
by sending signals through
synaptic connection weight matrices,
perfect representation and communication
are assumed.
In other words, there is no neural
or synaptic model in DIVA;
neural activations are defined as
differential equations
which are computed directly.
The state variables in these equations
are multiplied by connection weights,
but signals propagate
and with no delay, degradation or noise.
In order to represent temporal effects,
signals can be perfectly delayed
by arbitrary amounts of time
\citep{nieto-castanon2011},
an operation that is difficult
to perform with biologically realistic
neuron and synapse models.
While it may be possible to approximate
the computations that
DIVA assumes are possible
with biologically plausible spiking neural networks,
it is not clear whether
neural approximations will
be sufficient to produce similar behavior
as the idealized DIVA model.

The Kr\"{o}ger model also
focuses on learning and development,
rather than a full adult speech system.
The primary system modeled
is the sensorimotor trajectory generation system,
which is learned using
auditory and somatosensory information.
This model also adds
a linguistic processing system,
allowing the motor system to be driven
by syllable representations.
Unlike DIVA, the Kr\"{o}ger model
assumes speech motor organization
at the level of syllables,
which matches Sermo's motor organization,
and linguistic theory.
The Kr\"{o}ger model also
introduces an intermediate representation
between the motor plan (i.e., trajectory)
and the articulatory synthesizer,
playing a similar role as Sermo's
motor expansion system.
Thus, in terms of gross structure,
the Kr\"{o}ger model covers a larger subset
of Sermo compared to DIVA.

The Kr\"{o}ger model also uses
a slightly more sophisticated neural network model,
the growing self-organized map (GSOM).
GSOMs build on the self-organized map
\citep{kohonen1982,kohonen2007}
by introducing a growing procedure
in which new neurons can be added
to existing maps when the error
in a certain region is above
a certain threshold
\citep{cao2014}.
While spiking versions of
self-organized maps
have been proposed
(e.g., \citealp{choe1998}),
a growing self-organized map
would require either neurogenesis
(which only occurs in a few brain areas)
or recruiting neurons in nearby areas
with novel synaptic connections,
a form of structural plasticity
which is currently not well understood.

In terms of whether other operations
of the model can be realized in spiking neurons,
the outlook is more promising
for the Kr\"{o}ger model.
The auditory representation used
is the power spectrum of the audio signal,
which is computed as a Bark-scaled spectrogram.\footnote{
  The Bark scale is a psychoacoustic
  perceptual pitch scale,
  similar in many respects to the Mel scale
  \citep{zwicker1961}.}
While this is an ideal mathematical transform,
it performs the same function as the
auditory periphery models used in Sermo.
However, while the neural network model
is an improvement over DIVA,
it still uses an overly simplistic
representation scheme.
Words in the phonemic map
are represented with one neuron each;
this localist representation scheme is unrealistic,
and when considering the number of possible
interactions between words,
does not scale to adult vocabularies
\citep{crawford2014}.

\section{Subsystems modeled in this thesis}

In the subsequent chapters of this thesis,
I will present mechanistic models
implemented in spiking neurons
spanning several systems
in the conceptual Sermo model
(see Figure~\ref{fig:sermo-implemented}).

\fig{sermo-implemented}{0.9}{Parts of Sermo implemented in this thesis.}{
  The parts of Sermo implemented in this thesis.
  Colors correspond to neural models.
  Green (left) refers to the auditory feature extraction model
  (i.e., neural cepstral coefficients).
  Purple (right) refers to the syllable sequencing
  and production model.
  Yellow (top-right) refers to the syllable classification model.}

The first model presents an implementation
of the auditory feature extraction system
that performs similar operations as
the frontend in an automatic speech recognition system,
but does so in an online fashion
using an auditory periphery model
and a spiking neural network.
The resulting representations are called
neural cepstral coefficients,
and we will compare their ability
to classify speech with
existing approaches used in ASR.

The second model implements
a syllable sequencing model
as a stand-in for a
full linguistic processing system,
and generates production information trajectories
from those syllables.
The production information will be
used to synthesize speech samples
with the VocalTractLab articulatory synthesizer
\citep{birkholz2006,birkholz2013}.
We will compare the generated trajectories
to target trajectories,
and make qualitative observations
of the synthesized speech.

The third model implements
the trajectory classification portion
of the sensorimotor integration system.
Production information trajectories
will be provided to the trajectory classifier,
which we evaluate by
comparing classified syllables
to the intended target syllables.

\section{Limitations}

Two primary issues limit the Sermo model
as presented in this chapter.
First, the subset of Sermo
that is implemented in this thesis
is more thoroughly defined than
the parts that are not modeled in detail,
such as the linguistic processing system
and non-auditory sensory modalities.
Existing models like the WEAVER++ model
have given us guidelines as to
what computations are necessary,
but how those computations might be implemented
in spiking neural networks
remains unanswered,
and may require different computations
than have been assumed thus far.

It is therefore important to emphasize
that we envision the Sermo model
to be a continually growing and adapting model,
incorporating new modeling ideas
and new empirical results
across disciplines.
The current presentation of Sermo
was developed independently,
primarily informed by the literature
summarized in Chapter~\ref{chapt:bg}.
We hope that upon presentation,
feedback from other researchers will
fill in missing pieces of the model,
and clarify existing pieces,
making the development of Sermo
a collaborative effort.

However, I recognize that
organizing a collaborative effort
like Sermo brings additional challenges.
Therefore, to enable easier collaboration,
we will use modern Internet-based tools
like Github and the Jupyter notebook
to visualize
and disseminating information,
in addition to the traditional
scientific publication system.
These tools provide mechanisms
for proposing and discussing changes
to a shared repository of materials,
which will include text, figures, and code.

A second issue that directly opposes
the idea that this model must be
developed through a large collaborative effort
is that Sermo's insistence on
using mechanistic models that
can be implemented with spiking neural networks
limits the number of researchers
that have the background necessary
to contribute models that implement
Sermo subsystems.
However, we believe that with the growing
interest in neuromorphic hardware
and the development of deep learning approaches
in spiking neural networks,
interest and expertise in spiking neural networks
will steadily increase in the coming years.
In the same vein, it may be argued that
the constraints placed on candidate Sermo models
are too stringent given our current understanding
of the human speech system,
and spiking neural networks in general.
I believe that the models presented
in the rest of this thesis
serve as evidence that
progress can be made with
the tools currently available.
As interest and expertise in
the modeling techniques used
in this thesis grows,
so will the possibility
that all of the components in Sermo
can be implemented efficiently.

\chapter{Previous work}
\label{chapt:previouswork}

In this chapter, we review
existing complete or partial solutions
to the three problems
solved by the models
presented in the subsequent chapters.

\section{Auditory feature extraction}

As summarized in Section~\ref{sec:model-ncc},
the auditory feature extraction system
is based on the feature extraction pipeline
used as the frontend
of automatic speech recognition (ASR) systems.
The pipeline pictured in Figure~\ref{fig:asr}
is similar across most systems.

\subsection{Mel-frequency cepstral coefficients (MFCCs)}
\label{sec:prev-mfcc}

The most common feature extracted
and used in ASR systems
is called Mel-frequency cepstral coefficients (MFCCs).
It has been widely used
in both hidden Markov model-based
(HMM; \citealp{hain1999,gales2008,gaikwad2010})
and deep learning-based
\citep{graves2006,graves2008,fernandez2008} ASR systems,
including those that achieve
the lowest error rates
on the popular corpus TIMIT.
It has also been shown that
MFCCs are well suited
for both speech and music inputs
\citep{logan2000}.

Mel-frequency coefficients are inspired
by the human auditory system,
in the sense that they perform
frequency decomposition
of the audio signal
in order to determine the relative power
at frequencies distributed over the mel scale.
A simple algorithm for computing an
mel-frequency cepstral vector
for a frame of audio is as follows.

\begin{enumerate}
  \item Compute the discrete Fourier transform
    of the audio frame.
  \item Take the log of the power spectrum.
  \item Convolve the power spectrum
    with triangular filters distributed
    according to the Mel scale.
  \item Apply the inverse discrete cosine transform (iDCT)
    to the triangular filter outputs.
\end{enumerate}
The resulting coefficients obtained from
the iDCT are called ``cepstral'' coefficients.\footnote{
  The term ``cepstrum'' comes from
  the word ``spectrum'' with the first four letters reversed,
  as the spectrum is obtained with the Fourier transform
  and the cepstrum is obtained with the inverse Fourier transform.
  Similarly, the domain of the cepstrum is not frequency,
  but ``quefrency.''}

Recall from Section~\ref{sec:psychoacoustics}
that the Mel scale
describes the relationship between
absolute frequency and perceived pitch;
it is a logarithm of the frequency.
The triangular filters are spaced
equidistantly on the Mel scale,
resulting in more filters
at lower frequencies than at higher frequencies.
Typically, at least twenty triangular filters
are used in order to ensure that all frequencies
are captured in more than one filter
(i.e., there is overlap between adjacent filters).
The inverse discrete cosine transform
is designed to decorrelate the signal,
which results in the same amount
of information being represented
with fewer coefficients;
typically, the first thirteen coefficients
are used in the feature vector
of ASR systems.

\subsection{Delta MFCCs}

In addition to the thirteen MFCCs,
many ASR systems,
both HMM-based \citep{hain1999,gales2008}
and deep learning-based
\citep{graves2006,graves2008,fernandez2008},
also append the first,
and sometimes the second,
temporal derivatives of the MFCC
to the feature vector.

The justification for including derivatives
in the feature vector is typically
a practical one,
in that most ASR systems
achieve higher accuracy with
derivative information than without.
Theoretically,
most sources justify time derivatives
by noting that the derivatives
incorporate dynamics into the state representation.
However,
even recurrent deep learning systems
like \citet{graves2008}
use MFCC derivatives
despite the fact that
state information from many previous frames
is available at the current timestep.
It therefore seems likely that
a sufficiently sophisticated machine learning algorithm
would learn that temporal derivatives
are a useful feature;
incorporating it into the feature vector
is not necessary,
but it effectively bootstraps the learning process
by providing it a feature
that would otherwise
have to be learned.

Temporal derivatives are not the only
MFCC transformation that are done in ASR frontends.
Some systems apply an additional transformation
analogous to low-pass filtering
called ``liftering'' that emphasizes
the lower part of the cepstrum.
However, while we are confident
that these other transforms
can be computed with spiking neural networks,
we limit our model to producing
MFCC and delta-MFCC-like features
in spiking neural networks.

\subsection{Spectral analysis with auditory periphery models}

While the analogy
between the frontend of ASR systems
and the human auditory system
is usually just an analogy,
many systems have experimented with
more physiologically accurate
auditory periphery models
to replace the idealized spectral analysis step
in the feature extraction pipeline.

\citet{tchorz1999,dimitriadis2005,schluter2007,shao2009}
have separately proposed variants of MFCCs
that use Gammatone filters
to do the spectral analysis step in an ASR system.
All four of these studies
found that using Gammatone filters
lowered word or phone error rates
on recognition tasks in which
noise was added to speech samples.
Other studies has achieved similar results;
see \citet{stern2012} for a review.

While all of these systems
have been applied successfully
for noisy ASR tasks,
they are not suitable
for inclusion in Sermo in their current form.
None of the methods currently available,
to our knowledge, produce spiking behavior
that could
be easily integrated with the rest of Sermo.
Additionally, these networks compute
functions of the filter output that
may be difficult for neurons to implement.
\citet{tchorz1999}, for example,
implements short-term adaption
through loops that perform
a lowpass filter and a division.
Division is, in general,
difficult to approximate with spiking neural networks.
A full neural implementation
may be possible,
but not trivial.

On the other hand,
silicon cochlea models face the opposite problem.
Silicon cochlea models are hardware implementations
of auditory filter models
designed to interact with other neuromorphic systems,
or to be directly implanted in
patients with hearing loss
in order to partially recover the sense of hearing.
Existing silicon cochlea models include
\citet{chan2007,hamilton2008,wen2009,karuppuswamy2013}
(see \citealt{liu2010} for a review of
silicon cochleas
and other neuromorphic sensory systems).
These systems produce spikes
that emulate the spikes
traveling down the auditory nerve,
and since they are implemented in hardware,
they run much faster than software system.
However, because they produce spikes,
they have not been used
as the frontend to any ASR systems,
to our knowledge,
as it is not straightforward to
construct feature vectors
out of spike trains.

\section{Syllable sequencing and production}

Currently,
we are aware of only two existing models
explaining how sequences of syllables
might be represented
and translated to trajectories of
production information in the brain.
However, there have been many attempts
to solve similar problems
and could be applicable to speech;
specifically,
models of song generation in songbirds
and serial working memory.

??? separate into sequencing and trajectory generation

??? talk about WEAVER/WEAVER++ (roelofs)
in the sequencing section

\subsection{Song generation in songbirds}

The avian song system
exhibits remarkable similarities
to the human speech system
(see \citealt{bolhuis2010}).
As such, models of song sequencing
and generation in birds
may be applicable to models of speech.

\citet{troyer2000} presented a model
of birdsong sensorimotor learning
in which song sequencing is broken
into two subproblems:
syllable learning,
in which the system learns
to associate an ensemble of neurons
with a top-down syllable representation,
and sequence learning,
in which the activation
of an ensemble of neurons
associated with a particular syllable
is linked to the next syllable
in the learned sequence.
The end result of the model
that activating a particular
ensemble of syllable-specific neurons
begins a sequence of activations
representing a particular syllable sequence
associated with a target song.

The model uses a
simple associative learning rule
to learn sensory predictions
of motor representations,
and motor predictions
of sensory representations.
The sequence of syllables is produced
by a motor representation
making a prediction of the
sound that will be produced,
which is associated with
the next motor action to be produced,
which activates a sensory prediction,
and so on.

\fig{troyer}{0.6}{???}{???}

While the learning method
may be useful for future iterations of Sermo,
the rest of the model is not suitable for Sermo
because it does allow for
flexible temporal dynamics
in the motor trajectories within and between syllables,
which is one of the hallmarks of human speech.
In \citet{troyer2000},
all syllables are assumed
to take the same amount of time,
and activate all of the neurons
within the ensemble for the entirety
of the syllable activation
(see Figure~\ref{fig:troyer}).
The time taken for each syllable
is not easily modifiable since
it is defined by the amount of time
it takes for the motor action
to activate the sensory prediction
and switch to the next motor action,
which is an intrinsic property
of the synapse connecting the neuron models.

\citet{drew2003} presented a model
which learns to associate specific neural ensembles
to particular sensory inputs,
similar to \citet{troyer2000},
but also to sequences of sensory inputs.
For example, given syllables A and B,
some neurons would activate
when syllable A is presented,
some when syllable B is presented,
and some only when syllable B is presented
immediately following syllable A.
Since the temporal sequence is
coded in the connections between
these ensembles,
\citeauthor{drew2003} hypothesized that
sequences could be generated,
rather than recognized,
but delivering a generic timing pulse
which emulated hearing all possible sensory
inputs at once.
The first time the pulse is delivered,
ensembles sensitive to a single sound
would activate.
On the next time the pulse is delivered,
ensembles representing sequences of length two
would activate, and so on for longer sequences
(see Figure~\ref{fig:drew}).
Ensembles representing sequences of length one
are not active on the second timing pulse
due to the intrinsic properties
of the neurons in the ensembles;
specifically, after spiking
for the previous timing pulse,
they enter a refractory period
in which they become insensitive to further input.

\fig{drew}{0.5}{???}{???}

While this approach is more temporally flexible
than \citet{troyer2000} because
the timing pulse could arrive at any moment,
it only moves the responsibility for flexible timing
from the neurons involved in the sequence
to whatever mechanism generates the timing pulse.
In the paper, the timing pulse is provided
by the experimenter;
they note that the spacing of syllables
can be controlled by varying the frequency
of timing pulses,
but do not provide any mechanism
for generating the pulses
or for how frequency could be varied.
Additionally, the length of the sequence
is extremely limited,
both in terms of the total length of time
and the number of syllables.
The refractory period of each neuron
is only long enough to be insensitive
to the next timing pulse;
therefore, for sequence of length three,
\citeauthor{drew2003} added inhibitory connections
from sequence-specific ensembles
to non-sequence-specific ensembles.
The connectivity patterns required
for sequences of longer lengths
are not obvious,
and depends on the refractory period,
which is abnormally long.
We therefore do not think that
this model is useful for Sermo.

While other models for songbird generation exist
(e.g., \citealt{fee2004}),
none exhibit the temporal flexibility
that is required for speech.
As shown in \citet{fee2010},
cooling a part of the avian brain
that projects to motor cortex
results in slowing the trajectory
in proportion to the temperature.
Therefore, birds may not possess
the ability to flexibly time their songs
in the same manner that humans time speech,
meaning we must look elsewhere for
more temporally flexible models.

\subsection{Serial working memory}

We assume that a sequence of syllables
is likely to be represented
in a similar manner
as sequences of other static representations.
Therefore, a useful paradigm for studying
how humans represent sequences
is to investigate serial working memory tasks;
e.g., a subject is asked to remember
a list of numbers,
and then later recall that list,
or elements from that list.
Several models have been proposed
to solve these tasks
in biologically plausible ways.
The most sophisticated such model
is the ordinal serial encoding model
\citep{choo2010},,
which has been incorporated into
Spaun \citep{eliasmith2012}.

The model takes inspiration from
earlier serial working memory models,
namely CADAM, TODAM, and TODAM2
(see \citealt{choo2010} for more details),
which are all based on
vector symbolic architectures
(see Section~\ref{sec:methods-sp}
for details on vector symbolic architectures).
Unlike these models,
\citeauthor{choo2010}'s model
is able to remember and recall
lists of up to seven items,
and exhibits primacy and recency effects,
meaning that items at the beginning
and end of a list are more likely
to be recalled.
His model is also implemented
in a spiking neural network,
making it applicable to Sermo.
Interestingly,
the OSE model does not exhibit
the primacy and recency effects
that are seen in humans
unless the model
is implemented in neurons;
direct simulation of the differential equations
behind the model show
poor performance for the second item
even though it should be easily remembered
due to primacy.

The ordinal serial encoding model,
then, is well suited for representing
discrete sequences of syllables in Sermo,
which typically have few elements
if we assume that most syllable sequences
correspond to words in one's lexicon.
However, the model is limited to
discrete sequences,
and therefore cannot be used
for detailed trajectory generation.
Additionally,
while the model is temporally flexible,
in that it can be queried for
a list element at any time,
there is a slight lag between
when the next element is queried
and when it has been recalled.
Syllables, however, are voiced in quick succession,
sometimes even blending into one another slightly.
In Sermo, we will use a separate but interacting model
for trajectory generation,
and will present a solution
to the issue of pauses between items
in Section~\ref{sec:impl-prod-neuralmodel}.

\subsection{Task dynamics}

??? be more explicit that this is about
trajectory generation and not syllable
sequencing
% from Bernd:
%% the primary idea of task dynamics is not syllable sequencing but
%% calculation of movement contribution of articulators to a gesture
%% (e.g. contribution of jaw and tongue to an apical closing action)

A line of research encompassing work by
several investigators at Haskins Laboratories
developed a set of techniques
under the name Task Dynamics
for generating temporal sequences
of production information
and using that information to drive
an articulatory synthesizer \citep{nam2004}.

The key insight in Task Dynamics
is to dissociate the task state
from the motor trajectory
along that state.
By doing this, the dynamics of the task state
can be considered separately from motor trajectories,
while trajectories can be implemented
as a function of the task state.
In the initial publication of task dynamics
\citep{saltzman1987},
two types of task dynamics are presented.
Point attractor dynamics,
modeled by critically damped mass-spring systems,
are useful for one-time actions.
Cyclic attractor dynamics,
modeled by harmonic oscillators
with a nonlinear escapement function,
are useful for repetitive actions.
Initially, these task dynamics were then
related to articulator trajectories
for a two degree-of-freedom arm model
by mapping the task space
to body-centered coordinates,
joint coordinates,
and finally articulator positions,
called the task network.
In a subsequent publication,
\citet{saltzman1989}
applied task dynamics to speech production
using a simplified mapping,
from task space
to gesture space,
and then to articular space,
similar to that described in Sermo
(see Section~\ref{sec:model-sm}).
Realistic articulatory trajectories
were achieved
by associating a task dynamic network
with point attractor dynamics
with each speech gesture.
Each gesture defines
the point which the task dynamic network
is attracted to.
The state of each gesture network
influences one or more articulator positions.
The final articulator trajectory
is the combined effect
of all of the gesture networks
on all of the articulators.

Given a complete gesture trajectory
for an utterance,
the task dynamic approach
generate articular position trajectories
which can be synthesized
by the HLSyn synthesizer \citep{hanson1999}.
While this model has not been
modified since its introduction in 1989,
the same group has done significant work
in automatically generating gesture scores
with a similar approach.
In this extension,
instead of knowing \textit{a priori}
when each gesture should occur,
the point attractors associated
with each gesture are coupled
to one another,
such that the timing of each gesture
is controlled by the state
of the gestures to which it is coupled
\citep{saltzman2000}.
Task dynamic gestural timing
is able to capture precise timing
in syllables with complex onsets
and codas \citep{nam2003}.
Perhaps more importantly,
it allows for the articulatory synthesizer
HLSyn to accept text as input,
which is converted to a gesture score
by looking up a syllabification
of the word in a database
and using gestural coupling rules
defined by linguists
to create a system of coupled oscillators
whose activities represent a gestural score
\citep{goldstein2009,nam2004}.

In all, the task dynamic approach
to inter-gestural and inter-articulator timing
is the most temporally flexible
syllable production system
currently published
(to our knowledge).
The dynamical systems approach
also makes it a natural fit for Sermo,
as it is defined in continuous time,
and can be readily implemented
in spiking neural networks,
though we are not aware of any
neural implementations currently available.

\subsection{REACH}

The final model informing
the Sermo syllable sequencing
and production model presented
in this thesis is the
Recurrent Error-driven Adaptive Control Hierarchy (REACH)
model \citep{dewolf2015}.
In particular, we use REACH's
neural implementation
of dynamic movement primitives (DMPs).

The REACH model is a general motor control
model implemented in spiking neurons.
It uses DMPs  to generate trajectories
for the system to follow;
these trajectories are mapped into
motor space using operational space control,
and unexpected changes in system dynamics
are accounted for online
using nonlinear adaptive control
\citep{slotine1987}.
The model is able to control
a nonlinear three-link arm model
in handwriting and reaching tasks,
even when an unknown force field
is applied to the end effector.

DMPs are a general method for generating motor trajectories;
they will be discussed in more detail
in Section~\ref{sec:methods-dmp}.
DMPs are similar to Task Dynamics in many respects.
Both define methods to generate trajectories
for one-time and rhythmic actions.
Both use point attractor dynamics
for one-time actions
and cyclic attractor dynamics
for rhythmic actions.
Both dissociate the temporal dynamics
of the task from trajectories
in motor space,
allowing them to advance the system state
at variable rates,
and compute the trajectory
as a function of the system state.

The primary difference between DMPs and Task Dynamics
is that Task Dynamics generates
the trajectory as a function
of the system state,
while DMPs generates the trajectory
as the system state
plus a separate function
of another dynamical system.
Dissociating the system state
from the nonlinear task-specific function
makes DMPs more flexible and general.
Another important difference for Sermo
is that DMPs have been implemented
in spiking neural networks successfully
\citep{dewolf2015}.
We will explain DMPs in more detail
in Section~\ref{sec:methods-dmp} and present a model
using rhythmic DMPs for syllable production
in Section~\ref{sec:impl-prod-neuralmodel}.

\section{Syllable recognition}

Syllable recognition, in general,
is a task that can be solved
by most ASR systems
using a labeled acoustic data set.
In the sensorimotor integration system
in Sermo, however,
we aim to classify syllables
based only on production information
which is decoded from acoustic information.
It should be noted that we do not expect
this system to perform perfectly,
as the dominant speech decoding system
will be linguistic;
however, we are nevertheless able
to hear infrequent syllables
and voice them.
The ability to differentiate
between frequent and infrequent syllables
may depend on whether they
can be classified
on the basis of production information.

One of the few attempts to classify speech
solely on the basis production information
is \citet{mitra2014}.
\citeauthor{mitra2014} were able to
decode continuous production information
from auditory information
using MFCCs and Gammatone filter-based
cepstral features \citep{mitra2012},
and were able to use a combined feature vector
consisting of MFCCs and production information
to lower word error rates
in various noisy environments.
However, word error rates
when using only continuous production information
as the feature vector
were high ($\sim$70\% with no noise).

The apparent difficulty
in classifying sounds based on
production information alone
could be due to several factors.
For one, the details of the decoding mechanism
in \citet{mitra2014} are not clear.
The authors note that they use a
``deep neural network''
with as many as six hidden layers;
however, the choice of neural activation function,
optimization procedure,
and many other hyperparameters
can affect how well the network learns.
In particular, it does not seem as though
the network has recurrent connections,
which are commonly used in
current state-of-the-art ASR systems.

Alternatively, the statistical approach
used in most ASR systems
may not be well suited to
trajectories of production information.
In order to broaden our search
for other types of techniques,
we investigated general solutions
for trajectory classification
which are used in applications
such as gesture recognition
and automatic video analysis.

\subsection{Trajectory classification}
\label{sec:prev-classification}

% ??? list some desirable qualities of a traj classifier
% from later on; e.g.,
% speed invariance (works regardless of how fast
% the traj is playing out),
% online classification, ...

Unsurprisingly, many of the existing
solutions for trajectory classification
are based on generative statistical models,
in particular Hidden Markov Models
as has been dominant in speech recognition
(see \citealt{mlich2008,nascimento2010}
and \citealt{mitra2007,weinland2011,rautaray2015}
for reviews).
Given the failure of the statistical approach
in \citet{mitra2014},
we instead focused on systems
that recognize trajectories
through tracking the trajectory
either in comparison to some known template,
or as the state in a dynamical system.
We identify three such systems
that provide inspiration
for the trajectory classification model
described in Section~\ref{sec:impl-recog}.

\citet{kiliboz2015} proposed
a gesture recognition system
for 2D trajectories
using a finite state machine approach.
Finite state machines are composed of
discrete states and a set of state-specific
functions that transition between states.
During a learning phase,
a target trajectory is played
several times,
and one or more finite state machines
are constructed such that
the target trajectory
causes the state machine
to transition to a final accepting state.
During recognition,
the continuous input trajectory
is presented to all finite state machines
in the system;
the first to reach the accepting state is recognized.
The system attains a 73\% accuracy rate
in a real-world user study.
As the system operates continuously online,
part of the system is applicable to Sermo;
however, the discrete state space and
gesture sequence representation
cannot be easily adapted to speech.
Each timestep in the gesture trajectory
is represented by a string denoting whether the
end effector has moved significantly
in the $x$ or $y$ direction since the last frame;
the overall trajectory is represented
by a regular expression generalized from
the strings seen in the learning phase.
It is not clear that this representation scheme
would scale to $n$-dimensional spaces,
as is required for production information trajectories.

\citet{quiroga2013}
proposed a system called competitive neural classifiers (CNC)
that can recognize arabic number hand gestures
using small training sets (three examples per gesture).
CNC uses a collection of sub-classifiers
that compete in order to collectively
classify the overall trajectory.
The trajectory is segmented
into a set of subtrajectories,
each of which is evaluated by
a sub-classifier neural network
whose output neural activations
represent the probability that
the subtrajectory is produced by
the gesture associated with that output neuron.
The overall classification aggregates the results
across of all sub-classifiers,
producing the correct classification
in 98\% of test cases.
However, the system's impressive results
are partly due to an elaborate preprocessing step
in which a recorded trajectory
is normalized and resampled
such that the actual trajectory used as input
has a fixed number of sample points
uniformly distributed over
the total length of the trajectory
(see Figure~\ref{fig:quiroga}).
This type of preprocessing
requires knowledge of the whole trajectory,
and therefore could not be implemented
in an online manner.
Additionally, the operations done
on the neural network outputs
are difficult to implement
with spiking neural networks
(e.g., division, argmax),
making this system unsuitable
for use in Sermo.

\fig{quiroga}{0.5}{???}{???}

Finally, \citet{caramiaux2014}
presented an algorithm called the
Gesture Variation Follower (GFV),
based on an online HMM-based
trajectory tracking technique
called Gesture Following (GF)
\citep{bevilacqua2010,bevilacqua2011}.
The goal of the algorithm is to match
an input gesture sequence
with a set of predefined template gestures.
Unlike the HMM-based GF algorithm,
GFV views the trajectory as a dynamical system,
and uses a Particle Filtering algorithm
to learn a set of weights
that denote the importance of
each randomly generated particle
(i.e., dynamical system state)
to overall recognition accuracy.

The overall flow of the algorithm
is as follows.
First, a set of predefined example trajectories
are defined in the system.
Second, a set of particles
are randomly generated
in an $n$-dimensional space
with a uniform distribution,
and a set of weights are initialized
with each particle weighted equally.
In the main loop of the algorithm,
a random sample is drawn
according to each particle's position
in state space,
and the weights associated with each particle
are updated based on the distance
between the particle's sample
and the observed input.
If too many particles have small weights
associated with them,
then a resampling procedure is done
based on the current weights.

What differentiates this algorithm from
other particle filtering algorithms
is the structure of the state space.
Briefly, the state space that each particle is in
encodes the target gesture that this particle
is associated with,
and a probability distribution
over the phase of the trajectory
(i.e., how far along the trajectory
we currently are),
and the speed of the trajectory
(i.e., how far along the trajectory
do we expect to be on the next timestep).
The state evolves over time according to
predefined state transition functions,
and on each timestep,
each particle emits an observation
according to a possibly non-linear function.

A useful analogy for this algorithm
is that it implements an online version of an HMM.
Like HMMs, it maintains some internal representation
that can be used to estimate the probability
of a particular sequence
by forming a prediction
of the next observation.
The system is queried by providing
an observation,
which changes the internal representation
such that the next observation
is processed in the context of
the sequence of past observations.

GVF has been used successfully
in music generation systems
in which the music sample,
volume, and speed are
controlled by gestures
that are tracked by the GVF algorithm.
It achieves over 98\% accuracy
in a 2D gesture recognition task
with 16 possible gestures,
and was employed successfully
in a 3D hand gesture user study
with 10 participants.

While GVF is one of the most promising
algorithms for doing trajectory recognition
in Sermo,
it has two main limitations in its current formulation.
First, while the classifier can be used online,
it is formulated in discrete time;
this weakness should be possible to overcome,
however, as continuous time particle filtering
has been done in the past
\citep{ng2005}.
Second, while the dynamical system
at the core of GVF is well suited
to be implemented in spiking neurons,
the resampling process is not.
On each step of the algorithm,
each particle represents a probability
distribution over the system state,
which is sampled
(sampling has been shown to be
possible with spiking neurons;
see \citealt{buesing2011}).
However, when the importance weights
of a sufficient number of particles
is below a particular threshold,
a new set of particles
is randomly generated
to replace those with low importance weights.
This procedure would translate to
a significant reorganization
of the neurons implementing
the sampling procedure,
which we believe
would not have evolved if
procedures that do not require
reorganization exist.

We will propose a trajectory classification technique
in Section~\ref{sec:impl-recog}
that shares many of the positive
characteristics of the GVF algorithm,
but can be implemented in a spiking neural network.
Like GVF, it is also based on the idea
of inferring internal dynamical system state
based on observations;
however, it performs the inference
in a manner that we relate to
DMPs (see Section~\ref{sec:impl-recog-overview}).

\chapter{Design and methodology}

%% ~15-30 pages

%% - continuing from Chapter 2 explain the issues
%% - outline your solution / extension / refutation

Natural speech is not a series of phonemes
recited as clearly as possible
to maximize intelligibility.
Natural speech includes variation in
pitch and volume in order to transfer
non-linguistic information between speakers.
Current automated speech systems
focus solely on the linguistic content
of speech;
by deciphering and reproducing both
linguistic and non-linguistic content,
we aim to produce more natural speech
recognition and production
than the existing state of the art.

\section{Recognition system}

The recognition system is composed of
three layers connected in a feed-forward manner.

???come up with a name maybe

???figure of whole system

\begin{itemize}
\item \textbf{Auditory periphery.} The auditory periphery layer
  takes incoming air pressure waves and converts them
  to a frequency representation,
  mimicking the function of the human ear.
\item \textbf{Auditory preprocessing.} The auditory preprocessing layer
  represents the frequency information from the auditory periphery,
  and through recurrent connections within the layer,
  also represents temporal dynamics of frequency information
  (e.g., time derivatives).
\item \textbf{Features.} The feature layer
  represents speech-relevant features;
  in this work, we represent phonemes, pitch, and volume.
  Features are computed from information
  in the auditory preprocessing layer.
\end{itemize}

This system is similar to many previous systems,
and aside from its feedforward nature,
follows naturally from the organization
of the human speech system (???ref or ???section).
The primary way in which this system
deviates from previous models
is by imposing biological constraints.
Specifically, the system operates
in continuous time,
and only uses locally available information.
Adhering to these constraints differentiates
this system from similar systems,
such as those based on deep learning (???refsection),
in which... ???more, it's discrete
This system is also ???something about NEF

\subsection{Auditory periphery}

???figure of periphery

\subsection{Auditory preprocessing}

\subsection{Features}

NEF stuff

Possible future in which this is learned

\subsection{The role of feedback}

The above system is primarily feed-forward
(though feedback is used extensively in the
auditory preprocessing layer).
It is important to note that
a complete auditory processing system
would match the brain (???refs) and have
feedback connections between all layers
in order to refine each layer's ability
to provide useful output downstream layers.
(??? more stuff from refs)

Feedback is an undeniably important
component of a full system that operates
for long periods of time with sensors
that are constantly changing and degrading.
In this work, however, we assume
that the auditory periphery does not change
its ability to process sounds over a long timescale,
and therefore we do not include corrective feedback
signals between layers in the auditory system.
This type of feedback could be added
to this system in future work.

\subsection{Evaluation}

\section{Synthesis system}

\subsection{Evaluation}

\section{Integrated speech system}

\subsection{Evaluation}

\chapter{Implementation}

In this thesis, we implement and evaluate
four neural models that make up
parts of the conceptual Sermo model.
Each model provides
an explanation for how the brain might
solve a task necessary for
speech recognition or synthesis
using the computational devices at hand,
namely spiking neurons.
However, where possible,
we also present analogous non-neural models
that could be incorporated in
non-neural speech recognition and synthesis systems.

\section{Neural cepstral coefficients}

The frontend of most automatic speech recognition systems
extracts a feature vector from the incoming audio waveform.
As reviewed in section ???,
mel-frequency cepstral coefficients (MFCCs)
have been used successfully in many ASR systems ???cite.
To a certain extent, MFCCs are motivated
by the organization of the human auditory system.
We propose neural cepstral coefficients (NCCs)
as an alternative feature vector representation
for automatic speech recognition
and other tasks involving speech processing.
Unlike MFCCs, NCCs are both motivated
by the organization of the human auditory system
and are implemented the same basic computational units
as the human auditory system,
spiking neural networks.

??? NCC pipeline, like AuditoryBasedFeatureVectors.pdf
or KumarKimSternICA11.pdf

Figure~??? shows the NCC processing pipeline,
as compared to the MFCC processing pipeline.
While the pipeline implements
essentially the same computations,
the manner in which those computations
are performed differs significantly.
The primary difference is that
incoming audio is processed
in a continuous online fashion,
rather than in a frame-based fashion.
Typical MFCC extraction algorithms
break the audio signal into
overlapping fixed-length windows
called frames,
which are processed independently
until the smoothing stage,
where discontinuities
due to frame boundaries are
minimized.
The NCC extraction algorithm,
on the other hand,
maintains internal state
at each processing stage,
and updates that internal state
for each incoming audio sample.
As each internal state update
depends on the current state,
changes occur smoothly through time,
which obviates the need for
an explicit smoothing step.

Another implementation difference
is in the computation of
the inverse discrete cosine transform.
Typically, this computation is done as
\begin{equation}
  y_k = \frac{x_0}{\sqrt{N}} + \sqrt{\frac{2}{N}} \sum_{n=1}^{N-1}
  x_n \cos \left( \frac{\pi}{N} n \left( k + \frac{1}{2} \right) \right)
  \text{ for } 0 \le k < N
\end{equation}
where $y_k$ is the $k$th cepstral coefficient,
$x_n$ is the $n$th auditory filter output,
and $N$ is the number of auditory filter outputs.
In matrix notation,
this equation can be expressed as
\begin{align}
  \label{idct}
  \mathbf{k} &= \left[ 0, 1, \ldots, N-1 \right] & 1 \times N \text{ vector} \nonumber \\
  \mathbf{s} &= \left[ \sqrt{2}, 1, 1, \ldots, 1 \right] & 1 \times N \text{ vector} \nonumber \\
  \mathbf{T} &= \sqrt{2}{N} \, \mathbf{s} \circ \cos \left( \frac{\pi}{N} \left(
    \mathbf{k} + \frac{1}{2} \right) \otimes \mathbf{k} \right)
    & N \times N \text{ matrix} \nonumber \\
  \mathbf{y} &= \mathbf{T}\mathbf{x} & N \times 1 \text{ vector}
\end{align}
where $\circ$ is the Hadamard (element-wise) product,
and $\otimes$ is the outer product.
This rearranged equation
allows us to precompute a matrix, $T$,
that maps auditory filter outputs
directly to cepstral coefficients,
which enables us to embed this matrix
in the connection weights between
two neural population
(see ??? network diagram for more details).
By restricting $\mathbf{T}$
to the first $M$ rows,
equation \eqref{idct} yields the first
$M$ cepstral coefficients;
typically, ASR systems
use $N \approx 26$ auditory filter outputs
and $M \approx 13$ cepstral coefficients.

\subsection{NCC neural model}

The NCC pipeline is implemented
in a Nengo network that uses
the Brian hears library
of auditory filter models
??? cite Brian hears,
and two layers
leaky integrate-and-fire (LIF) neurons ensembles
connected in a feed-forward manner
(see ??? figure).
While Brian has the capability
to simulate spiking neurons,
we only use Brian's implementation of
auditory periphery models,
which do not emit spikes.
These models,
summarized in ??? sect prev work,
emulate the effect of the
audio waveform on the basilar membrane,
and the resulting inner hair cell activity
arising from basilar membrane deflections,
which may include rectification
and compression
(i.e., it accomplishes the first
??? steps of the MFCC pipeline).
The output of these models
can be though of as
the current input
to spiral ganglion cells in the ear.
These auditory periphery models
accomplish a significant portion
of the MFCC extraction pipeline,
which is unsurprising as MFCC extraction
is designed to emulate certain aspects
of the human auditory system.

??? figure: net diagram

The simulation timesteps
for the Nengo model and the contained
Brain auditory filter model
are uncoupled.
The auditory filter model
is always run with the same timestep
as the auditory stimulus;
that is, its internal state is updated
for each audio sample.
The Nengo model, on the other hand,
runs at a timestep independent
of the auditory stimulus.
On each Nengo timestep,
therefore, one or more samples
of the audio input
is fed through the
auditory filter model.
The \textit{auditory nerve} ensembles
receives as input the current
state of the auditory filter model,
regardless of how many timesteps
the Brian model has advanced.

The \textit{auditory nerve} ensembles
mimic the activity of spiral ganglion cells
projecting down the auditory nerve.
For each characteristic frequency,
a heterogeneous population of neurons
is driven by the current output by
the auditory filter model.
Each characteristic frequency
drives an independent neural ensemble,
meaning that this layer of ensembles
can be thought of as being cochleotopically organized,
as the mean activity of an individual neuron
only gives information about the output
of a single auditory filter
(although that filter may be modulated
by power in nearby frequencies).
Together, the decoded output of
each ensemble gives an approximation of
the compressed power in the part of the spectrum
that the auditory filter is sensitive to.
This quantity is essentially
the output of the auditory filter
(i.e., steps ???--??? in the MFCC pipeline),
but now represented in
cochleotopically organized spiking neurons.
It is also the $X$ vector in
the inverse discrete cosine transform,
equation~\eqref{idct}.

The \textit{cepstra} ensembles
each represent one cepstral coefficient.
The inverse discrete cosine transform
for that coefficient is computed
through the connections between
the auditory nerve ensembles
and the cepstral ensembles.
Specifically, each connection
between an auditory periphery
and cepstrum ensemble
scales the decoded value of the
cepstrum ensemble by
the corresponding value
the $\mathbf{T}$ matrix.
Since this is a linear transform,
it is done by computing
normal decoding weights
according to equation~\eqref{???}
and scaling the decoders by $T_{i,j}$.
Since the currents coming from
multiple connections to an ensemble
are summed, the cepstral ensembles
end up representing the
dot product between the row of
$T$ that corresponds to its cepstrum
and the $X$ vector,
which is collectively represented
by the auditory periphery layer.

The end result of the networks
is the decoded output
of the \textit{cepstra} ensembles,
which is a continuously varying,
automatically smoothed set of cepstral coefficients.
We hypothesize that these coefficients can be used
as the feature vectors in an
automatic speech recognition system.

??? make an appendix with a simplified Nengo model
and a table of parameters and typical values

\subsection{Delta neural cepstral coefficients}

Another common technique in designing frontends
for automatic speech recognition is to
include temporal derivatives
in the feature vector.
Appending MFCC derivatives
to the feature vector
(i.e., MFCC velocities;
called Delta-Cepstral Coefficients [DCC])
improve ASR accuracy.
Appending derivatives of the MFCC derivatives
(i.e., MFCC accelerations;
called Double Delta-Cepstral Coefficients [DDCC])
further improves recognition,
though by a smaller margin than
the improvement from the first derivative
??? find good cite.
??? Kumar et al kumarkimsternica11.pdf
used the derivative of the auditory filter output
instead of the raw auditory filter output
in a MFCC-like pipeline
(called Delta-Spectral Cepstral Coefficients [DSCC])
and found that the ASR system
was more robust to noise and reverberation.

Temporal derivatives in these systems
are typically computed through straightforward
finite differences.
That is, for an MFCC $y_k$
at frame $t$, the corresponding delta is
\begin{equation}
  \delta_k(t) = y_k(t+n) - y_k(t-n),
\end{equation}
where $n$ is a small number of frames
(typically between one and three).
The double delta applies the same operation
to the $\delta_k$ vectors.

An additional nonlinearity is applied
for DSCCs,
as they use the derivative of the power spectrum.
Using the raw power spectrum derivatives
results in very small cepstral coefficients,
so ??? Kim et al proposed a ``Gaussianization''
step in which the power spectrum
is made to follow a Gaussian distribution
by inverting the normal cumulative distribution function.
??? more?

\subsection{NDCC and NSCC neural models}

??? talk about how to do derivatives
in neurons

?? discuss whether we can do gaussianization?
we probably can... but maybe not in a bioplaus manner.

??? make an appendix with a simplified Nengo model
and a table of parameters and typical values

\subsection{Evaluation}

??? it's not clear if the derivatives will help
since the SVM should manage these,
so if they don't help then I will
just use the derivatives

\section{Syllable recognition}

Go from a vocal tract gesture score
to a syllable semantic pointer, in neurons.

\subsection{Network diagram}

\subsection{Evaluation}

\section{Syllable production}

Go from a syllable semantic pointer
to audio signal, in neurons.

??? Do some fine tuning?

\subsection{Network diagram}

\subsection{Evaluation}

\section{Syllable consolidation}

Go from vocal tract gesture score
to audio signal, in neurons.

Learn a syllable recognition
and a syllable production
network from this.

\subsection{Network diagram}

\subsection{Evaluation}

\chapter{Results and Evaluation}

%% ~15-30 pages

%% - adequacy, efficiency, productiveness, effectiveness
%%   (choose your criteria, state them clearly and justify them)
%% - be careful that you are using a fair measure, and that you are
%%   actually measuring what you claim to be measuring
%% - if comparing with previous techniques those techniques
%%   must be described in Chapter 2
%% - be honest in evaluation
%% - admit weaknesses

List of experiments (fill this in as you write other sections!!)

These should maybe be included in the methods?

\section{Recognition system}

\subsection{Metrics}

Phoneme decode error rate

Pitch error rate

Volume error rate

\subsection{Experiments}

1. Do frequency interacting transformation matter?

- generate networks with N random transformations
  in preprocessing layer and collect the metrics

- systematically increase the number of interacting
  transformations and see if it matters

2. Vowels: should we use temporal information or just
   the current moment's info?

3. Vowels: do diphthongs count?

- try having no categories for diphthongs and see if
  we can get the two components of it

- try having explicit diphthong categories and see
  if we get better at it

4. Separate vowel / consonant populations, or just one

- See if error rates differ

5. Consonants vs vowels:

- Vary the number of features and/or neurons and see
  how much is needed for equivalent error rates

  - Should error rates be normalized by the total number
    of possible outcomes?

6. Synthesized vs natural speech:

- How much easier is synthesized speech compared to natural?

7. Relative pitch experiment:

Find something from the literature to evaluate relative pitch...

8. Relative volume expt:

Find something from the literature to evaluate relative volume...

9. The usefulness of noise

- P. 65 of Kollier et al has a bunch of noise added in.
  Try injecting noise, see if it helps.

10. Preprocessing choices

- Pool or don't pool
- Use nonlinear derivative glides, or just pure derivatives

11. Phonemes or gestures

- Try decoding gestures instead of phonemes
- What's better / easier?

\section{Synthesis system}

\subsection{Metrics}

Speech intelligibility

RMSE between recognized and decoded speech?
Is that helpful?

Can use the recognition system to evaluate
this synthesis system relative to other
synthesis systems.
In a sense, this is one of the benefits
of this type of model.

Neural resources used

\subsection{Experiments}

1. Biologically plausible fluctuations.

- Ideal control methods have no variability, they hit things at the same time
- Record when consonantal closures / releases happen, show that there's
  a certain amount of variance
- Hopefully can show that this is similar to biology?

2. Oscillator / trajectory stability

- Show how accurate / fast the coupled oscillators can be
- Contrast to trajectory generation with Aaron's stuff
  (if that's possible)
  - Also contrast to trajectory gen with DMPs?

3. Scalability

- Take the control system and look at how much cortex (neurons, synapses)
  is taken up by each element (word, syllable, phoneme, etc).

- Extrapolate to human sized vocabularies, make sure it'll scale

- Show that if we had a separate oscillator / population
  for each word or syllable that this wouldn't scale

\section{Integrated system}

\subsection{Metrics}

\subsection{Experiments}

1. Shadowing proof of concept

- Show it works

??? more

4. Syllable learning proof of concept

- show it works

5. The effect of speed on syllable learning

- Slow it down. Should be better

\chapter{Discussion}

\section{Summary and implications of results}

\subsection{Neural cepstral coefficients}

The most important and somewhat surprising result
is that NCCs can yield significantly
better training and test accuracy rates than MFCCs,
which are the most commonly used
feature vector in ASR systems.
Our goal in these experiments was to
show that NCCs are at least as good as MFCCs,
as the primary motivation behind
NCCs is to generate a feature vector
entirely in spiking neurons,
therefore making it suitable
for use in Sermo,
and more broadly being applicable
in neuromorphic systems using
a silicon cochlea.
The fact that they also yielded
better accuracy rates
is promising,
though it should be kept in mind
that these results
do classification through
an ideal statistical linear SVM classifier,
rather than being embedded
in a full continuous ASR system.
It remains to be seen
whether NCCs will be successful
in that more realistic setting.

One advantage of NCCs is that
no explicit normalization is required,
as shown in Section~???,
where it was determined
that NCCs are best used
without z-scoring, unlike MFCCs.
It is important to note, however,
that NCCs do still employ a method
of normalization;
spiking neurons have a refractory period,
meaning that they have a strict maximum firing rate,
which puts an effective limit on the
decoded output of an ensembles.
Unlike MFCCs, however,
this normalization is applied
for every step in the pipeline,
rather than just once at the end.
It is possible that
the normalization inherent in
spiking neural networks
is responsible for the better performance
compared to MFCCs.

While a similar argument could be made
for why adding derivatives had
a deleterious effect on
NCC test accuracy,
it is also possible that
using offline SVM classification
is responsible for decreased performance.
Having many more features
may make the supervised learning procedure
more difficult,
or perhaps since we have the full history,
the derivative is redundant
and only makes it more difficult
to linearly separate data.

We view the final NCC experiment,
in which the auditory periphery model was varied,
as being the main contribution
of this model.
Because the model bridges the gap
between detailed auditory periphery models
and automatic speech recognition systems,
we are able to do apples-to-apples comparisons
of how well the periphery models
process realistic speech samples.
In general, the more realistic the model,
the better it is for speech,
with the notable exception of the Gammatone filter,
which is the least realistic,
but the cheapest to compute,
and as accurate as
the most complicated Tan Carney model.

There are two reasons why we might
be skeptical of the impressive Gammatone results.
For one, the determination of good parameter sets
in the other experiments in Section~???
all used the Gammatone filter.
It is possible that performing the same
experiments with other filters
would find parameters sets that
allow those auditory periphery models
to perform as well or better than
the Gammatone filter.
Additionally, arguably the most difficult part
of continuous ASR systems is
determining where phone boundaries lie,
rather than classifying pre-segmented phones.
While we believe that the experiments
performed in Section~???
are a useful comparison between MFCCs and NCCs,
some of the benefits of more sophisticated models
like the Tan Carney model
may lie in their ability to segment speech.
The same argument holds
for adaptive neuron types;
while they may not be essential
for differentiating between pre-segmented speech samples,
they may be essential for
segmenting continuous speech.
Looking at the impact of these
NCC representation choices
in a continuous ASR system
is therefore an important next step.

\subsection{Syllable sequencing and production}

% ??? note that we are doing a lot of stuff
% here with inhibition.
% While it is possible to do this stuff
% without neurons in a similar manner,
% the ability to directly inhibit ensembles
% improves the robustness of the network,
% which we take as a concrete benefit
% of neurally implementing the algorithm

% ??? temporal output associative memories
% provide a robust link between the discrete,
% symbolic world of high-level representations
% that can be flexibly combined (see, e.g., Dan's RPM work)
% and the continuous, subsymbolic world
% of low-level representations that must
% be precisely timed (see e.g., Travis's work).

% ??? temporal input associative memories
% provide an analogous link between
% continuously varying low-level sensory inputs
% and high-level symbolic representations

% ??? should we do all of this with direct production info
% and not gestures? It's worth looking into...

% ??? Do some fine tuning?

% \subsection{Vocal tract parameters}

% ??? Instead of gestures, we could instead
% generate parameters. Pro: continuous, lower dimension.
% Con: can't optimize the mapping between
% gesture and parameters.
% However, worth trying out?

% Or, obvious next step!


\subsection{Extensive use of DMPs}

% DMPs: they are useful.
% For different things though, depending on the task.
% Here are some guidelines for
% what DMPs are useful for what, and when.

% Ramping DMPs: require constant input; discrete.
% Good for inverse case.

% Deadzone DMPs: one-time input; discrete.
% Good to effectively call a temporal function.

% Rhythmic DMPs: one-time input; rhythmic.
% To do discrete, must explicitly turn off.
% Kind of essential for possibly repetitive action.

% Perhaps there's some useful generalization of these?
% I.e., something that has the benefits of all but
% few downsides?

\section{Comparison to existing models}

\section{Contributions}

% ??? we made a speech recognition thing

% ??? we made a synthesizer

% ??? we made a neural control method for synthesizers

% ??? we put them together

% ??? mostly, we've integrated existing parts in a large scale model

% ??? none of the models by themselves are terribly groundbreaking,
% and can use some work to fix up,
% but the contextual framework of Sermo
% is a large undertaking,
% but we believe it can provide the basis
% for a ton of worthwhile future research.

% ??? additionally, while my stuff is applied to speech,
% the various networks and ideas (the sequencer network,
% trajectory classification, etc) are useful/necessary
% for dealing with any temporally varying stuff in spaun, e.g. gestures, etc.
% I.e., temporal working memories are likely to be
% a big part of future iterations of Spaun.

\subsection{Contributions to computer science}

% Mostly we've talked about brain modeling,
% arguably part of the domain of neuroscience,
% and control, which is traditionally an engineering topic.
% But, this is a CS thesis, so there should be some CS contributions.

% somewhat disappointing: the point of Sermo
% and the goal of the thesis is
% to close the loop, and create an integrated system
% that speaks and hears itself.
% the missing connection to close the loop
% is to decode production information (vocal tract gestures)
% from auditory information (power spectra and cepstra).
% we hope to present this as a machine learning problem
% (benchmark) that can help make machine learning progress

\section{Predictions}

% need more cortex for consonants than vowels?

% - lots of people talk about spectro-temporal features,
%   gabor filters across time and space, etc.
%   Those are useful models, but not directly implementable
%   in neurons; we have to have everything available at
%   the same timestep, so derivatives make more sense

\subsection{Mapping of model to brain areas}

% In the background we discussed the brain areas
% involved in speech recognition and production,
% in part to place connectivity constraints
% on the model as a whole.
% However, in order to generate testable predictions,
% we can also impose a mapping from
% the neural structures in the model
% to speech-related brain regions.

% ??? do, and make predictions based on this.

% ??? possible issues:

% ??? cepstral coefficients currently require activity
% from all frequencies. But, this may not be anatomically
% consistent. Could be possible, instead, to do this
% hierarchically.

\section{Limitations}

% iDMPs aren't complete yet; we recover
% the canonical system state, but more work
% needs to be done to get the point attractor

\section{Future work}

% ??? add bursting neurons in both production
% and recognition systems for better timing

% ??? more sophisticated classification mechanisms
% in classification section

% ??? add speed control to the DMPs;
% pretty easy, just add a dimension to the oscillator.

% ??? neural implementation of mapping
% from gesture sequence to articulator positions

% Summary of things from other sections:

% \subsection{Recognition system}

% How to deal with semivowels, semiconsonants and glides?
% Should it just be one monolithic phoneme detector?

% Represent prosody in a second feature layer (hierarchical organization).

% Learn all of this stuff rather than optimizing for it.

% How to set baseline pitch and volume on the fly?

% Can we use artificial cochleas as is?

% \subsection{Synthesis system}

% We can use this synthesis system to explore
% important phonological questions.
% For example, syllabic consonants
% Do they sound right as separate syllables?
% Or do they sound right as protracted versions
% of the analogous syllable with the vowel included?
% Can these be distinguished from one another?

% We haven't done forward model prediction
% to determine how to adapt
% when things go wrong.
% However, since we've implemented everything
% with the NEF and SPA,
% we can just use the REACH model.

% ??? there's also a bunch of prosodic stuff;
% this could be added (see comments below)

% Aside from phonemes, we also represent pitch and volume.
% Neither of these features are useful
% as an absolute quantity---most humans are poor
% at judging absolute pitch and volume ???cite---so
% we aim to represent relative pitch and volume.
% In both of these cases,
% we must determine a baseline pitch or volume,
% and a method for comparing the current
% pitch or volume to the baseline.

% In speech, baseline pitch is primarily determined
% by speaker identity.
% Each speaker has a baseline pitch,
% determined in part by the shape of their vocal folds,
% so it is likely that we learn
% and remember the baseline pitch
% of speakers that we interact with frequently.
% On the other hand,
% changes in baseline pitch
% (e.g., through illnesses affecting the vocal tract)
% require little to no adaptation period,
% so the mechanism through which we determine
% a speaker's baseline pitch
% is likely to be relatively simple
% and flexible.
% In this work,
% we will not posit a neural mechanism
% for determining baseline pitch,
% and will instead compute it
% from the data offline and provide it as input.

% Given the baseline pitch as input,
% relative pitch will be computed
% primarily through neural inhibition.
% See section ???implementation
% for how this is accomplished.

% Unlike pitch, baseline volume
% is not speaker specific;
% it is easy to notice when a speaker
% talks louder or quieter
% than the norm.
% ???baseline is the overall activity
% of a long-timescale filtered version
% of the current moment's power spectrum?
% ???relative volume is that minus a
% short-timescale filtered version?
% So it's basically just a derivative?
% ???not sure if we even care about volume
% to be honest...
% % http://www.sengpielaudio.com/calculator-loudness.htm

% \section{Syllable consolidation}

% ??? I didn't get to this,
% but it can definitely be done.

% Go from vocal tract gesture score
% to audio signal, in neurons.

% Learn a syllable recognition
% and a syllable production
% network from this.

% Contrast this to syllable production because it's
% an infrequently voiced syllable.

% \subsection{Bootstrapped syllable learning}

% ??? A full picture of speech development involves:
% learning vocal tract gestures
% through reinforced motor babbling,
% learning basic syllables
% through mimicry?,
% and scaling up to the elements of this model.
% While we think that our model
% is a useful starting point for such
% a developmental model,
% as it provides an end-target,
% we don't claim to do this kind of structural learning.
% However, we believe that error-based learning
% could result in this kind of system;
% to show that this is a possibility,
% we consider a minimal learning situation:
% learning voice a new syllable
% given a set of existing syllables.
% This type of thing probably happens
% in second language acquisition,
% when novel combinations of phonemes
% are encountered,
% or even during first language acquisition
% as pronunciation is refined over
% the course of one's life;
% words that were once awkward combinations
% of many syllables are compressed into
% nearly equivalent sequences of fewer,
% more complex syllables.

% As opposed to conversational shadowing,
% which highlights the high-level strengths
% of the integrated speech system,
% syllable learning highlights
% the low-level strengths of this system.
% Syllable learning involves
% learning a novel set of gestures
% and associated articulator trajectories
% that will voice a syllable
% that is encountered for the first time.

% We call our syllable learning system ``bootstrapped''
% because we assume that our system
% has an existing repertoire of syllables
% that it is already able to voice.
% These existing syllables will be
% used in learning the new syllable.
% Bootstrapped syllable learning contrasts with
% the type of syllable learning
% done as an infant and toddler,
% which uses reinforced speech babbling
% to learn novel syllables.
% While we believe that learning syllables
% from babbling is an important research direction
% that can be explored in this system,
% it has also been explored in many other systems
% (???cite Diva etc),
% and so we have chosen to leave this type of learning
% for future work.

% The bootstrapped syllable learning system
% learns new syllables in three steps.

% \begin{enumerate}
% \item Initialize the new syllable from the most
%   similar existing syllable.
%   For example, when learning to voice
%   the syllable /ba/, the system should
%   start from the syllable /fa/ if it is known.
% \item Swap compatible speech gestures.
%   For example, vowel producing gestures
%   would be compatible, allowing for modifying
%   a /ba/ to a /bu/, and so on.
%   The choice of which gesture to swap and
%   how to swap it will be informed by
%   ???figure out.
% \item Fine-tune the voiced syllable
%   until it can be recognized as the
%   syllable to be learned.
%   ???more
% \end{enumerate}

% ??? NB: the oscillator being learned should exist outside of
% the normal speech system so that they can both run
% at the same time, but there should be a switch kind of thing
% to put the new syllable through.

% ??? hypothesis (not sure where to put this):
% the mapping between phoneme to gesture
% is such that is not advantageous
% to represent in the synthesis
% (maybe also recognition?) system(s).
% Therefore, it may be the case that
% not all people have phoneme representations.
% However, we propose that phonemes
% are a useful construct for learning to voice
% novel syllables in a second-language learning situation.

% These steps require several systems
% that have already been implemented
% in the recognition and synthesis systems separately;
% for example, the ability to compare
% voiced syllables to those already known
% is one of the primary goals of the recognition system itself,
% so it can be leveraged when trying to learn new syllables.
% However, these steps also point to new systems
% that must be implemented.
% First, the system requires a method
% to transfer a syllable-producing function
% from one ensemble to another.
% Second, some knowledge of which gestures
% are compatible must be built into the system.
% Finally, ???fine-tuning.

% It has been shown that in second-language learning,
% slowing down the syllable can have
% a significant increase in learning effectiveness
% ???cite.
% We believe that our system emulates
% how a native speaker of one language
% would learn to voice novel syllables in a second language.
% Therefore, we predict that slowing down
% the speed at which the syllables are heard
% and uttered will improve ???learning speed
% and / or quality of learned syllable.

% ??? Mental syllabary (1-s2.0-S009394X...pdf)
% supports our architecture

\chapter{Conclusions and future work}

%% ~5-10 pages

%% - State what you've done and what you've found
%% - Summarize contributions (achievements and impact)
%% - Outline open issues/directions for future work

\section{Predictions}

need more cortex for consonants than vowels?

\appendix

\chapter*{APPENDICES}
\addcontentsline{toc}{chapter}{APPENDICES}

% example appendix
%% \chapter[shorttitle]{Long Title}
%% \label{AppendixA}

%% - Include technical material that would disrupt the flow of the thesis.
%% - Included for curious or disbelieving readers

\cleardoublepage
\phantomsection
\renewcommand*{\bibname}{References}

\addcontentsline{toc}{chapter}{\textbf{References}}

\nocite{*}
\bibliographystyle{plainnat}
\bibliography{phd}


\end{document}

%% Before handing in a copy of what you've written you should proof
%% read it and make corrections yourself. Be critical of your own work
%% when you do this. You should think not only about syntax and
%% grammar but about the structure of the document and whether or not
%% your are making good arguments and whether or not someone else will
%% be able to follow and believe what you are saying. You should
%% repeat this process a large number of times before you hand in a
%% copy. Far too many people type something in, print it out and hand
%% it in. If this is the case you as a student are not doing your
%% job. It is not your supervisor's job to write your thesis.

%% Try to aim for around 100 pages or less.

%% Including a glossary or list of acronyms may be helpful.

%% Start thinking about what your contributions are early on.
%% - How is what you are doing interesting and important?
%% - How will it make the world a better place?
%% - What are you doing or discovering that hasn't already been done
%%   or isn't already known?

%% For many people it is best to start by writing the "guts" of the
%% thesis, Chapters 3, 4 and 5. In some cases the results and
%% conclusions may not be known (or may change) while doing these
%% chapters.

%% Chapters 3, 4 and 5 can take on different forms depending on the
%% thesis and approaches being used.

%% Sometimes design, implementation and performance are subsections
%% within chapters and the chapters are broken down by other criteria.

%% Remember (especially those doing experiments) that you must include
%% enough detail in your thesis so that someone else could read your
%% thesis and reproduce your results - without ever talking to you.

%% The word performance is by itself quite meaningless. Stating that
%% you've improved performance significantly does not tell the reader
%% anything. There are problems with the word performance and the word
%% improved. Remember that there are often a number of different
%% performance metrics that can be applied to a system. Instead of
%% using the word performance state precisely what performance metric
%% is improved. Also improved may also be potentially ambiguous. State
%% precisely what you mean. For example: The mean response time has
%% been decreased by 20%. Peak bandwidth has been increased by 40%.

%% Try to get an outline and style guidelines from someone else for
%% the system you use for formatting your thesis.

%% All figures included should add to the work. As such, there should
%% be text included that refers to the figures (preferably before the
%% figure is encountered). The text should explain what the reader
%% should get from the figure - what are they supposed to notice and
%% what is the figure explaining. Often people just include a figure
%% with no reference to the figure and no explanation of what the
%% figure is for - if the figure was not included no one would notice
%% (this is not a good approach). Note that when referring to a figure
%% or a section by name they should be capitalized as in -- Figure 3
%% shows the architecture of our system or in Section 4 we describe
%% the experimental methodology.
